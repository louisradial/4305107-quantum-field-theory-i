\documentclass[english]{artigo}

\author{Louis Bergamo Radial\\8992822}
\title{Quantum Field Theory I\\Problem set I}
\begin{document}
   \maketitle
   \begin{theorem}{Noether's theorem}{noether}
      Let \(x^\mu \mapsto \tilde{x}^\mu = x^\mu + \Delta x^\mu\) and \(\phi_j(x) \mapsto \tilde{\phi}_j(\tilde{x}) = \phi_j(x) + \Delta\phi_j(x)\) be an infinitesimal transformation that leaves the action \(S = \int \dln4x \mathcal{L}\) invariant. Then the current \(j^\mu = \diffp{\mathcal{L}}{(\partial_\mu \phi_j)} \delta \phi_j + \mathcal{L} \Delta x^\mu\) satisfies the continuity equation \(\partial_\mu j^\mu = 0\), where \(\delta \phi_j(x) = \tilde{\phi}_j(x) - \phi_j(x)\).
   \end{theorem}
   \begin{proof}
      Notice we have
      \begin{align*}
         \Delta\phi_j(x) = \tilde{\phi_j}(\tilde{x}) - \phi_j(x) &= \tilde{\phi_j}(x) + \Delta x^\mu \partial_\mu \tilde{\phi_j}(x) - \phi_j(x) = \Delta x^\mu \partial_\mu \tilde{\phi_j}(x) + \delta \phi_j(x)\\
                                                                 &= \Delta x^\mu \partial_\mu \left[\phi_j(x) + \Delta \phi_j(x)\right] + \delta\phi_j(x)\\
                                                                 &= \Delta x^\mu \partial_\mu \phi_j(x) + \delta\phi_j(x)
      \end{align*}
      and
      \begin{equation*}
         \tilde{\partial}_\mu = \diffp{x^\nu}{\tilde{x}^\mu} \partial_\nu = \left(\delta\indices{^\nu_\mu} - \partial_\mu \Delta x^\nu\right)\partial_\nu
      \end{equation*}
      in first order of \(\Delta x^\mu\) and \(\Delta \phi_j(x)\). This yields
      \begin{align*}
         \Delta (\partial_\mu \phi_j)(x) = \tilde{\partial}_\mu \tilde{\phi_j}(\tilde{x}) - \partial_\mu \phi_j(x) 
         &= \left(\delta\indices{^\nu_\mu} - \partial_\mu \Delta x^\nu\right) \partial_\nu \left[\phi_j(x) + \Delta \phi_j(x)\right] - \partial_\mu \phi_j(x)\\
         &= \partial_\mu \Delta\phi_j(x) - (\partial_\mu \Delta x^\nu) \partial_\nu \phi_j(x)\\
         &= \partial_\mu \left[\delta \phi_j(x)\right] + \partial_\mu \left[\Delta x^\nu \partial_\nu \phi_j(x)\right] - (\partial_\mu \Delta x^\nu) \partial_\nu \phi_j(x)\\
         &= \partial_\mu \left[\delta \phi_j(x)\right] + \Delta x^\nu \partial_\mu \partial_\nu \phi_j(x)\\
         &= \partial_\mu \left[\delta \phi_j(x)\right] + \Delta x^\nu \partial_\nu \partial_\mu \phi_j(x)
      \end{align*}
      in the same first order approximation. Let us denote 
      \begin{equation*}
         \tilde{\mathcal{L}} = \mathcal{L}[\tilde{\phi}(\tilde{x}), \tilde{\partial}\tilde{\phi}(\tilde{x}), \tilde{x}]\quad\text{and}\quad
         \mathcal{L} = \mathcal{L}[\phi(x), \partial \phi(x), x], 
      \end{equation*}
      then
      \begin{align*}
         \tilde{\mathcal{L}} &= \mathcal{L}\left[\phi(x) + \Delta \phi(x), \partial \phi(x) + \Delta \partial\phi(x), x + \Delta x\right]\\
                             &= \mathcal{L} + \diffp{\mathcal{L}}{\phi_j} \Delta \phi_j(x) + \diffp{\mathcal{L}}{(\partial_\mu \phi_j)} \Delta \partial_\mu \phi_j(x) + \diffp{\mathcal{L}}{x^\mu} \Delta x^\mu\\
                             &= \mathcal{L} + \diffp{\mathcal{L}}{\phi_j} \left[\Delta x^\mu \partial_\mu \phi_j(x) + \delta\phi_j(x)\right] + 
                             \diffp{\mathcal{L}}{(\partial_\nu \phi_j)} \left\{\partial_\nu \left[\delta \phi_j(x)\right] + \Delta x^\mu \partial_\mu \partial_\nu \phi_j(x)\right\}
                             + \diffp{L}{x^\mu} \Delta x^\mu\\
                             &=  \mathcal{L} + \diff{\mathcal{L}}{x^\mu} \Delta x^\mu + \diffp{L}{\phi_j} \delta \phi_j(x) + \diffp{\mathcal{L}}{(\partial_\nu \phi_j)} \partial_\nu\left[\delta \phi_j(x)\right]\\
                             &=  \mathcal{L} + \diff{\mathcal{L}}{x^\mu} \Delta x^\mu + \delta \mathcal{L},
      \end{align*}
      where we defined \(\delta \mathcal{L} = \diffp{\mathcal{L}}{\phi_j} \delta \phi_j(x) + \diffp{\mathcal{L}}{(\partial_\nu \phi_j)} \partial_\nu\left[\delta \phi_j(x)\right]\). Notice if \(\phi_j\) satisfies the equations of motion,
      \begin{equation*}
         \diffp{\mathcal{L}}{\phi_j} = \partial_\sigma \diffp{\mathcal{L}}{(\partial_\sigma \phi_j)},
      \end{equation*}
      we have
      \begin{equation*}
         \delta \mathcal{L} = \partial_\nu \left(\diffp{\mathcal{L}}{(\partial_\nu \phi_j)}\right) \delta \phi_j(x) + \diffp{\mathcal{L}}{(\partial_\nu \phi_j)} \partial_\nu \left[\delta \phi_j(x)\right] = \partial_\nu\left[\diffp{\mathcal{L}}{(\partial_\nu \phi_j)} \delta\phi_j(x)\right] = \diff*{\left[\diffp{\mathcal{L}}{(\partial_\nu \phi_j)} \delta\phi_j(x)\right]}{x^\nu}.
      \end{equation*}

      Recall that if a matrix \(A\) has (complex) eigenvalues \(\sigma(A),\) then the matrix \(\unity + A\) has eigenvalues \(\sigma(\unity + A) = \setc{1 + a}{a \in \sigma(A)}\). Therefore, if \(\varepsilon\) is an infinitesimal parameter, we have
      \begin{equation*}
         \det(\unity + \varepsilon A) = \prod_{\lambda \in \sigma(\unity + \varepsilon A)} \lambda = \prod_{a \in \sigma(A)} (1 + \varepsilon a) = 1 + \varepsilon \sum_{a \in \sigma(A)} a = 1 + \varepsilon \Tr(A)
      \end{equation*}
      up to linear order in \(\varepsilon\). As a result, the Jacobian of the infinitesimal transformation is such that
      \begin{equation*}
         \dln4{\tilde{x}} = \left(1 + \partial_\mu\Delta x^\mu\right) \dln4x,
      \end{equation*}
      thus the difference in action is given by
      \begin{align*}
         \Delta S &= \int \dln4{\tilde{x}} \tilde{\mathcal{L}} - \int \dln4x \mathcal{L}\\
                  &= \int \dln4x \left\{\left(1 + \partial_\sigma \Delta x^\sigma\right)\left[\mathcal{L} + \diff{\mathcal{L}}{x^\mu}\Delta x^\mu + \delta \mathcal{L}\right] - \mathcal{L}\right\}\\
                  &= \int \dln4x \left(\mathcal{L} \partial_\sigma \Delta x^\sigma + \diff{\mathcal{L}}{x^\mu}\Delta x^\mu + \delta \mathcal{L}\right)\\
                  &= \int \dln4x \left[\diff*{\left(\mathcal{L} \Delta x^\sigma\right)}{x^\sigma} + \delta \mathcal{L}\right].
      \end{align*}
      Then, for a solution of the equations of motion, this yields
      \begin{equation*}
         \Delta S = \int \dln4x \diff*{\left[\diffp{\mathcal{L}}{(\partial_\sigma \phi_j)} \delta \phi_j(x) + \mathcal{L} \Delta x^\sigma\right]}{x^\sigma},
      \end{equation*}
      therefore the integrand must be equal to zero if the infinitesimal transformation is a symmetry, as desired.
   \end{proof}

   % vim: spl=en
\begin{problem}{Feynman rules for an interacting theory of scalar field and charged scalar field}{p1}
   Consider a complex scalar field \(\Phi\) and a real one \(S\). The Lagrangian density describing this model is
   \begin{equation*}
         \mathcal{L} = \partial_\nu \conj{\Phi} \partial^\nu \Phi - \mu^2 \conj{\Phi}\Phi - \frac{\lambda}{4} (\conj{\Phi}\Phi)^2 + \frac12 \partial_\nu S\partial^\nu S - \frac12 m^2 S^2 - \frac{g}{4!} S^4 - \Delta \conj{\Phi}\Phi S,
   \end{equation*}
   where \(\mu, \lambda, m, g,\) and \(\Delta\) are constants.
   \begin{enumerate}[label=(\alph*)]
       \item Obtain the Feynman rules for this model in the coordinate space.
       \item Obtain the Feynman rules for this model in the momentum space.
   \end{enumerate}
\end{problem}
\begin{proof}[Solution]
   We'll write the free action as
   \begin{equation*}
      S_0[\Phi, \conj{\Phi}, S] = \int_{\mathbb{R}^4}{\dln4x \left[\partial_\nu \conj{\Phi} \partial^\nu \Phi - (\mu^2 - i \epsilon) \conj{\Phi}\Phi + \frac12 \partial_\nu S\partial^\nu S - \frac12 (m^2 - i \epsilon) S^2\right]},
   \end{equation*}
   and the interaction action as
   \begin{equation*}
      S_\mathrm{int}[\Phi, \conj{\Phi}, S] = -\int_{\mathbb{R}^4}{\dln4x\left(\frac{\lambda}{4} (\conj{\Phi}\Phi)^2  + \frac{g}{4!} S^4 + \Delta \conj{\Phi}\Phi S\right)},
   \end{equation*}
   then the generating functional is
   \begin{align*}
      Z[\conj{J}, J, K] &= \int{\dlp{S} \dlp{\Phi}\dlp{\conj{\Phi}} e^{iS_\mathrm{int}[\Phi, \conj{\Phi}, S]} e^{i S_0[\Phi, \conj{\Phi}, S]} \exp\left[i\int_{\mathbb{R}^4}{\dln4x \left(\conj{J}\Phi + J \conj{\Phi} + K S\right)}\right]}\\
                        &= \int{\dlp{S} \dlp{\Phi}\dlp{\conj{\Phi}} \exp\left(iS_\mathrm{int}\left[\frac{1}{i} \diff.d.{}{\conj{J}}, \frac1i \diff.d.{}{J}, \frac1i \diff.d.{}{K}\right]\right) e^{i S_0[\Phi, \conj{\Phi}, S] + i\int_{\mathbb{R}^4}{\dln4x \left(\conj{J}\Phi + J \conj{\Phi} + K S\right)}}}\\
                        &= \exp\left(iS_\mathrm{int}\left[\frac{1}{i} \diff.d.{}{\conj{J}}, \frac1i \diff.d.{}{J}, \frac1i \diff.d.{}{K}\right]\right) Z_0[\conj{J}, J, K]\\
                        &= \sum_{k = 0}^\infty{\frac{(-i)^k}{k!} \left\{\int_{\mathbb{R}^4}{\dln4z\left[\frac{\lambda}{4} \left(\diff.d.{}{J_z} \diff.d.{}{\conj{J}_z}\right)^2  + \frac{g}{4!} \left(\diff.d.{}{K_z}\right)^4 + i\Delta \diff.d.{}{J_z}\diff.d.{}{\conj{J}_z} \diff.d.{}{K_z}\right]}\right\}^kZ_0[\conj{J}, J, K]},
   \end{align*}
   where \(Z_0[\conj{J},J,K]\) is the generating functional for the free theory, given by
   \begin{equation*}
      Z_0[\conj{J},J,K] = Z_0[0] \exp\left[-\int_{\mathbb{R}^4}{
            \dln4{x}\int_{\mathbb{R}^4}{
               \dln4{y} \left(\frac12 K(x)D_S(x-y)K(y) + \conj{J}(x) D_\Phi(x - y) J(y)\right)
            }
      }\right],
   \end{equation*}
   where
   \begin{equation*}
      D_S(x-y) = \int_{\mathbb{R}^4}\frac{\dln4p}{(2\pi)^4} \frac{ie^{-ip(x-y)}}{p_\nu p^\nu - m^2 + i \epsilon}
      \quad\text{and}\quad
      D_\Phi(x-y) = \int_{\mathbb{R}^4}\frac{\dln4p}{(2\pi)^4} \frac{ie^{-ip(x-y)}}{p_\nu p^\nu - \mu^2 + i \epsilon}
   \end{equation*}
   are the propagators of each field, satisfying the differential equations
   \begin{equation*}
      (\square_x + m^2) D_S(x-y) = -i \delta(x - y)
      \quad\text{and}\quad
      (\square_x + \mu^2) D_\Phi(x - y) = -i \delta(x-y).
   \end{equation*}

   For each interaction term we'll compute at first order the relevant correlation function to obtain the vertices of the interacting theory. In first order of \(g,\) we consider
   \begin{align*}
      G_g^{(4)}(x_1, x_2, x_3,x_4) &= \bra{0}T\left\{S(x_1)S(x_2) S(x_3)S(x_4)\right\}\ket{0}\\
                                   &= \frac{\braket{0}{0}}{Z[0]} \diff.d.{Z[\conj{J}, J, K]}{K_{x_1},K_{x_2},K_{x_3},K_{x_4}}[\substack{\conj{J} = 0\\J=0\\K=0}]\\
                                   &= \frac{\braket{0}{0} Z_0[0]}{Z[0]}\diff.d.*{\left[1 - \frac{ig}{4!}\int_{\mathbb{R}^4}{\dln4z \left(\diff.d.{}{K_z}\right)^4}\right]e^{-\frac12 KD_SK}}{K_{x_1},K_{x_2},K_{x_3},K_{x_4}}[K=0],
   \end{align*}
   where we denote \(KD_SK \equiv \int_{\mathbb{R}^4} \dln4x \int_{\mathbb{R}^4} \dln4y K(x) D_S(x-y) K(y)\) as shorthand. The prefactor \(\frac{\braket{0}{0} Z_0[0]}{Z[0]}\) is responsible for removing disconnected diagrams, then we need only consider the fourth order expansion of \(e^{-\frac12 KD_SK},\) yielding
   \begin{align*}
      G_g^{(4)} &= \frac{(-ig)}{2^4(4!)^2} \diff.d.*{ \int_{\mathbb{R}^4}\dln4z \left(\diff.d.{}{K_z}\right)^4 \left[\int_{\mathbb{R}^4} \dln4x \int_{\mathbb{R}^4} \dln4y K(x)D_S(x - y)K(y)\right]^4 }{K_{x_1},K_{x_2},K_{x_3},K_{x_4}}[\substack{K=0\\\text{connected}}]\\
                &= (-ig)\frac{8 \times 6 \times 4 \times 2 \times 4!}{2^4 (4!)^2} \int_{\mathbb{R}^4} \dln4z D_S(x_1 - z) D_S(x_2 - z) D_S(x_3 - z) D_S(x_4 - z)\\
                &= (-ig) \int_{\mathbb{R}^4} \dln4z D_S(x_1 - z) D_S(x_2 - z) D_S(x_3 - z) D_S(x_4 - z),
   \end{align*}
   where we have used that, in order to get the connected diagrams, the internal point functional derivatives must act on different \(KD_SK,\) yielding the \(2^4 4!\) symmetry factor, and then the other functional derivatives yield the additional \(4!\) symmetry factor. In momentum space, we have
   \begin{align*}
      \tilde{G}_g^{(4)}(p_1, p_2, p_3, p_4) &= \int_{\mathbb{R}^4} \dln4{x_1}\int_{\mathbb{R}^4} \dln4{x_2}\int_{\mathbb{R}^4} \dln4{x_3}\int_{\mathbb{R}^4} \dln4{x_4}e^{i(p_1 x_1 + p_2 x_2 + p_3x_3 + p_4 x_4)} G^{(4)}_g(x_1, x_2,x_3,x_4)
   \end{align*}
   then as
   \begin{align*}
      \int_{\mathbb{R}^4} \dln4x  e^{ipx} D(x - z) &= \int_{\mathbb{R}^4} \frac{\dln4k}{(2\pi)^4} \int_{\mathbb{R}^4}\dln4x e^{ikz} \frac{ie^{ix(p - k)}}{k_\nu k^\nu - M^2 + i \epsilon}\\
                                                   &= \int_{\mathbb{R}^4} \dln4k \frac{ie^{ikz}}{k_\nu k^\nu - M^2 + i \epsilon} \delta(p - k)\\
                                                   &= \frac{i e^{ipz}}{p_\nu p^\nu - M^2 + i \epsilon},
   \end{align*}
   we get
   \begin{align*}
      \tilde{G}_g^{(4)}(p_1, p_2, p_3, p_4) &= -ig \int_{\mathbb{R}^4} \dln4z \prod_{\ell = 1}^4{\frac{i e^{ip_\ell z}}{p_\ell^2 - m^2 + i \epsilon}}\\
                                            &= -ig (2\pi)^4 \delta(p_1 + p_2 + p_3 + p_4) \prod_{\ell = 1}^4{\frac{i}{p_\ell^2 - m^2 + i \epsilon}},
   \end{align*}
   where \(p_\ell^2 = (p_\ell)_\nu (p_\ell)^\nu.\)

   We now consider the first order of \(\lambda,\) following similar arguments. The correlation function of interest is the four-point function
   \begin{align*}
      G^{(4)}_\lambda(x_1, x_2,x_3,x_4) &= \bra{0}T\left\{\conj{\Phi}(x_1)\Phi(x_2)\conj{\Phi}(x_3)\Phi(x_4)\right\}\ket{0}\\
                                        &= \frac{\braket{0}{0} Z_0[0]}{Z[0]}\diff.d.*{\left[1 - \frac{i\lambda}{4}\int_{\mathbb{R}^4}{\dln4z \left(\diff.d.{}{J_z}\diff.d.{}{\conj{J}_z}\right)^2}\right]e^{-J^*D_\Phi J}}{J_{x_1},\conj{J}_{x_2},J_{x_3},\conj{J}_{x_4}}[\substack{\conj{J} = 0\\J=0}]\\
                                        &= \frac{(-i \lambda)}{2^2 4!}\diff.d.*{\left[\int_{\mathbb{R}^4}{\dln4z \left(\diff.d.{}{J_z}\diff.d.{}{\conj{J}_z}\right)^2}\right]\left(J^*D_\Phi J\right)^4}{J_{x_1},\conj{J}_{x_2},J_{x_3},\conj{J}_{x_4}}[\substack{\conj{J} = 0\\J=0\\\text{connected}}]\\
                                        &= -i\lambda\frac{4 \times 3 \times 2 \times 2 \times 2}{2^2 4!}\int_{\mathbb{R}^4} \dln4z D_\Phi(z - x_1) D_\Phi(x_2 - z) D_\Phi(z - x_3) D_\Phi(x_4 - z)\\
                                        &= -i \lambda \int_{\mathbb{R}^4} \dln4z\dln4z D_\Phi(z - x_1) D_\Phi(x_2 - z) D_\Phi(z - x_3) D_\Phi(x_4 - z),
   \end{align*}
   where the symmetry factor arises from a similar argument as before. In momentum space, we have
   \begin{align*}
      \tilde{G}^{(4)}_{\lambda}(p_1, p_2, p_3, p_4) 
                                              = -i \lambda (2\pi)^4 \delta(p_1 + p_3 + p_2 + p_4) \prod_{\ell = 1}^{4}{\frac{i}{p_{\ell}^2 - \mu^2 + i \epsilon}},
   \end{align*}
   since we have \(D_\Phi(x - y) = D_\Phi(y -x),\) or, explicitly,
   \begin{align*}
      \int_{\mathbb{R}^4} \dln4x  e^{ipx} D(z - x) &= \int_{\mathbb{R}^4} \frac{\dln4k}{(2\pi)^4} \int_{\mathbb{R}^4}\dln4x e^{-ikz} \frac{ie^{ix(p + k)}}{k_\nu k^\nu - M^2 + i \epsilon}\\
                                                   &= \int_{\mathbb{R}^4} \dln4k \frac{ie^{-ikz}}{k_\nu k^\nu - M^2 + i \epsilon} \delta(p + k)\\
                                                   &= \frac{i e^{ipz}}{p_\nu p^\nu - M^2 + i \epsilon}.
   \end{align*}

   Finally, we consider the first order of \(\Delta.\) The correlation function that we consider is
   \begin{align*}
      G_{\Delta}^{(3)}(x_1,x_2,x_3) &= \bra{0}T\left\{S(x_1)\conj{\Phi}(x_2) \Phi(x_3)\right\}\ket{0}\\
                                    &= i\frac{\braket{0}{0}Z_0[0]}{Z[0]} \diff.d.*{\left[1 + \Delta \int_{\mathbb{R}^4}\dln4z \diff.d.{}{J_z}\diff.d.{}{\conj{J}_z} \diff.d.{}{K_z}\right] e^{-\frac12 KD_SK - \conj{J}D_\Phi J}} {K_{x_1},J_{x_2},\conj{J}_{x_3}}[\substack{K=0\\J=0\\\conj{J}=0}]\\
                                    &= \frac{(-i\Delta)}{4} \diff.d.*{\int_{\mathbb{R}^4}\dln4z \diff.d.{}{J_z}\diff.d.{}{\conj{J}_z} \diff.d.{}{K_z}\left(KD_S K\right)\left(\conj{J}D_\Phi J\right)^2} {K_{x_1},J_{x_2},\conj{J}_{x_3}}[\begin{subarray}{l}K=0\\J=0\\\conj{J}=0\\\text{connected}\end{subarray}]\\
                                    &= -i \Delta \frac{2 \times 2}{4} \int_{\mathbb{R}^4} \dln4z D_S(x_1 - z) D_\Phi(x_2 - z) D_\Phi(x_3 - z)\\
                                    &= -i \Delta \int_{\mathbb{R}^4} \dln4z D_S(x_1 - z) D_\Phi(x_2 - z) D_\Phi(x_3 - z).
   \end{align*}
   As before, in momentum space we have
   \begin{equation*}
      \tilde{G}_{\Delta}^{(3)}(p_1, p_2, p_3) = - i \Delta (2\pi)^4 \delta(p_1 + p_2 + p_3) \frac{i}{p_1^2 - m^2 + i \epsilon}\prod_{\ell = 2}^3 \frac{i}{p_\ell^2 - \mu^2 + i \epsilon}.
   \end{equation*}

   We now compile the Feynman rules for our theory. In coordinate space, 
   \begin{enumerate}[label=(\alph*)]
      \item write down the external points \(x_i\) at which the correlation function is evaluated and draw the corresponding particle line from them:  \feynmandiagram [inline=(a.base),small, horizontal=a to b] {a [particle=\(x_i\)] -- [scalar] b}; for a scalar, and \feynmandiagram [inline=(a.base),small, horizontal=a to b] {a [particle=\(x_i\)] -- [charged scalar] b}; for a charged scalar;
      \item for each vertex of type \(g,\) multiply by a factor \(-ig\) and integrate over the internal point \(z_j^g,\) and do analogously for the other vertex types;
      \item a scalar line \feynmandiagram [inline=(a.base),small, horizontal=a to b] {a [particle=\(y_i\)] -- [scalar] b [particle=\(y_j\)]}; corresponds to the propagator \(D_S(y_i - y_j)\);
      \item a charged scalar line \feynmandiagram [inline=(a.base),small, horizontal=a to b] {a [particle=\(y_i\)] -- [charged scalar] b [particle=\(y_j\)]}; corresponds to the propagator \(D_\Phi(y_i - y_j)\);
      \item multiply by the symmetry factor.
   \end{enumerate}
   In momentum space,
   \begin{enumerate}[label=(\alph*)]
       \item for each vertex multiply by the appropriate coupling factor, as in the coordinate space;
       \item internal scalar lines \feynmandiagram [inline=(b.base),small, horizontal=a to b] {a -- [scalar, momentum=\(p\)] b}; correspond to \(\frac{i}{p^2 - m^2 + i \epsilon}\);
       \item internal charged scalar lines \feynmandiagram [inline=(b.base),small, horizontal=a to b] {a -- [charged scalar, momentum=\(p\)] b}; correspond to \(\frac{i}{p^2 - \mu^2 + i \epsilon}\);
       \item conserve momentum at each vertex with \((2\pi)^4\delta(\sum_i q_i)\), where \(q_i\) are the incoming momenta at the vertex;
       \item integrate over the undetermined momenta \(\int_{\mathbb{R}^4}\frac{\dln4k}{(2\pi)^4};\)
       \item multiply by the symmetry factor.
   \end{enumerate}
\end{proof}

   % vim: spl=en
\begin{problem}{Generating functional for a free complex scalar field}{p2}
    Obtain the generating functional for a free complex scalar field.
\end{problem}
\begin{proof}[Solution]
    We recall the Feynman propagator for a free complex scalar field,
    \begin{equation*}
        D_F(x,y) = i \int_{\mathbb{R}^4}\frac{\dln4p}{(2\pi)^4} \frac{e^{-ip(x-y)}}{p_\mu p^\mu - m^2 + i \epsilon},
    \end{equation*}
    and consider the differential operator \(L_x = \square_x + m^2 - i \epsilon,\) then \(iD_F(x,y)\) is a Green function for this operator as we have
    \begin{align*}
        L_x D_F(x,y) &= i \int_{\mathbb{R}^4} \frac{\dln4p}{(2\pi)^4} \frac{L_x e^{-i p(x-y)}}{p_\mu p^\mu - m^2 + i \epsilon}\\
                     &= i \int_{\mathbb{R}^4} \frac{\dln4p}{(2\pi)^4} \frac{-p_\nu p^\nu + m^2 - i \epsilon}{p_\mu p^\mu - m^2 + i \epsilon} e^{-ip(x-y)}\\
                     &= -i \int_{\mathbb{R}^4} \frac{\dln4p}{(2\pi)^4} e^{-ip(x-y)}\\
                     &= -i \delta(x-y).
    \end{align*}
    For the linear coupled Lagrangian density, \(\mathcal{L}^{J,J^*} = \partial_\mu \varphi^* \partial^\mu \varphi - (m^2 - i \epsilon) \varphi^* \varphi + J^* \varphi + J \varphi^*,\) the equations of motion are \(L_x \varphi(x) = J(x)\) and \(L_x \varphi^*(x) = J^*(x),\) hence the classical solutions are
    \begin{equation*}
        \psi(x) = i \int_{\mathbb{R}^4} \dln4{y}D_F(x,y) J(y)
        \quad\text{and}\quad
        \psi^*(x) = i \int_{\mathbb{R}^4} \dln4{y}D_F(x,y) J^*(y).
    \end{equation*}

    We'll obtain the generating functional
    \begin{equation*}
        Z[J,J^*] = \int \dlp{\varphi(x)} \dlp{\varphi^*(x)} \exp\left(i \int_{\mathbb{R}^4} \dln4x \mathcal{L}^{J,J^*}\right),
    \end{equation*}
    by using the previous discussion in order to compute the action \(S^{J,J^*} = \int_{\mathbb{R}^4} \dln4x \mathcal{L}^{J,J^*}\). Integrating by parts we obtain
    \begin{align*}
        S^{J,J^*} &= \int_{\mathbb{R}^4} \dln4x \left[\partial_\mu \varphi^* \partial^\mu \varphi - (m^2 - i \epsilon) \varphi^* \varphi + J^* \varphi + J \varphi^*\right]\\
                  &= \int_{\mathbb{R}^4} \dln4x \left[- \varphi^* \square_x \varphi - (m^2 - i \epsilon) \varphi^* \varphi + J^* \varphi + J\varphi^*\right]\\
                  &= \int_{\mathbb{R}^4} \dln4x \left[- \varphi^* L_x \varphi + J^* \varphi + J \varphi^*\right],
    \end{align*}
    using that the boundary terms all vanish as the fields are assumed to go to zero sufficiently fast. Notice the substitutions \(\varphi(x) = \tilde{\varphi}(x) + \psi(x)\) and \(\varphi^*(x) = \tilde{\varphi}^*(x) + \psi^*(x)\) have unitary jacobian for the path integral and that we have
    \begin{align*}
        \varphi^* L_x \varphi &= \left[\tilde{\varphi}^*(x) + \psi^*(x)\right]L_x \left[\tilde{\varphi}(x) + \psi(x)\right]\\
                              &= \left[\tilde{\varphi}^*(x) + \psi^*(x)\right] \left[L_x \tilde{\varphi}(x) + J(x)\right]\\
                              &= \tilde{\varphi}^*(x) L_x \tilde{\varphi}(x) + \tilde{\varphi}^*(x) J(x) + \psi^*(x) L_x \tilde{\varphi}(x) + \psi^*(x) J(x),
    \end{align*}
    thus
    \begin{align*}
        S^{J, J^*} &= - \int_{\mathbb{R}^4} \dln4x \left[ \tilde{\varphi}^* L_x \tilde{\varphi} + \tilde{\varphi}^* J + \psi^* L_x \tilde{\varphi} + \psi^* J  - J^*(\tilde{\varphi} + \psi) - J(\tilde{\varphi}^*  +\psi^*) \right]\\
                   &= - \int_{\mathbb{R}^4} \dln4x \left[\tilde{\varphi}^* L_x \tilde{\varphi} + (-1)^{2}(L_x\psi^*) \tilde{\varphi} - J^* (\tilde{\varphi} + \psi)\right]\\
                   &= - \int_{\mathbb{R}^4} \dln4x \left[\tilde{\varphi}^* L_x \tilde{\varphi} - J^* \psi\right].
    \end{align*}
    As a result, the generating functional is
    \begin{align*}
        Z[J, J^*] &= \int \dlp{\tilde{\varphi}(x)} \dlp{\tilde{\varphi}^*(x)} \exp\left[-i \int_{\mathbb{R}^4} \dln4x \left(\tilde{\varphi}^*(x)L_x \tilde{\varphi}(x) - J^*(x) \psi(x)\right)\right]\\
                  &=\exp\left[i\int_{\mathbb{R}^4} \dln4x J^*(x) \psi(x)\right]\underbrace{\int \dlp{\tilde{\varphi}(x)} \dlp{\tilde{\varphi}^*(x)} \exp\left[-i \int_{\mathbb{R}^4} \dln4x \tilde{\varphi}^*(x)L_x \tilde{\varphi}(x) \right]}_{Z[0]}\\
                  &= Z[0] \exp\left[-\int_{\mathbb{R}^4} \dln4x \int_{\mathbb{R}^4} \dln4y J^*(x) D_F(x,y) J(y)\right].
    \end{align*}
    
    We'll compute the two point function \(\bra{0} T \varphi(x_1) \herm{\varphi}(x_2) \ket{0}\) using the generating functional and obtain the Feynman propagator. By definition, the two point function is
    \begin{equation*}
        \bra{0} T \varphi(x_1) \herm{\varphi}(x_2) \ket{0} = \left(\frac1i\right)^2 \frac{\braket{0}{0}}{Z[0]}\diff.d.{Z[J, J^*]}{J^*(x_1),J(x_2)}[\substack{J = 0\\J^* = 0}] = - \frac{1}{Z[0]} \diff.d.{Z[J,J^*]}{J^*(x_1),J(x_2)}[\substack{J = 0\\J^*=0}],
    \end{equation*}
    so it remains to compute the first two functional derivatives of \(Z.\) For the first derivative, we have
    \begin{align*}
        \diff.d.{Z[J, J^*]}{J(x_2)} &= - Z[J, J^*] \diff.d.*{\left(\int_{\mathbb{R}^4} \dln4x \int_{\mathbb{R}^4} \dln4y J^*(x) D_F(x,y) J(y)\right)}{J(x_2)}\\
                                    &= - Z[J, J^*] \int_{\mathbb{R}^4} \dln4x \int_{\mathbb{R}^4} \dln4y J^*(x)D_F(x,y) \delta(y - x_2)\\
                                    &= - Z[J, J^*] \int_{\mathbb{R}^4} \dln4x  J^*(x) D_F(x, x_2)
    \end{align*}
    and
    \begin{align*}
        \diff.d.{Z[J, J^*]}{J^*(x_1)} &= - Z[J, J^*] \diff.d.*{\left(\int_{\mathbb{R}^4} \dln4x \int_{\mathbb{R}^4} \dln4y J^*(x) D_F(x,y) J(y)\right)}{J^*(x_1)}\\
                                    &= - Z[J, J^*] \int_{\mathbb{R}^4} \dln4x \int_{\mathbb{R}^4} \dln4y \delta(x - x_1)D_F(x,y) J(y)\\
                                    &= - Z[J, J^*] \int_{\mathbb{R}^4} \dln4y  D_F(x_1, y) J(y),
    \end{align*}
    then the second functional derivative of interest is
    \begin{align*}
        \diff.d.{Z[J,J^*]}{J^*(x_1),J(x_2)} &= -\diff.d.*{Z[J,J^*] \int_{\mathbb{R}^4} \dln4x J^*(x) D_F(x, x_2)}{J^*(x_1)}\\
                                           &= Z[J, J^*] \left(\int_{\mathbb{R}^4} \dln4y D_F(x_1,y) J(y) \int_{\mathbb{R}^4} \dln4x J^*(x) D_F(x, x_2) - \int_{\mathbb{R}^4} \dln4x \delta(x - x_1) D_F(x, x_2)\right)\\
                                           &= Z[J, J^*] \left(\int_{\mathbb{R}^4} \dln4y D_F(x_1,y) J(y) \int_{\mathbb{R}^4} \dln4x J^*(x) D_F(x, x_2) - D_F(x_1, x_2)\right).
    \end{align*}
    These results yield
    \begin{align*}
        \bra{0}T \varphi(x_1) \herm{\varphi}(x_2) \ket{0} &= - \left[\frac{Z[J, J^*]}{Z[0]} \left(\int_{\mathbb{R}^4} \dln4y D_F(x_1,y) J(y) \int_{\mathbb{R}^4} \dln4x J^*(x) D_F(x, x_2) - D_F(x_1, x_2)\right)\right]_{\substack{J = 0\\J^*=0}}\\
                                                          &= \frac{Z[0]}{Z[0]} D_F(x_1, x_2)\\
                                                          &= D_F(x_1,x_2),
    \end{align*}
    as desired.
\end{proof}

   % vim: spl=en
\begin{problem}{Cross section for annihilation of electron-positron in QED}{p3}
    In QED, evaluate the unpolarized cross section for \(e^+ e^- \to \gamma \gamma\) as a function of the final state photon polarizations\footnote{We'll ignore this and sum over photon polarizations.}. Take into account the non-vanishing electron mass.
\end{problem}
\begin{proof}[Solution]
    For the annihilation \(e^+ e^- \to \gamma \gamma\) we consider the lowest order diagrams
    \begin{center}
        \feynmandiagram[inline=(a.base),medium,vertical=a to b]{
            i1 [particle=\(e^-\)] -- [fermion, momentum'=\(p_1\)] a -- [photon, momentum'=\(p_3\)] f1 [particle=\(\gamma\)];
            a -- [fermion, momentum=\(p_1 - p_3\)] b;
            b -- [photon, momentum=\(p_4\)] f2 [particle=\(\gamma\)];
            i2 [particle=\(e^+\)] -- [anti fermion, momentum=\(p_2\)] b;
        };
        and
        \feynmandiagram[inline=(a.base),medium,vertical=a to b]{
            i1 [particle=\(e^-\)] -- [fermion, momentum'=\(p_1\)] a -- [photon, momentum'=\(p_4\)] f1 [particle=\(\gamma\)];
            a -- [fermion, momentum=\(p_1 - p_4\)] b;
            b -- [photon, momentum=\(p_3\)] f2 [particle=\(\gamma\)];
            i2 [particle=\(e^+\)] -- [anti fermion, momentum=\(p_2\)] b;
        };
    \end{center}
    with amplitudes
    \begin{align*}
        i \mathcal{M}_t &= \bar{v}(p_2) (i e \gamma^\nu) \epsilon^*_{\nu}(p_4) i\frac{\slashed{p}_1 - \slashed{p}_3 + m}{t - m^2} \epsilon^*_{\mu}(p_3) (i e \gamma^\mu) u(p_1) \\
                        &= i(ie)^2 \bar{v}(p_2) \slashed{\epsilon}^*(p_4) \frac{\slashed{p}_1 - \slashed{p}_3 + m}{t - m^2} \slashed{\epsilon}^*(p_3) u(p_1)
    \end{align*}
    and
    \begin{equation*}
        i\mathcal{M}_u = i(ie)^2 \bar{v}(p_2) \slashed{\epsilon}^*(p_3) \frac{\slashed{p}_1 - \slashed{p}_4 + m}{u - m^2} \slashed{\epsilon}^*(p_4) u(p_1).
    \end{equation*}
    In order to compute the cross section, we'll compute the amplitude
    \begin{equation*}
        i \mathcal{M} = i\mathcal{M}_t +  i\mathcal{M}_u,
    \end{equation*}
    sum over photon polarizations, and average over the initial spins
    \begin{equation*}
        \overline{\abs{i\mathcal{M}}^2} = \frac14\sum_{\lambda,\lambda'}\sum_{\mathrm{spin}}\abs{i \mathcal{M}}^2 = \underbrace{\frac14\sum_{\lambda,\lambda'}\sum_{\mathrm{spin}}\abs{i \mathcal{M}_t}^2}_{\overline{\abs{i\mathcal{M}_t}^2}} + \underbrace{\frac14\sum_{\lambda,\lambda'}\sum_{\mathrm{spin}} \abs{i\mathcal{M}_u}^2}_{\overline{\abs{i\mathcal{M}_u}^2}} + \underbrace{\frac12\sum_{\lambda,\lambda'}\sum_{\mathrm{spin}} \Re\left(\mathcal{M}_t^*\mathcal{M}_u\right)}_{2\overline{\Re\left(\mathcal{M}_t^*\mathcal{M}_u\right)}}.
    \end{equation*}
    As we have
    \begin{equation*}
        (\bar{v} \Lambda u)^* = \bar{u} \gamma^0 \herm{\Lambda} \gamma^0 v,
    \end{equation*}
    we compute
    \begin{equation*}
        [\slashed{a}^* (\slashed{b} + c) \slashed{d}^*]^\dag = \gamma^0 \slashed{d} \gamma^0 \gamma^0 (\slashed{b} + c) \gamma^0 \gamma^0 \slashed{a} \gamma^0 = \gamma^0 \slashed{d} (\slashed{b} + c) \slashed{a} \gamma^0,
    \end{equation*}
    then
    \begin{equation*}
        (i \mathcal{M}_t)^* = -i(ie)^2 \bar{u}(p_1) \slashed{\epsilon}(p_3) \frac{\slashed{p}_1 - \slashed{p}_3 + m}{t - m^2} \slashed{\epsilon}(p_4) v(p_2).
    \end{equation*}
    % and
    % \begin{equation*}
    %     (i \mathcal{M}_u)^* = -i(ie)^2 \bar{u}(p_1) \slashed{\epsilon}(p_4) \frac{\slashed{p}_1 - \slashed{p}_4 + m}{u - m^2} \slashed{\epsilon}(p_3) v(p_2) 
    % \end{equation*}
    For the \(t\)-channel we have
    \begin{equation*}
        \abs{i\mathcal{M}_t}^2 = e^4 \bar{u}_a^{s_1}(p_1) \left[\slashed{\epsilon}(p_3) \frac{\slashed{p}_1 - \slashed{p}_3 + m}{t - m^2}\slashed{\epsilon}(p_4)\right]_{ab} v^{s_2}_b(p_2)\bar{v}_c^{s_2}(p_2)\left[\slashed{\epsilon}^*(p_4) \frac{\slashed{p}_1 - \slashed{p}_3 + m}{t - m^2}\slashed{\epsilon}^*(p_3)\right]_{cd}u^{s_1}_d(p_1)
    \end{equation*}
    then
    \begin{align*}
        \frac14 \sum_{\mathrm{spin}}{\abs{i\mathcal{M}_t}^2} &= \frac{e^4}{4(t - m^2)^2} \Tr\left[(\slashed{p}_1 + m)\slashed{\epsilon}(p_3)(\slashed{p}_1 - \slashed{p}_3 + m) \slashed{\epsilon}(p_4)(\slashed{p}_2 - m)\slashed{\epsilon}^*(p_4) (\slashed{p}_1 - \slashed{p}_3 + m) \slashed{\epsilon}^*(p_3)\right]\\
                                                             &= \frac{e^4}{4(t - m^2)^2} \Tr\left[\slashed{\epsilon}^*(p_3)(\slashed{p}_1 + m)\slashed{\epsilon}(p_3)(\slashed{p}_1 - \slashed{p}_3 + m) \slashed{\epsilon}(p_4)(\slashed{p}_2 - m)\slashed{\epsilon}^*(p_4) (\slashed{p}_1 - \slashed{p}_3 + m)\right].
    \end{align*}
    When we sum over photon polarizations, we will get
    \begin{equation*}
        \sum_{\lambda} \sum_{\lambda'} \slashed{\epsilon}^*(p_3, \lambda) \dots \slashed{\epsilon}(p_3,\lambda) \dots \slashed{\epsilon}(p_4,\lambda') \dots \slashed{\epsilon}^*(p_4, \lambda') \dots  = \gamma^\mu \dots \gamma_\mu \dots \gamma^\nu \dots \gamma_\nu \dots,
    \end{equation*}
    then we may use \cref{prob:p2} to obtain
    \begin{equation*}
        \gamma^\nu (\slashed{p}_2 - m) \gamma_\nu = - 2(\slashed{p}_2 + 2m)\quad\text{and}\quad
        \gamma^\mu (\slashed{p}_1 + m) \gamma_\mu = -2(\slashed{p}_1 - 2m),
    \end{equation*}
    hence
    \begin{equation*}
        \overline{\abs{i\mathcal{M}_t}^2} = \frac{e^4}{(t - m^2)^2} \Tr\left[(\slashed{p}_1 - 2m)(\slashed{p}_1 - \slashed{p}_3 + m) (\slashed{p}_2 + 2m)(\slashed{p}_1 - \slashed{p}_3 + m)\right].
    \end{equation*}
    Henceforth we will use the following results concerning Mandelstam variables: first, their sum,
    \begin{equation*}
        s + t + u = p_1^2 + p_2^2 + p_3^2 + p_4^2 = 2m^2,
    \end{equation*}
    then the momentum product identities, for \(s\)
    \begin{equation*}
        2 p_1 p_2 = s - p_1^2 - p_2^2 = s - 2m^2,\quad\text{and}\quad
        2 p_3 p_4 = s - p_3^2 - p_4^2 = s
    \end{equation*}
    for \(t\)
    \begin{equation*}
        2 p_1 p_3 = p_1^2 + p_3^2 - t = m^2 - t,\quad\text{and}\quad
        2 p_2 p_4 = p_2^2 + p_4^2 - t = m^2 - t
    \end{equation*}
    and for \(u\)
    \begin{equation*}
        2 p_1 p_4 = p_1^2 + p_4^2 - u = m^2 - u,\quad\text{and}\quad
        2 p_2 p_3 = p_2^2 + p_3^2 - u = m^2 - u.
    \end{equation*}
    As the product of an odd number of \(\gamma\) matrices is traceless, we need only collect the terms of the \(m\) polynomial with even degree. For \(m^0\) we have
    \begin{align*}
        \Tr[\slashed{p}_1 (\slashed{p}_1 - \slashed{p}_3) \slashed{p}_2 (\slashed{p}_1 - \slashed{p}_3)] 
        &= 8(p_1^2 - p_1p_3)(p_2 p_1 - p_2 p_3) - 4 p_1 p_2 t\\
        &= 2(2 m^2 - 2p_1 p_3)(2p_2 p_1 - 2p_2 p_3) - 2 (s - 2m^2)t\\
        &= 2(m^2 + t)(s - 3m^2 + u) - 2(s - 2m^2)t\\
        &= 2(m^2 + t)(s + u + t - 3m^2 - t) - 2(s - 2m^2)t\\
        &= - 2(m^4 + 2m^2 t + t^2) - 2(s - 2m^2)t\\
        &= -2m^4 -2t^2 -2st,
    \end{align*}
    for \(m^2\) we have
    \begin{align*}
        \Tr[\slashed{p}_1 \slashed{p}_2 + 4(\slashed{p}_1 - \slashed{p}_2) (\slashed{p}_1 - \slashed{p}_3) - 4 (\slashed{p}_1 - \slashed{p}_3)^2] 
        &= 4 p_1 p_2 + 16 (p_1 - p_2)(p_1 - p_3) - 16 t\\
        &= 2 s - 4m^2 +  8(2m^2 - 2p_1 p_3 - 2p_1 p_2 + 2p_2 p_3) - 16t\\
        &= 2s - 4m^2 + 8(4m^2 + t - s - u) - 16t\\
        &= 2s - 4m^2 + 8(2t + 2m^2) - 16t\\
        &= 2s + 12m^2
    \end{align*}
    then
    \begin{align*}
        \overline{\abs{i\mathcal{M}_t}^2} &= \frac{e^4\left[(-2m^4 - 2t^2-2st)m^0 + (2s + 12m^2)m^2 - 16m^4\right]}{(t - m^2)^2}\\
                                          &= \frac{2e^4\left(m^2s-3m^4 - t^2 - st\right)}{(t - m^2)^2}.
    \end{align*}
    Replacing \(t \to u\) we obtain the \(u\) channel amplitude,
    \begin{align*}
        \overline{\abs{i\mathcal{M}_u}^2} = \frac{2e^4\left(m^2s-3m^4 - u^2 - su\right)}{(u - m^2)^2}.
    \end{align*}
    For the interference term, we have
    \begin{equation*}
        \mathcal{M}_t^*\mathcal{M}_u = 
        e^4 \bar{u}^{s_1}_a(p_1) \left[\slashed{\epsilon}(p_3) \frac{\slashed{p}_1 - \slashed{p}_3 + m}{t - m^2} \slashed{\epsilon}(p_4)\right]_{ab} v^{s_2}_b(p_2)\bar{v}^{s_2}_c(p_2) \left[\slashed{\epsilon}^*(p_3) \frac{\slashed{p}_1 - \slashed{p}_4 + m}{u - m^2} \slashed{\epsilon}^*(p_4)\right]_{cd} u^{s_1}_d(p_1),
    \end{equation*}
    then
    \begin{equation*}
        \frac12 \sum_{\mathrm{spin}} \mathcal{M}_t^* \mathcal{M}_u = \frac{e^4\Tr\left[(\slashed{p}_1 + m) \slashed{\epsilon}(p_3)(\slashed{p}_1 - \slashed{p}_3 + m) \slashed{\epsilon}(p_4) (\slashed{p}_2 - m) \slashed{\epsilon}^*(p_3) (\slashed{p}_1 - \slashed{p}_4 + m) \slashed{\epsilon}^*(p_4)\right]}{2(t - m^2)(u - m^2)}.
    \end{equation*}
    Before summing over polarizations, we consider the term between the \(\slashed{\epsilon}(p_3, \lambda)\) polarizations
    \begin{align*}
        (\slashed{p}_1 - \slashed{p}_3 + m) \slashed{\epsilon}(p_4) (\slashed{p}_2 - m) 
        &= (\slashed{p}_1 - \slashed{p}_3 + m)( \slashed{\epsilon}(p_4) \slashed{p}_2 - m \slashed{\epsilon}(p_4))\\
        &= (\slashed{p}_1 - \slashed{p}_3)\slashed{\epsilon}(p_4) \slashed{p}_2 - m(\slashed{p}_1 - \slashed{p}_3)\slashed{\epsilon}(p_4) + m \slashed{\epsilon}(p_4) \slashed{p}_2 - m^2 \slashed{\epsilon}(p_4)\\
        &=(\slashed{p}_1 - \slashed{p}_3)\slashed{\epsilon}(p_4) \slashed{p}_2 + m \left[\slashed{\epsilon}(p_4) \slashed{p}_2 - (\slashed{p}_1 - \slashed{p}_3) \slashed{\epsilon}(p_4)\right] - m^2 \slashed{\epsilon}(p_4)
    \end{align*}
    then when we sum over \(\lambda\) this term becomes
    \begin{equation*}
        \sum_{\lambda} \slashed{\epsilon}(p_3)(\slashed{p}_1 - \slashed{p}_3 + m) \slashed{\epsilon}(p_4) (\slashed{p}_2 - m) \slashed{\epsilon}^*(p_3) = 2 \slashed{p}_2 \slashed{\epsilon}(p_4) (\slashed{p}_1 - \slashed{p}_3) - 4m(p_2 + p_3 - p_1) \epsilon(p_4) - 2m^2 \slashed{\epsilon}(p_4)
    \end{equation*}
    and hence 
    \begin{align*}
        \frac12 \sum_{\lambda} \sum_{\mathrm{spin}} \mathcal{M}_t^*\mathcal{M}_u 
        &= -\frac{e^4}{(t - m^2)(u - m^2)}\left\{m^2\Tr\left[(\slashed{p}_1 + m) \slashed{\epsilon}(p_4) (\slashed{p}_1 - \slashed{p}_4 + m) \slashed{\epsilon}^*(p_4)\right]\right. + \\
        &{}\phantom{=-{(t - m^2)(u - m^2)}} {}+ 2m(p_2 + p_3 - p_1) \epsilon(p_4) \Tr\left[(\slashed{p}_1 + m) (\slashed{p}_1 - \slashed{p}_4 + m) \slashed{\epsilon}^*(p_4)\right]+\\
        &{}\phantom{=-{(t - m^2)(u - m^2)}} \left.{}- \Tr\left[(\slashed{p}_1 + m) \slashed{p}_2 \slashed{\epsilon}(p_4) (\slashed{p}_1 - \slashed{p}_3) (\slashed{p}_1 - \slashed{p}_4 + m) \slashed{\epsilon}^*(p_4)\right]\right\}.
    \end{align*}
    When we sum over \(\lambda'\) we will get the following developments
    \begin{equation*}
        \sum_{\lambda'}\slashed{\epsilon}(p_4)(\slashed{p}_1 - \slashed{p}_4 + m) \slashed{\epsilon}^*(p_4) = 2(\slashed{p}_1 - \slashed{p}_4 - 2m),
    \end{equation*}
    \begin{equation*}
        \sum_{\lambda'} \epsilon_\mu(p_4) \Tr\left[(\slashed{p}_1 + m) (\slashed{p}_1 - \slashed{p}_4 + m) \slashed{\epsilon}^*(p_4)\right] = - g_{\mu\nu} \Tr\left[(\slashed{p}_1 + m)(\slashed{p}_1 - \slashed{p}_4 + m) \gamma^\nu\right],
    \end{equation*}
    and
    \begin{equation*}
        \sum_{\lambda'} \slashed{\epsilon}(p_4) (\slashed{p}_1 - \slashed{p}_3) (\slashed{p}_1 - \slashed{p}_4 + m) \slashed{\epsilon}^*(p_4) = - 4(p_1 - p_3)(p_1 - p_4) + 2m (\slashed{p}_1 - \slashed{p}_3)
    \end{equation*}
    then
    \begin{align*}
        \frac12 \sum_{\lambda, \lambda'} \sum_{\mathrm{spin}} \mathcal{M}_t^* \mathcal{M}_u 
        &= - \frac{e^4}{(t - m^2)(u - m^2)} \left\{2m^2 \Tr\left[(\slashed{p}_1 + m)(\slashed{p}_1 - \slashed{p}_4 - 2m)\right]+\right.\\
        &{}\phantom{=-(t - m^2)(u - m^2)} -2m \Tr\left[(\slashed{p}_1 + m)(\slashed{p}_1 - \slashed{p}_4 + m)(\slashed{p}_2 + \slashed{p}_3 - \slashed{p}_1)\right]\\
        &{}\phantom{=-(t - m^2)} \left.+ 4(p_1 - p_3)(p_1 - p_4) \Tr\left[(\slashed{p}_1 + m)\slashed{p}_2\right] - 2m \Tr\left[(\slashed{p}_1 + m)\slashed{p}_2(\slashed{p}_1 - \slashed{p}_3)\right]\right\}.
    \end{align*}
    We compute the traces separately,
    \begin{align*}
        \Tr\left[(\slashed{p}_1 + m)(\slashed{p}_1 - \slashed{p}_4 - 2m)\right] 
        &= \Tr\left[\slashed{p}_1(\slashed{p}_1 - \slashed{p}_4)\right] - 8m^2\\
        &= 2(2p_1^2 - 2p_1 p_4) - 8m^2\\
        &= 2(m^2 + u) - 8m^2\\
        &= 2u - 6m^2,
    \end{align*}
    \begin{align*}
        \Tr\left[(\slashed{p}_1 + m)(\slashed{p}_1 - \slashed{p}_4 + m)(\slashed{p}_2 + \slashed{p}_3 - \slashed{p}_1)\right] 
        &= m\Tr\left[(2\slashed{p}_1 - \slashed{p}_4)(\slashed{p}_2 + \slashed{p}_3 - \slashed{p}_1)\right]\\
        &= 2m (4p_1p_2 + 4p_1p_3 - 4p_1^2 - 2p_4p_2 - 2p_4p_3 + 2p_4p_1)\\
        &= 2m[(4p_1 p_2 - 2p_3 p_4) + (4p_1 p_3 - 2p_2 p_4) + 2p_1 p_4 - 4p_1^2]\\
        &= 2m(s - 4m^2 + m^2 - t + m^2 - u - 4m^2)\\
        &= 2m[2s - (s + t + u) - 6m^2]\\
        &= 4m(s - 4m^2)
    \end{align*}
    \begin{align*}
        (p_1 - p_3)(p_1 - p_4)\Tr\left[(\slashed{p}_1 + m) \slashed{p}_2\right] 
        &= (2p_1^2 - 2p_1 p_4 - 2p_1 p_3 + 2p_3 p_4)(2p_1 p_2)\\
        &= (2m^2 - m^2 + u - m^2 + t + s)(s - 2m^2)\\
        &= 2m^2 (s - 2m^2)
    \end{align*}
    and
    \begin{align*}
        \Tr\left[(\slashed{p}_1 + m)\slashed{p}_2(\slashed{p}_1 - \slashed{p}_3)\right]
        &= m \Tr\left[\slashed{p}_2 (\slashed{p}_1 - \slashed{p}_3)\right]\\
        &= 2m (2p_1 p_2 -2 p_2 p_3)\\
        &= 2m (s - 3m^2 + u)\\
        &= -2m (m^2 + t)
    \end{align*}
    then noticing all of these results are real, we have
    \begin{align*}
        2\overline{\Re(\mathcal{M}_t^*\mathcal{M}_u)} &= -\frac{e^4}{(t - m^2)(u - m^2)} \left[2m^2(2u - 6m^2) - 8m^2(s - 4m^2) + 8m^2(s - 2m^2) + 4m^2(m^2 + t)\right]\\
                                                      &= -\frac{4m^2e^4\left[(u - 3m^2) - 2(s - 4m^2) + 2(s - 2m^2) + m^2 + t\right]}{(t - m^2)(u - m^2)}\\
                                                      &= - \frac{4m^2 e^4 (2m^2 + u + t)}{(t - m^2)(u - m^2)}
    \end{align*}
    Then, the amplitude is
    \begin{equation*}
        \overline{\abs{i\mathcal{M}}^2} = \frac{2e^4\left(m^2s-3m^4 - t^2 - st\right)}{(t - m^2)^2} + \frac{2e^4\left(m^2s-3m^4 - u^2 - su\right)}{(u - m^2)^2}- \frac{4m^2 e^4(t + u + 2m^2)}{(t - m^2)(u - m^2)},
    \end{equation*}
    which yields
    \begin{equation*}
        \overline{\abs{i\mathcal{M}}^2} \to -2e^4\left(\frac{t + s}{t} + \frac{u + s}{u}\right) = 2e^4 \left(\frac{u}{t} + \frac{t}{u}\right)
    \end{equation*}
    in the high energy limit \(m \to 0.\)

    The differential cross section of the pair annihilation is
    \begin{equation*}
        \dl{\sigma} = \dli{\Pi_2} \frac{\overline{\abs{i\mathcal{M}}^2}}{F},
    \end{equation*}
    where the differential phase space is, by \href{https://github.com/louisradial/4305107-quantum-field-theory-i/releases/tag/pset8}{Problem 3 of Problem Set VIII},
    \begin{equation*}
        \dl{\Pi_2} = \dli{\Omega} \frac{1}{32\pi^2},
    \end{equation*}
    and, as we've shown in \href{https://github.com/louisradial/4305107-quantum-field-theory-i/releases/tag/pset7}{Problem 4 of Problem Set VII}, the flux factor in the center of momentum frame is
    \begin{equation*}
    F = (2E_1)(2E_2) \sqrt{(\vetor{v}_1 - \vetor{v}_2)^2 - (\vetor{v}_1 \times \vetor{v}_2)^2} = 2 s\sqrt{1 - \frac{4m^2}{s}},
    \end{equation*}
    hence
    \begin{equation*}
        \diff{\sigma}{\Omega} = \frac{e^4}{32\pi^2 s\sqrt{1 - \frac{4m^2}{s}}}\left[\frac{m^2s-3m^4 - t^2 - st}{(t - m^2)^2} + \frac{m^2s-3m^4 - u^2 - su}{(u - m^2)^2}- \frac{2m^2 (t + u + 2m^2)}{(t - m^2)(u - m^2)}\right]
    \end{equation*}
    is the cross section for \(e^+ e^- \to \gamma \gamma.\)
\end{proof}

   \begin{problem}{Conserved quantity due to electromagnetism's Lorentz invariance}{p4}
   Consider the vector field \(A^\mu\) associated to the electromagnetic field. Obtain the conserved quantities due to the Lorentz invariance of the theory. Use the Langrangian density given in the first problem.
\end{problem}
\begin{proof}[Solution]
   We consider the infinitesimal Lorentz transformation
   \begin{equation*}
      x^\mu \mapsto \tilde{x}^\mu = \left(\delta\indices{^\mu_\nu} + \omega\indices{^\mu_\nu}\right)x^\nu
      \quad\text{and}\quad
      A_\mu(x) \mapsto \tilde{A}_\mu(\tilde{x}) = \diffp{x^\nu}{\tilde{x}^\mu} A_\nu(x).
   \end{equation*}
   As the transformation is infinitesimal, we have \(\diffp{x^\nu}{\tilde{x}^\mu} = \delta\indices{^\nu_\mu} - \omega\indices{^\nu_\mu}\), hence
   \begin{align*}
      \delta A_\mu(x) = \tilde{A}_\mu(x) - A_\mu(x) 
      &=\left(\delta\indices{^\nu_\mu} - \omega\indices{^\nu_\mu}\right) A_\nu(x - \omega x) - A_\mu(x)\\
      &=\left(\delta\indices{^\nu_\mu} - \omega\indices{^\nu_\mu}\right) \left[A_\nu(x) - \omega\indices{^\rho_\sigma} x^\sigma \partial_\rho A_\nu(x)\right] - A_\mu(x)\\
      &=  - \omega\indices{^\nu_\mu} A_\nu(x) - \omega\indices{^\lambda_\sigma} x^\sigma \partial_\lambda A_\mu(x)
   \end{align*}
   and \(\Delta x^\rho = \omega\indices{^\rho_\eta}x^\eta\). As we have shown in \cref{prob:p1}, we have
   \begin{equation*}
      \diffp{\mathcal{L}}{(\partial_\rho A_\mu)} = - F^{\rho\mu},
   \end{equation*}
   then the conserved current guaranteed by \nameref{thm:noether} is
   \begin{align*}
      j^\rho = \diffp{L}{(\partial_\rho A_\mu)}\delta A_\mu(x) + \mathcal{L} \Delta x^\rho
      &= F^{\rho\mu} \left[\omega\indices{^\sigma_\mu} A_\sigma + \omega\indices{^\sigma_\lambda} x^\lambda \partial_\sigma A_\mu\right] - \frac14 F^{\alpha \beta}F_{\alpha \beta} \omega\indices{^\rho_\lambda} x^\lambda\\
      &= \omega\indices{^\sigma_\lambda}\left[F^{\rho \mu}\left(\delta\indices{^\lambda_\mu} A_\sigma + x^\lambda \partial_\sigma A_\mu\right) - \frac14 F^{\alpha \beta}F_{\alpha \beta} \delta\indices{^\rho_\sigma}x^\lambda\right]\\
      &= \omega_{\nu \lambda}\left[F^{\rho \lambda} A^\nu + x^\lambda \left(F^{\rho \mu} \partial^\nu A_\mu - \frac14 g^{\rho \nu} F^{\alpha \beta} F_{\alpha \beta}\right)\right]\\
      &= \omega_{\nu \lambda} \left[F^{\rho \lambda} A^\nu + x^\lambda F^{\rho \mu} \partial_\mu A^\nu + x^\lambda T^{\rho \nu}\right]\\
      &= \omega_{\nu \lambda} \left[F^{\rho \mu} \delta\indices{^\lambda_\mu} A^\nu + x^\lambda F^{\rho \mu} \partial_\mu A^\nu + x^\lambda T^{\rho \nu}\right]\\
      &= \omega_{\nu \lambda} \left[F^{\rho \mu} A^\nu \partial_\mu x^\lambda + x^\lambda F^{\rho \mu} \partial_\mu A^\nu + x^\lambda T^{\rho \nu}\right]\\
      &= \omega_{\nu \lambda} \left[F^{\rho \mu} \partial_\mu( x^\lambda A^\nu) + x^\lambda T^{\rho \nu}\right]\\
      &= \frac12 \omega_{\nu \lambda} \left[F^{\rho \mu} \partial_\mu( x^\lambda A^\nu) + x^\lambda T^{\rho \nu}\right] + \frac12 \omega_{\lambda \nu}\left[F^{\rho \mu} \partial_\mu( x^\nu A^\lambda) + x^\nu T^{\rho \lambda}\right]\\
      &= \frac12 \omega_{\nu \lambda} \left[x^{\lambda} T^{\rho \nu} - x^\nu T^{\rho \lambda} + F^{\rho \mu} \partial_\mu (x^\lambda A^\nu - x^\nu A^{\lambda})\right] 
   \end{align*}
   where \(T^{\rho\nu}\) is the symmetric stress-energy tensor defined in \cref{prob:p1}. As the Lorentz transformation components are arbitrary, the tensor
   \begin{equation*}
      \tilde{M}^{\lambda \rho \nu} = x^{\lambda} T^{\rho \nu} - x^\nu T^{\rho \lambda} + F^{\rho \mu} \partial_\mu(x^\lambda A^\nu - x^\nu A^\lambda)
   \end{equation*}
   is conserved with \(\partial_\rho \tilde{M}^{\lambda \rho \nu} = 0\). The equations of motion then yield the conservation of the tensor \(M^{\lambda \rho \nu} = x^{\lambda}T^{\rho \nu} - x^{\nu}T^{\rho \lambda}\) as we have
   \begin{equation*}
      \partial_\rho (\tilde{M}^{\lambda \rho \nu} - M^{\lambda \rho \nu}) = \partial_\rho \left[F^{\rho \mu} \partial_\mu(x^\lambda A^\nu - x^\nu A^\lambda)\right]  = F^{\rho \mu} \partial_\rho \partial_\mu (x^\lambda A^\nu - x^\nu A^\lambda) = 0,
   \end{equation*}
   since \(F^{\rho \mu}\) is antisymmetric and the partial derivatives commute.
\end{proof}

\end{document}
