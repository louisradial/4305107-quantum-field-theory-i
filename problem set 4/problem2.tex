% vim: spl=en
\begin{problem}{Dirac equation for massless particles}{p2}
   Consider the Dirac equation for massless particles, \(i \slashed{\partial} \Psi = 0.\)
   \begin{enumerate}[label=(\alph*)]
      \item Show that the operators \(P_R = \frac{1 + \gamma^5}{2}\) and \(P_L = \frac{1 - \gamma^5}{2}\) are orthogonal projectors. Note that \(P_R + P_L = 1.\)
      \item Show  for massless Dirac spinors that the eigenstates of helicity are also eigenstates of chirality, i.e., \(\gamma^5\). Do this for the positive and negative eigenstates of the Hamiltonian.
      \item Show that for any Dirac spinor \(\Psi\) (possibly massive) the following spinors are eigenstates of chirality: \(\Psi_{L,R} = P_{L,R} \Psi.\)
      \item Write the Dirac equation in terms of \(\Psi_{L,R}.\)
      \item How do \(\Psi_{L,R}\) transform under \(C,\) \(P,\) and \(T?\)
      \item Show that the Dirac Lagrangian density for a massless particle is invariant under the chiral transformation \(U = \exp(-i \alpha \gamma^5)\). What is the conserved current associated to this symmetry? Is this true if the particle has a mass \(m?\)
   \end{enumerate}
\end{problem}
\begin{proof}[Solution]
   Notice 
   \begin{equation*}
      \mu \neq \nu \implies \gamma^\mu \gamma^\nu = \anticommutator{\gamma^\mu}{\gamma^\nu} - \gamma^\nu \gamma^\mu = - \gamma^\nu \gamma^\mu,
   \end{equation*}
   then \(\gamma^5\) is a Hermitian operator with \((\gamma^5)^2 = \unity\), as we have
   \begin{align*}
      \herm{\gamma^5} &= -i \herm{\gamma^3} \herm{\gamma^2} \herm{\gamma^1} \herm{\gamma^0} \\
                      &= -i (\gamma^0 \gamma^3 \gamma^0) (\gamma^0 \gamma^2 \gamma^0) (\gamma^0 \gamma^1 \gamma^0) (\gamma^0 \gamma^0 \gamma^0) \\
                      &= -i \gamma^0\gamma^3 \gamma^2 \gamma^1 \\
                      &= -i \gamma^0 \gamma^1 \gamma^3 \gamma^2 \\
                      &= i \gamma^0 \gamma^1 \gamma^2 \gamma^3 \\
                      &= \gamma^5,
   \end{align*}
   and
   \begin{equation*}
      (\gamma^5)^2 = - \gamma^0 \gamma^1 \gamma^2 \gamma^3 \gamma^0 \gamma^1 \gamma^2 \gamma^3 = - \gamma^0 \gamma^0 \gamma^1 \gamma^1 \gamma^2 \gamma^2 \gamma^3 \gamma^3 = \unity.
   \end{equation*}
   Then the operators \(P_R\) and \(P_L\) are orthogonal projectors as
   \begin{equation*}
      \left[\frac12 \left(\unity \pm \gamma^5\right)\right]^2 = \frac14 \left(\unity \pm 2 \gamma^5 + (\gamma^5)\right) = \frac12 \left(\unity \pm \gamma^5\right)
   \end{equation*}
   and they are real polynomials on \(\gamma^5,\) hence hermitian. It is clear that these orthogonal projectors are complementary as \(P_L + P_R = \unity.\) Moreover, as \(P_L\) and \(P_R\) are polynomials on \(\gamma^5,\) it is clear that \(\commutator{\gamma^5}{P_{L,R}} = 0\)


   From \cref{prob:p1}, the following spinors in the Weyl representation are eigenstates of helicity
   \begin{equation*}
      u_{\uparrow}(\vetor{p}) = \sqrt{2E} \begin{pmatrix}
         0\\
         \xi_\uparrow(\vetor{p})
      \end{pmatrix},\quad
      u_{\downarrow}(\vetor{p}) = \sqrt{2E} \begin{pmatrix}
         \xi_\downarrow(\vetor{p})\\
         0
      \end{pmatrix},\quad
      v_{\uparrow}(\vetor{p}) = \sqrt{2E} \begin{pmatrix}
         \xi_\uparrow(\vetor{p})\\
         0
      \end{pmatrix},\quad\text{and}\quad
      v_{\downarrow}(\vetor{p}) = \sqrt{2E} \begin{pmatrix}
         0\\
         \xi_\downarrow(\vetor{p})
      \end{pmatrix},
   \end{equation*}
   where \(\vetor{\sigma} \cdot \frac{\vetor{p}}{\norm{\vetor{p}}}\xi_s(\vetor{p}) = s \xi_s(\vetor{p}).\) In the Weyl representation, \(\gamma^5 = \left(\begin{smallmatrix} - \unity &&\\ && \unity \end{smallmatrix}\right)\), then
   \begin{equation*}
      \gamma^5 u_{\uparrow} = u_{\uparrow},\quad
      \gamma^5 u_{\downarrow} = - u_{\downarrow},\quad
      \gamma^5 v_{\uparrow} = -v_{\uparrow},\quad\text{and}\quad
      \gamma^5 v_{\downarrow} = v_{\downarrow},
   \end{equation*}
   hence they are also eigenstates of chirality.

   Notice 
   \begin{equation*}
       \frac12 \gamma^5 (\unity \pm \gamma^5) = \frac12 (\gamma^5  \pm \unity) = \pm \frac12 (\unity \pm \gamma^5),
   \end{equation*}
   then \(\gamma^5 P_{R} = P_R\) and \(\gamma^5 P_L = -P_L\). As a result, we have \(\gamma^5 P_{R} \Psi = P_R \Psi\) and \(\gamma^5 P_L \Psi = - P_L \Psi,\) for any spinor \(\Psi,\) hence \(P_{L,R}\) are eigenstates of chirality. We may use this to define \(\Psi_{L,R} = P_{L,R} \Psi\) for any spinor \(\Psi\) and use it for a solution to the Dirac equation. In the Weyl representation, \(P_L = \left(\begin{smallmatrix} \unity &&\\&& \end{smallmatrix}\right)\) and \(P_R = \left(\begin{smallmatrix} &&\\&& \unity\end{smallmatrix}\right)\), hence 
   \begin{equation*}
       \Psi_R = \begin{pmatrix}
           0\\
           \psi_R
       \end{pmatrix}
       \quad\text{and}\quad
       \Psi_L = \begin{pmatrix}
           \psi_L\\
           0
       \end{pmatrix},
   \end{equation*}
   where \(\psi_L\) and \(\psi_R\) are two-component Weyl spinors. In the Weyl representation, the Dirac equation operator is 
   \begin{equation*}
       i\slashed{\partial} - m \doteq \begin{pmatrix}
          -m && i \partial_0 + i \sigma^j \partial_j\\
          i \partial_0 - i \sigma^j \partial_j && -m
       \end{pmatrix},
   \end{equation*}
   hence
   \begin{equation*}
      (i \slashed{\partial} - m)\Psi \doteq \begin{pmatrix}
          -m \psi_L + i \partial_0 \psi_R + i \sigma^j \partial_j \psi_R\\
          i \partial_0 \psi_L - i \sigma^j \partial_j \psi_L - m \psi_R
      \end{pmatrix}
   \end{equation*}
   and the Dirac equation is translated to the coupled equations
   \begin{equation*}
      (i \slashed{\partial} - m)\Psi = 0 \implies \begin{cases}
          i \partial_0 \psi_R + i \sigma^j \partial_j \psi_R = m \psi_L\\
          i \partial_0 \psi_L - i \sigma^j \partial_j \psi_L = m \psi_R.
      \end{cases}
   \end{equation*}
   Notice that for massless particles, this representation decouples the equations in terms of the chirality eigenstates, yielding the Weyl equations
   \begin{equation*}
      i \sigma^\mu \partial_\mu \psi_R = 0 \quad\text{and}\quad i \bar{\sigma}^\mu \partial_\mu \psi_L = 0,
   \end{equation*}
   where \(\sigma^\mu = (\unity, \sigma^j)\) and \(\bar{\sigma}^\mu = (\unity, -\sigma^j).\)

   % TODO: I would like to show this when I get the time to
   The charge conjugation in the Weyl representation is \(C\Psi = -i \gamma^2\Psi^*,\) then
   \begin{equation*}
      C\Psi = \begin{pmatrix}
          && -i \sigma^2\\
         i\sigma^2 && 
      \end{pmatrix}
      \begin{pmatrix}
         \psi_L^*\\
         \psi_R^*
         \end{pmatrix} = \begin{pmatrix}
         -i\sigma^*2 \psi_R\\
         i \sigma^2 \psi_L^*
      \end{pmatrix}
   \end{equation*}
   that is, under charge conjugation we have
   \begin{equation*}
       C : \begin{cases}
          \psi_L(t, \vetor{x}) \to -i \sigma^2 \psi_R^* (t, \vetor{x})\\
          \psi_R(t, \vetor{x}) \to i \sigma^2 \psi_L^* (t, \vetor{x}).
       \end{cases}
   \end{equation*}
   Parity is represented as \(P \Psi(t, \vetor{x}) = \gamma^0 \psi(t, - \vetor{x}),\) then as \(\gamma^0 = \left(\begin{smallmatrix} &&\unity\\\unity&& \end{smallmatrix}\right)\), we have
   \begin{equation*}
       P : \begin{cases}
          \psi_L(t, \vetor{x}) \to \psi_R(t, -\vetor{x})\\
          \psi_R(t, \vetor{x}) \to \psi_L(t, -\vetor{x}).
       \end{cases}
   \end{equation*}
   Time reversal is represented as \(T\Psi(t, \vetor{x}) = \gamma^1 \gamma^3 \Psi(-t, \vetor{x}),\) then as \(\gamma^1 \gamma^3 = -\sigma^1\sigma^3\unity = \sigma^2 \unity,\) we have
   \begin{equation*}
       T : \begin{cases}
          \psi_L(t, \vetor{x}) \to \sigma^2\psi_L(-t, \vetor{x})\\
          \psi_R(t, \vetor{x}) \to \sigma^2\psi_R(-t, \vetor{x}).
       \end{cases}
   \end{equation*}

   We consider the massless Dirac Lagrangian density \(\mathcal{L} = i\bar{\Psi} \slashed{\partial} \Psi\) and the chiral transformation \(\Psi \to \tilde{\Psi} = \exp(-i \alpha \gamma^5) \Psi.\)
   It is easy to verify \(\anticommutator{\gamma^5}{\gamma^\mu} = 0,\) as it requires three adjacent swaps to make each term have the same product of \(\gamma\) matrices, hence they add to zero, for instance,
   \begin{equation*}
      \anticommutator{\gamma^5}{\gamma^1} = i \gamma^0 \gamma^1 \gamma^2 \gamma^3 \gamma^1 + i \gamma^1 \gamma^0 \gamma^1 \gamma^2 \gamma^3 = i \gamma^0 \gamma^1 \gamma^1 \gamma^2 \gamma^3 - i \gamma^0 \gamma^1\gamma^1 \gamma^2 \gamma^3 = 0.
   \end{equation*}
   As a result, it is easy to see with induction that \(\gamma^\mu (\gamma^5)^n = (-\gamma^5)^n\gamma^\mu\) for all \(n \in \mathbb{N},\) hence
   \begin{equation*}
      \gamma^\mu \exp(i \alpha \gamma^5) = \gamma^\mu \sum_{k = 0}^{\infty} \frac{(i \alpha \gamma^5)^k}{k!} = \sum_{k = 0}^{\infty} \frac{(i \alpha)^k \gamma^\mu (\gamma^5)^k}{k!} = \sum_{k=0}^\infty \frac{(-i \alpha \gamma^5)^k}{k!} \gamma^\mu = \exp(-i \alpha \gamma^5) \gamma^\mu.
   \end{equation*}
   We may now show the Lagrangian density is invariant under the chiral transformation, as 
   \begin{equation*}
      \tilde{\mathcal{L}} = i\herm{\tilde{\Psi}} \gamma^0 \slashed{\partial} \tilde{\Psi} = i\herm{\Psi} \exp(i \alpha \gamma^5) \gamma^0 \gamma^\mu  \exp(-i \alpha \gamma^5) \partial_\mu\Psi = i\bar{\Psi} \exp(-i \alpha \gamma^5) \exp(i \alpha \gamma^5) \slashed{\partial} \Psi = \mathcal{L}.
   \end{equation*}
   We observe this is not the case for massive particle, since
   \begin{equation*}
       \herm{\tilde{\Psi}}\gamma^0 \tilde{\Psi} = \herm{\Psi} \exp(i \alpha \gamma^5) \gamma^0 \exp(-i \alpha \gamma^5) \Psi = \bar{\Psi} \exp(-2i \alpha \gamma^5) \Psi,
   \end{equation*}
   so the massive Lagrangian density is not invariant under such transformation. Under the infinitesimal chiral transformation, \(\Psi \to \tilde{\Psi} = (\unity - i \alpha \gamma^5)\Psi\), we have the conserved current
   \begin{equation*}
      J^\nu = -i \diffp{\mathcal{L}}{(\partial_\nu\Psi)} \gamma^5\Psi = \bar{\Psi} \gamma^\nu \gamma^5\Psi
   \end{equation*}
   for massless particles.
\end{proof}
