% vim: spl=en
\begin{problem}{Correlation functions for a massive Dirac field}{p1}
   Consider a free Dirac field of mass \(m\).
   \begin{enumerate}[label=(\alph*)]
      \item Obtain \(\bra{0}T \Psi(x_1) \bar{\Psi}(x_2) \Psi(x_3) \bar{\Psi}(x_4)\ket{0}\).
      \item Generalize the result to \(\bra{0} T \Psi(x_1) \bar{\Psi}(x_2) \dots \Psi(x_{2N - 1}) \bar{\Psi}(x_{2N})\ket{0}.\)
   \end{enumerate}
\end{problem}
\begin{proof}[Solution]
   We consider the generating functional
   \begin{equation*}
      Z[\eta, \bar{\eta}] = Z[0,0] \exp\left[- \int_{\mathbb{R}^4} \dln4x \int_{\mathbb{R}^4} \dln4y \bar{\eta}(x) S_F(x - y) \eta(y)\right],
   \end{equation*}
   where \(\eta\) and \(\bar{\eta}\) are Grassmann variables. The \(2N\)-point function is then
   \begin{align*}
      G^{(2N)}(x_1, \dots, x_{2N}) &= \bra{0} T \Psi(x_1) \bar{\Psi}(x_2) \dots \Psi(x_{2N - 1}) \bar{\Psi}(x_{2N})\ket{0}\\
                                   &= \frac{1}{Z[0,0]}\diff.d.{Z[\eta,\bar{\eta}]}{\bar{\eta}(x_{1})\eta(x_{2})\dots \bar{\eta}(x_{2N -1}) \eta(x_{2N})}[\eta = \bar{\eta} = 0].
   \end{align*}
   We'll obtain the four-point function and generalize the result to \(2N.\) To make notation less cumbersome, we'll use the shorthand 
   \begin{equation*}
      \bar{\eta} S \eta \equiv \int_{\mathbb{R}^4} \dln4x \int_{\mathbb{R}^4} \dln4{y} \bar{\eta}(x) S_F(x - y) \eta(y),
   \end{equation*}
   \begin{equation*}
      (\bar{\eta} S)_i \equiv \int_{\mathbb{R}^4} \dln4x \bar{\eta}(x) S_F(x - x_i),
      \quad\text{and}\quad
      (S \eta)_i \equiv \int_{\mathbb{R}^4} \dln4y S_F(x_i - y)\eta(y),
   \end{equation*}
   as well as \(\eta(x_i) = \eta_i,\) \(\bar{\eta}(x_i) = \bar{\eta}_i,\) and \(S(x_i - x_j) = S_{ij}\) Notice that
   \begin{align*}
      \diff.d.{e^{- \bar{\eta} S \eta}}{\bar{\eta}_j,\eta_i} &= \diff.d.*{}{\bar{\eta}_j} (\bar{\eta} S)_i e^{- \bar{\eta} S \eta} = S_{ji} e^{- \bar{\eta} S \eta} - (\bar{\eta} S)_i \diff.d.*{}{\bar{\eta}_j} e^{-\bar{\eta}S\eta} = S_{ji} e^{- \bar{\eta} S \eta} + (\bar{\eta} S)_i (S \eta)_j e^{- \bar{\eta} S \eta}\\
      &= \left[S_{ji} + (\bar{\eta} S)_i (S \eta)_j\right]e^{- \bar{\eta} S \eta},
   \end{align*}
   and that
   \begin{align*}
      \diff.d.*{\left[S_{ji} + (\bar{\eta} S)_i(S \eta)_j\right]f(\eta, \bar{\eta})}{\bar{\eta}_\ell,\eta_k}
      &= \left[S_{ji} + (\bar{\eta} S)_i (S \eta)_j\right] \diff.d.{f(\eta, \bar{\eta})}{\bar{\eta}_\ell,\eta_k} - S_{\ell i}S_{jk} f(\eta, \bar{\eta}).
   \end{align*}
   The immediate result is
   \begin{align*}
      \diff.d.{e^{- \bar{\eta} S \eta}}{\bar{\eta}_1,\eta_2,\bar{\eta}_3,\eta_4} 
      % &= \diff.d.*{\left[S_{34}  + (\bar{\eta} S)_4 (S \eta)_3\right]e^{- \bar{\eta} S \eta}}{\bar{\eta}_1,\eta_2}\\
      &= \left[S_{34}  + (\bar{\eta} S)_4 (S \eta)_3\right]\left[S_{12}  + (\bar{\eta} S)_2 (S \eta)_1\right]e^{- \bar{\eta} S \eta} - S_{14} S_{32} e^{- \bar{\eta} S \eta},
   \end{align*}
   hence
   \begin{equation*}
      G^{(4)}(x_1 x_2 x_3 x_4) = S_{34} S_{12} - S_{14} S_{32} = S_F(x_1 - x_2) S_F(x_3 - x_4) - S_F(x_1 - x_4) S_F(x_3 - x_2)
   \end{equation*}
   is the four point function. To generalize, we note that instead of the four-point function being the sum over the possible contractions, there is a relative sign for the exchange of two indices, hence
   \begin{equation*}
      G^{(2N)}(x_1, \dots, x_{2N}) = \sum_{\sigma \in S_N} \sgn(\sigma) \prod_{j = 1}^N S_F(x_{2j-1} - x_{2\sigma(j)})
   \end{equation*}
   is the \(2N\)-point function, where \(S_N\) is the permutation group of \(N\) elements.
\end{proof}
   % To verify this, we should get
   % \begin{equation*}
   %    G^{(6)}(x_1, x_2,x_3,x_4,x_5,x_6) = S_{12} S_{34} S_{56} + S_{16} S_{32} S_{54} + S_{14} S_{36} S_{52} - S_{14} S_{32} S_{56} - S_{16} S_{34} S_{52} - S_{12} S_{36} S_{54}
   % \end{equation*}
   % with the explicit computation. We use the previous intermediate results to get
   % \begin{align*}
   %    \diff.d.{e^{- \bar{\eta} S \eta}}{\bar{\eta}_1,\eta_2,\bar{\eta}_3,\eta_4,\bar{\eta}_5,\eta_6}[\substack{\eta = 0\\ \bar{\eta} = 0}]
   %    &= \diff.d.*{\left\{\left[S_{56}  + (\bar{\eta} S)_6 (S \eta)_5\right]\left[S_{34}  + (\bar{\eta} S)_4 (S \eta)_3\right] - S_{36} S_{54} \right\}e^{- \bar{\eta} S \eta}}{\bar{\eta}_1,\eta_2}[\substack{\eta = 0\\ \bar{\eta} = 0}]\\
   %    &= - S_{16}S_{52} S_{34} + S_{56}S_{14}S_{32} - (S_{56}S_{34} - S_{36}S_{54})S_{12},
   % \end{align*}
   % hence
   % \begin{equation*}
   %    G^{(6)}(x_1, x_2, x_3, x_4, x_5, x_6) = 
   % \end{equation*}
