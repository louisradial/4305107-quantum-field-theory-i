% vim: spl=en
\begin{lemma}{Eigenvectors of a real linear combination of Pauli matrices}{pauli}
   Let \(\vetor{n} = n^i \vetor{e}_i \in \mathbb{R}^3\) be a unit vector. Then the eigenvalues of \(\vetor{n}\cdot \vetor{\sigma}\) are \(\set{-1,1}\) and its eigenvectors are
   \begin{equation*}
      \psi_{s} = \begin{pmatrix}
          s( n^1 - i n^2)\\
          1 - s n^3
      \end{pmatrix},
   \end{equation*}
   where \(s = \pm1.\)
\end{lemma}
\begin{proof}
   Since \(\vetor{n} \cdot \vetor{\sigma} = \unity,\) it is clear its eigenvalues are either \(1\) or \(-1\) as we have
   \begin{equation*}
      \vetor{n} \cdot \vetor{\sigma} \psi_s = s \psi_s \implies \psi_s = s^2 \psi_s.
   \end{equation*}
   For \(s \in \set{-1,1},\) we have
   \begin{align*}
      \vetor{n} \cdot \vetor{\sigma} \psi_s &= \begin{pmatrix}
         n^3 && n^1 - i n^2\\
         n^1 + i n^2 && -n^3
      \end{pmatrix}
      \begin{pmatrix}
          s( n^1 - i n^2)\\
          1 - s n^3
      \end{pmatrix}\\
      &= 
      \begin{pmatrix}
         sn^3 (n^1 - in^2) + (n^1 - i n^2) (1 - sn^3)\\
         s(n^1 - in^2)(n^1 + in^2) - n^3(1 - sn^3)
      \end{pmatrix}\\
      &= \begin{pmatrix}
          n^1 - in^2\\
          s\vetor{n}\cdot\vetor{n} - n^3
      \end{pmatrix}\\
      &= s \begin{pmatrix}
          s(n^1 - in^2)\\
          1 - sn^3
      \end{pmatrix}\\
      &= s \psi_s,
   \end{align*}
   that is, \(\psi_s\) are the eigenvectors as claimed.
\end{proof}
\begin{problem}{Simultaneous eigenstates of the helicity operator and free Dirac Hamiltonian}{p1}
   The helicity operator is defined as
   \begin{equation*}
      h = \frac{\vetor{\Sigma}\cdot \vetor{p}}{\norm{\vetor{p}}}.
   \end{equation*}
   \begin{enumerate}[label=(\alph*)]
       \item Show that his operator can be diagonalized simultaneously with the free Dirac Hamiltonian (\(H_D\)).
       \item Obtain the positive and negative energy eigenstates of \(h\) and \(H_D\).
   \end{enumerate}
\end{problem}
\begin{proof}[Solution]
   We consider the Lagrangian density \(\mathcal{L} = \bar{\Psi} (i \slashed{\partial} - m) \Psi\) and its invariance under the infinitesimal Lorentz transformation 
   \begin{equation*}
      x^\mu \mapsto x'^\mu = x^\mu + \omega\indices{^\mu_\nu}x^\nu\quad\text{and}\quad\Psi(x) \mapsto \Psi'(x') = \left(\unity - \frac{i}{4} \omega_{\mu\nu} \sigma^{\mu\nu}\right)\Psi(x).
   \end{equation*}
   As
   \begin{align*}
      \delta \Psi(x) = \Psi'(x) - \Psi(x) &= \left(\unity - \frac{i}{4} \omega_{\mu\nu} \sigma^{\mu\nu}\right)\Psi(x - \omega x) - \Psi(x)\\
                                          &= -\frac{i}{4} \omega_{\mu\nu} \sigma^{\mu\nu}\Psi(x) - \omega\indices{^\mu_\nu} x^\nu\partial_\mu \Psi(x)\\
                                          &= -i \omega_{\mu\nu} \left(\frac14 \omega^{\mu\nu} + \frac1i x^\nu \partial^\mu\right)\Psi(x)
   \end{align*}
   and as the conserved current is \(J^\rho = i \bar{\Psi} \gamma^\rho \delta \Psi,\) the conserved charge for a rotation about the \(\vetor{n}\) axis by the infinitesimal angle \(\vartheta\), where \(\omega_{jk} = \vartheta n^i \epsilon_{ijk},\) is
   \begin{equation*}
      \int_{\mathbb{R}^3} \dln3x J^0 = \int_{\mathbb{R}^3} \dln3x \herm{\Psi} \vartheta n^i \epsilon_{ijk} \left(\frac14 \sigma^{jk} + \frac{1}{i} x^{k} \partial^j\right) \Psi(x) = \int_{\mathbb{R}^3} \dln3x \herm{\Psi} \left(\frac{\vartheta}2 n^i \Sigma_i + \vartheta n^i L_i\right)\Psi(x),
   \end{equation*}
   where \(L_i\) is the \(i\) component of the orbital angular momentum and 
   \begin{equation*}
      \Sigma_i = \frac12 \epsilon_{ijk} \sigma^{jk} = \frac{i}{4} \epsilon_{ijk} \commutator{\gamma^j}{\gamma^k} = \frac{i}{4} \epsilon_{ijk} \left(\anticommutator{\gamma^j}{\gamma^k} - 2\gamma^{k} \gamma^{j}\right) = -\frac{i}{2} \epsilon_{ijk} \gamma^{k} \gamma^{j} = \frac{i}{2} \epsilon_{ijk} \gamma^{j} \gamma^{k}
   \end{equation*}
   is the \(i\) component of the spin angular momentum, up to a factor of two. In the Weyl representation, where
   \begin{equation*}
      \gamma^0 = \begin{pmatrix}
           && \unity\\
         \unity && 
      \end{pmatrix}
      \quad\text{and}\quad
      \gamma^j = \begin{pmatrix}
           && \sigma^j\\
         -\sigma^j &&
      \end{pmatrix},
   \end{equation*}
   we have
   \begin{equation*}
      \Sigma_i = \frac12 \epsilon_{ijk} \gamma^j \gamma^k = -\frac12 \epsilon_{ijk} (\delta_{jk} + i \epsilon_{jk\ell} \sigma^{\ell}) \unity = -\delta_{i \ell} \sigma_\ell \unity \implies \Sigma^i = \begin{pmatrix}
         \sigma^i &&\\
                  && \sigma^i
      \end{pmatrix}.
   \end{equation*}

   From the Dirac equation we have
   \begin{equation*}
      i\slashed{\partial}\Psi = m \Psi \implies i \gamma^0\partial_0 \Psi = (m - i \gamma^j \partial_j)\Psi \implies i \partial_0 \Psi = \gamma^0 (m - i \gamma^j \partial_j) \Psi,
   \end{equation*}
   that is, 
   \begin{equation*}
      H_D = \gamma^0 (m + \gamma^j p_j)
   \end{equation*}
   is the free Dirac Hamiltonian. In the Weyl representation, we have
   \begin{equation*}
      H_D = \begin{pmatrix}
         -\sigma^j p_j && m\\
         m && \sigma^j p_j
         \end{pmatrix} = \begin{pmatrix}
         \vetor{\sigma} \cdot \vetor{p} && m\\
         m && -\vetor{\sigma} \cdot \vetor{p}
      \end{pmatrix},
   \end{equation*}
   where \(\vetor{\sigma} \cdot \vetor{p} = \delta_{ij} \sigma^i p^j.\) Notice the helicity, \(h = \frac{\vetor{\Sigma} \cdot \vetor{p}}{\norm{\vetor{p}}},\) satisfies
   \begin{align*}
      \commutator{h}{H_D} 
      &= \commutator*{\begin{pmatrix}
            \vetor{\sigma} \cdot \frac{\vetor{p}}{\norm{\vetor{p}}} &&\\
                                                                    && \vetor{\sigma} \cdot \frac{\vetor{p}}{\norm{\vetor{p}}}
      \end{pmatrix}}
      {\begin{pmatrix}
            \vetor{\sigma} \cdot \vetor{p} && m\\
            m && -\vetor{\sigma} \cdot \vetor{p}
      \end{pmatrix}}\\
      &=
      \begin{pmatrix}
         \left(\vetor{\sigma}\cdot\frac{\vetor{p}}{\norm{\vetor{p}}}\right)(\vetor{\sigma}\cdot \vetor{p}) && \left(\vetor{\sigma}\cdot\frac{\vetor{p}}{\norm{\vetor{p}}}\right)m\\
         \left(\vetor{\sigma}\cdot\frac{\vetor{p}}{\norm{\vetor{p}}}\right)m && -\left(\vetor{\sigma}\cdot\frac{\vetor{p}}{\norm{\vetor{p}}}\right)(\vetor{\sigma} \cdot \vetor{p})
      \end{pmatrix}-
      \begin{pmatrix}
         (\vetor{\sigma}\cdot \vetor{p})\left(\vetor{\sigma}\cdot\frac{\vetor{p}}{\norm{\vetor{p}}}\right) && m \left(\vetor{\sigma}\cdot\frac{\vetor{p}}{\norm{\vetor{p}}}\right)\\
         m \left(\vetor{\sigma}\cdot\frac{\vetor{p}}{\norm{\vetor{p}}}\right) && - (\vetor{\sigma} \cdot \vetor{p})\left(\vetor{\sigma}\cdot\frac{\vetor{p}}{\norm{\vetor{p}}}\right)
      \end{pmatrix}\\
      &= 0,
   \end{align*}
   then \(h\) and \(H_D\) are compatible, that is, there exists a basis with which both operators can be diagonalized simultaneously.

   Let \(\Psi_s\) be an element of such basis, that is, \(h \Psi_s = s \Psi_s\) and \(H_D \Psi_s = E \Psi_s.\) Notice \(h^2 = \unity,\) then \(s \in \set{-1,1}.\) If we write \(\vetor{p} = \norm{\vetor{p}} \left(\cos\phi \sin\theta \vetor{e}_1 + \sin\phi\cos\theta \vetor{e}_2 + \cos\theta \vetor{e}_3\right),\) we learn from \cref{lem:pauli} that
   \begin{equation*}
      \Psi_s = \begin{pmatrix}
         \lambda_s \psi_s\\
         \mu_s \psi_s
      \end{pmatrix},
      \quad\text{where}\quad
      \psi_s = \begin{pmatrix}
         se^{-i\phi} \sin\theta\\
         1 - s \cos\theta
      \end{pmatrix},
   \end{equation*}
   with \(\lambda_s, \mu_s\) depending on the energy eigenvalue. Using this on the energy eigenvalue problem yields
   \begin{equation*}
      \begin{cases}
         \lambda_s s \norm{\vetor{p}} \psi_s + m \mu_s \psi_s = \lambda_s E \psi_s\\
         m \lambda_s  \psi_s  - \mu_s s  \norm{\vetor{p}} \psi_s =  \mu_s E \psi_s
      \end{cases}
      \implies
      \begin{cases}
         \lambda_s (E - s \norm{\vetor{p}}) = m \mu_s\\
         \mu_s (E + s \norm{\vetor{p}}) = m \lambda_s
      \end{cases}
      % \implies
      % \begin{cases}
      %    \lambda_s (E - s \sqrt{E^2 - m^2}) = m \mu_s\\
      %    \mu_s (E + s \sqrt{E^2 - m^2}) = m \lambda_s
      % \end{cases}
   \end{equation*}
   as \(\psi_s\) is not a null spinor. For a massive particle, we have \(E \pm s \norm{\vetor{p}} \neq 0,\) then
   \begin{equation*}
      \Psi_s = \mu_s
      \begin{pmatrix}
         \frac{m}{E - s\norm{\vetor{p}}} \psi_s\\
         \psi_s
      \end{pmatrix}.
   \end{equation*}
   Imposing the normalization condition \(\herm{\Psi_s}\Psi_s = 2 E\) yields
   \begin{align*}
      \abs{\mu_s}^2 \left[\left(\frac{m}{E - s \norm{\vetor{p}}}\right)^2 + 1\right]\herm{\psi_s}\psi_s = 2E
      &\implies \abs{\mu_s}^2 \frac{2E^2 - 2s E\norm{\vetor{p}}}{(E - s \norm{\vetor{p}})^2}(1 - s\cos\theta) = E\\
      &\implies (1 - s\cos\theta)\abs{\mu_s}^2 = \frac12 \abs*{E - s \norm{\vetor{p}}},
   \end{align*}
   and we choose the phase of \(\mu_s\) such that
   \begin{equation*}
      \mu_s = \sqrt{\frac{E - s \norm{\vetor{p}}}{2 (1 - s \cos\theta)}},
   \end{equation*}
   hence
   \begin{equation*}
      \Psi_s = \sqrt{E - s \norm{\vetor{p}}} \begin{pmatrix}
         \frac{sm}{E - s \norm{\vetor{p}}} e^{-i\phi} \frac{\sin\frac{\theta}{2} \cos\frac\theta2}{\sqrt{\frac{1 - s \cos\theta}2}}\\
         \frac{m}{E - s \norm{\vetor{p}}} \sqrt{\frac{1 - s \cos\theta}{2}}\\
         s e^{-i\phi} \frac{\sin\frac{\theta}{2} \cos\frac\theta2}{\sqrt{\frac{1 - s \cos\theta}2}}\\
         \sqrt{\frac{1 - s \cos\theta}{2}}
      \end{pmatrix},
   \end{equation*}
   that is,
   \begin{equation*}
      \Psi_\uparrow =  \begin{pmatrix}
         \frac{m}{\sqrt{E - \norm{\vetor{p}}}} e^{-i\phi} \cos\frac\theta2\\
         \frac{m}{\sqrt{E - \norm{\vetor{p}}}} \sin\frac\theta2\\
         \sqrt{E - \norm{\vetor{p}}}e^{-i\phi} \cos\frac\theta2\\
         \sqrt{E - \norm{\vetor{p}}}\sin\frac\theta2
      \end{pmatrix}
      =
      \begin{pmatrix}
         \sqrt{E + \norm{\vetor{p}}} e^{-i\phi} \cos\frac\theta2\\
         \sqrt{E + \norm{\vetor{p}}} \sin\frac\theta2\\
         \sqrt{E - \norm{\vetor{p}}}e^{-i\phi} \cos\frac\theta2\\
         \sqrt{E - \norm{\vetor{p}}}\sin\frac\theta2
      \end{pmatrix}
   \end{equation*}
   and
   \begin{equation*}
      \Psi_\downarrow =  \begin{pmatrix}
         -\frac{m}{\sqrt{E + \norm{\vetor{p}}}} e^{-i\phi} \sin\frac\theta2\\
         \frac{m}{\sqrt{E + \norm{\vetor{p}}}} \cos\frac\theta2\\
         -\sqrt{E + \norm{\vetor{p}}}e^{-i\phi} \sin\frac\theta2\\
         \sqrt{E + \norm{\vetor{p}}}\cos\frac\theta2
      \end{pmatrix}
      =
      \begin{pmatrix}
         -\sqrt{E - \norm{\vetor{p}}} e^{-i\phi} \sin\frac\theta2\\
         \sqrt{E - \norm{\vetor{p}}} \cos\frac\theta2\\
         -\sqrt{E + \norm{\vetor{p}}}e^{-i\phi} \sin\frac\theta2\\
         \sqrt{E + \norm{\vetor{p}}}\cos\frac\theta2
      \end{pmatrix},
   \end{equation*}
   where we have used that \(m = \sqrt{E^2 - \norm{\vetor{p}}^2} = \sqrt{E - \norm{\vetor{p}}} \sqrt{E + \norm{\vetor{p}}}.\) Finally, the positive energy solutions, \(H_D u_s = E u_s\) with \(E > 0,\) are
   \begin{equation*}
      u_\uparrow = \begin{pmatrix}
         \sqrt{E + \norm{\vetor{p}}} e^{-i\phi} \cos\frac\theta2\\
         \sqrt{E + \norm{\vetor{p}}} \sin\frac\theta2\\
         \sqrt{E - \norm{\vetor{p}}}e^{-i\phi} \cos\frac\theta2\\
         \sqrt{E - \norm{\vetor{p}}}\sin\frac\theta2
      \end{pmatrix},
      \quad\text{and}\quad
      u_\downarrow =  \begin{pmatrix}
         -\sqrt{E - \norm{\vetor{p}}} e^{-i\phi} \sin\frac\theta2\\
         \sqrt{E - \norm{\vetor{p}}} \cos\frac\theta2\\
         -\sqrt{E + \norm{\vetor{p}}}e^{-i\phi} \sin\frac\theta2\\
         \sqrt{E + \norm{\vetor{p}}}\cos\frac\theta2
      \end{pmatrix},
   \end{equation*}
   and the negative energy solutions, \(H_D v_s = - E v_s\) with \(E > 0\), are
   \begin{equation*}
      v_\uparrow = \begin{pmatrix}
         i\sqrt{E - \norm{\vetor{p}}} e^{-i\phi} \cos\frac\theta2\\
         i\sqrt{E - \norm{\vetor{p}}} \sin\frac\theta2\\
         i\sqrt{E + \norm{\vetor{p}}}e^{-i\phi} \cos\frac\theta2\\
         i\sqrt{E + \norm{\vetor{p}}}\sin\frac\theta2
      \end{pmatrix},
      \quad\text{and}\quad
      v_\downarrow =  \begin{pmatrix}
         -i\sqrt{E + \norm{\vetor{p}}} e^{-i\phi} \sin\frac\theta2\\
         i\sqrt{E + \norm{\vetor{p}}} \cos\frac\theta2\\
         -i\sqrt{E - \norm{\vetor{p}}}e^{-i\phi} \sin\frac\theta2\\
         i\sqrt{E - \norm{\vetor{p}}}\cos\frac\theta2
      \end{pmatrix}.
   \end{equation*}
   Notice we may take the \(m \to 0\) limit without problems yielding
   \begin{equation*}
      u_\uparrow = \begin{pmatrix}
         \sqrt{2E} e^{-i\phi} \cos\frac\theta2\\
         \sqrt{2E} \sin\frac\theta2\\
         0\\
         0
      \end{pmatrix},
      \quad\text{and}\quad
      u_\downarrow =  \begin{pmatrix}
         0\\
         0\\
         -\sqrt{2E}e^{-i\phi} \sin\frac\theta2\\
         \sqrt{2E}\cos\frac\theta2
      \end{pmatrix},
   \end{equation*}
   \begin{equation*}
      v_\uparrow = \begin{pmatrix}
         0\\
         0\\
         i\sqrt{2E}e^{-i\phi} \cos\frac\theta2\\
         i\sqrt{2E}\sin\frac\theta2
      \end{pmatrix},
      \quad\text{and}\quad
      v_\downarrow =  \begin{pmatrix}
         -i\sqrt{2E} e^{-i\phi} \sin\frac\theta2\\
         i\sqrt{2E} \cos\frac\theta2\\
         0\\
         0\\
      \end{pmatrix},
   \end{equation*}
   which are the massless solutions.
\end{proof}
\begin{proof}[Alternative solution for the eigenstates up to a rotation]
   Let \(\Psi\) be an element of such basis, where we have \(h \Psi = s \Psi\) and \(H_D \Psi = E \Psi.\) In the Weyl representation, let us write \(\Psi = \left(\begin{smallmatrix}
         \Psi_1\\
         \Psi_2
   \end{smallmatrix}\right),\) then,
   \begin{equation*}
      \begin{cases}
         \vetor{\sigma}\cdot \vetor{p} \Psi_i = s \norm{\vetor{p}} \Psi_i\\
         \vetor{\sigma} \cdot \vetor{p} \Psi_1 + m \Psi_2 = E \Psi_1\\
         m \Psi_1 - \vetor{\sigma} \cdot \vetor{p} \Psi_2 = E \Psi_2
      \end{cases}
      \implies
      \begin{cases}
         \vetor{\sigma}\cdot \vetor{p} \Psi_i = s \norm{\vetor{p}} \Psi_i\\
         s \norm{\vetor{p}} \Psi_1 + m \Psi_2 = E \Psi_1\\
         m \Psi_1 - s \norm{\vetor{p}} \Psi_2 = E \Psi_2.
      \end{cases}
   \end{equation*}
   We may use a rotated frame for which \(\frac{\vetor{p}}{\norm{\vetor{p}}} = \vetor{e}_z,\) then the eigenstate equation for the helicity reads as \(\sigma_z \Psi_i = s \Psi_i,\) that is, \(s = \pm 1\) and we see \(\Psi_i\) is an eigenvector of \(\sigma_z\) in this frame. We'll first consider massive particles, and after a boost to its rest frame, we have \(E = \pm m\), hence
   \begin{equation*}
      \begin{cases}
         m \Psi_2 = \pm m \Psi_1\\
         m \Psi_1 = \pm m \Psi_2
      \end{cases} \implies \Psi_1 = \pm \Psi_2.
   \end{equation*}
   We have thus the following set of linearly independent solutions
   \begin{equation*}
      \Psi^+_\uparrow = \sqrt{m} \begin{pmatrix}
         1 \\ 0\\1 \\ 0
      \end{pmatrix},
      \quad
      \Psi^+_\downarrow = \sqrt{m} \begin{pmatrix}
         0 \\ 1\\0 \\ 1
      \end{pmatrix},
      \quad
      \Psi^-_\uparrow = \sqrt{m} \begin{pmatrix}
         1 \\ 0\\-1 \\ 0
      \end{pmatrix},
      \quad\text{and}\quad
      \Psi^-_\downarrow = \sqrt{m} \begin{pmatrix}
         0 \\ 1\\0 \\-1
      \end{pmatrix},
   \end{equation*}
   where the sign reflects that of the energy and the arrow denotes the helicity. We'll write
   \begin{equation*}
      \Psi^\pm_{s} = \sqrt{m}\begin{pmatrix} \xi_s\\\pm\xi_s \end{pmatrix},\quad\text{where}\quad \xi_s \in \set*{\begin{pmatrix} 1 \\0 \end{pmatrix}, \begin{pmatrix} 0\\1 \end{pmatrix}},
   \end{equation*}
   and boost each rest frame solution back to the frame where \(p^\mu = (E, \norm{\vetor{p}} \vetor{e}_z)\). The rapidity \(\phi\) for the boost is such that \(\norm{\vetor{p}} = \pm m \sinh\phi\) and \(E = m \cosh\phi\) then the spinor transformation is
   \begin{align*}
      \exp\left(\frac14 \phi \sigma^{03}\right) 
      &= \exp\left[\frac12 \phi \begin{pmatrix}
            - \sigma_z && \\
                       && \sigma_z
      \end{pmatrix}\right]\\
      &= \left[\cosh\left(\frac12 \phi\right) \unity + \sinh\left(\frac12 \phi\right)\begin{pmatrix}
            - \sigma_z && \\
                       && \sigma_z
      \end{pmatrix}\right]\\
      &= \begin{pmatrix}
         e^{\frac{\phi}2} \left(\frac{\unity - \sigma_z}{2}\right) + e^{-\frac{\phi}2} \left(\frac{\unity + \sigma_z}{2}\right) &&\\ && e^{\frac{\phi}2} \left(\frac{\unity + \sigma_z}2\right) + e^{-\frac{\phi}2} \left(\frac{\unity - \sigma_z}{2}\right)
      \end{pmatrix}\\
      &= \frac{1}{\sqrt{m}}\begin{pmatrix}
         \sqrt{E \pm \norm{\vetor{p}}} \left(\frac{\unity - \sigma_z}{2}\right) + \sqrt{E \mp \norm{\vetor{p}}} \left(\frac{\unity + \sigma_z}2\right)&&\\ &&\sqrt{E \pm \norm{\vetor{p}}} \left(\frac{\unity + \sigma_z}{2}\right) + \sqrt{E \mp \norm{\vetor{p}}} \left(\frac{\unity - \sigma_z}2\right)
      \end{pmatrix}.
   \end{align*}
   The boosted solutions are
   \begin{equation*}
      \Psi_s^{\pm} = \begin{pmatrix}
         \left[\sqrt{E \pm \norm{\vetor{p}}} \left(\frac{1 - s}{2}\right) + 
         \sqrt{E \mp \norm{\vetor{p}}} \left(\frac{1 + s}{2}\right)\right] \xi_s\\
         \pm\left[\sqrt{E \pm \norm{\vetor{p}}} \left(\frac{1 + s}{2}\right) + 
         \sqrt{E \mp \norm{\vetor{p}}} \left(\frac{1 - s}{2}\right)\right]\xi_s
      \end{pmatrix},
   \end{equation*}
   or
   \begin{equation*}
      \Psi^+_\uparrow = \begin{pmatrix}
         \sqrt{E - \norm{\vetor{p}}}\\
         0\\
         \sqrt{E + \norm{\vetor{p}}}\\
         0
      \end{pmatrix},\quad
      \Psi^+_\downarrow = \begin{pmatrix}
         0\\
         \sqrt{E + \norm{\vetor{p}}}\\
         0\\
         \sqrt{E - \norm{\vetor{p}}}
      \end{pmatrix},\quad
      \Psi^-_\uparrow = \begin{pmatrix}
         \sqrt{E + \norm{\vetor{p}}}\\
         0\\
         -\sqrt{E - \norm{\vetor{p}}}\\
         0
      \end{pmatrix},\quad\text{and}\quad
      \Psi^-_\downarrow = \begin{pmatrix}
         0\\
         \sqrt{E - \norm{\vetor{p}}}\\
         0\\
         -\sqrt{E + \norm{\vetor{p}}}
      \end{pmatrix}.
   \end{equation*}
   We now consider the Hamiltonian for a massless particle and the solutions above. For a massless particle we have \(E = \pm \norm{\vetor{p}},\) which means
   \begin{equation*}
      \Psi^+_\uparrow = \begin{pmatrix}
         0\\
         0\\
         \sqrt{2E}\\
         0
      \end{pmatrix},\quad
      \Psi^+_\downarrow = \begin{pmatrix}
         0\\
         \sqrt{2E}\\
         0\\
         0
      \end{pmatrix},\quad
      \Psi^-_\uparrow = \begin{pmatrix}
         0\\
         0\\
         -\sqrt{2E}\\
         0
      \end{pmatrix},\quad\text{and}\quad
      \Psi^-_\downarrow = \begin{pmatrix}
         0\\
         \sqrt{2E}\\
         0\\
         0
      \end{pmatrix},
   \end{equation*}
   are the independent energy eigenstates.
\end{proof}

   % We'll show the Hamiltonian and the helicity, \(h = \frac{\vetor{\Sigma} \cdot \vetor{p}}{\norm{\vetor{p}}} = g^{ij}\frac{\Sigma_i p_j}{\norm{\vetor{p}}},\) are compatible operators
   % \begin{align*}
   %    \commutator*{h}{H_D} %&= \commutator*{\frac{\vetor{\Sigma}\cdot\vetor{p}}{\norm{\vetor{p}}}}{\gamma^0 (m - \vetor{\gamma}\cdot \vetor{p})}\\
   %                        &= g^{ij}\commutator*{\frac{\Sigma_i p_j}{\norm{\vetor{p}}}}{\gamma^0\left(m + \gamma^k p_k\right)}\\
   %                        &= \frac{i}{2} g^{ij} \epsilon_{iab} \frac{p_j}{\norm{\vetor{p}}}\commutator*{\gamma^a \gamma^b}{\gamma^0\left(m + \gamma^kp_k\right)}\\
   %                        &= \frac{i}{2} g^{ij} \epsilon_{iab} \frac{p_j}{\norm{\vetor{p}}}\left(m\commutator*{\gamma^a \gamma^b}{\gamma^0} + \commutator*{\gamma^a \gamma^b}{\gamma^0 \gamma^k}p_k\right)\\
   %                        &= \frac{i}{2} g^{ij} \epsilon_{iab} \frac{p_j}{\norm{\vetor{p}}}p_k\commutator*{\gamma^a \gamma^b}{\gamma^0 \gamma^k}\\
   %                        &= \frac{i}{2} g^{ij} \epsilon_{iab} \frac{p_j}{\norm{\vetor{p}}}p_k\left(\commutator*{\gamma^a \gamma^b}{\gamma^0}\gamma^k + \gamma^0\commutator*{\gamma^a \gamma^b}{\gamma^k}\right)\\
   %                        &= \frac{i}{2} g^{ij} \epsilon_{iab} \frac{p_j}{\norm{\vetor{p}}}p_k\gamma^0\commutator*{\gamma^a \gamma^b}{\gamma^k}\\
   %                        &= \frac{i}{2} g^{ij} \epsilon_{iba} \frac{p_j}{\norm{\vetor{p}}}p_k\gamma^0\left(\anticommutator*{\gamma^k}{\gamma^a}\gamma^b - \gamma^a\anticommutator*{\gamma^k}{\gamma^b}\right)\\
   %                        &= i g^{ij} \epsilon_{iba} \frac{p_j}{\norm{\vetor{p}}}p_k\gamma^0\left(g^{ka}\gamma^b - \gamma^ag^{kb}\right)\\
   %                        &= \\
   %                        % &= \frac{i}{2} g^{ij} \epsilon_{iba} \frac{p_j}{\norm{\vetor{p}}} \left[\anticommutator*{\gamma^0\left(m + \gamma^k p_k\right)}{\gamma^a} \gamma^b - \gamma^a \anticommutator*{\gamma^0\left(m + \gamma^k p_k\right)}{\gamma^b}\right]\\
   %                        % &= \frac{i}{2} g^{ij} \epsilon_{iba} p_k \frac{p_j}{\norm{\vetor{p}}} \left(\anticommutator*{\gamma^0\gamma^k}{\gamma^a} \gamma^b - \gamma^a \anticommutator*{\gamma^0\gamma^k}{\gamma^b} + mg^{0a} \gamma^b - m g^{0b}\gamma^a\right)\\
   %                        % &= \frac{i}{2} g^{ij} \epsilon_{iba} p_k \frac{p_j}{\norm{\vetor{p}}} \left(\commutator*{\gamma^0\gamma^k}{\gamma^a} \gamma^b + \gamma^a \commutator*{\gamma^0\gamma^k}{\gamma^b}\right)\\
   %                        % &= \frac{i}{2} g^{ij} \epsilon_{iab} p_k \frac{p_j}{\norm{\vetor{p}}} \left(\anticommutator*{\gamma^a}{\gamma^0} \gamma^k \gamma^b - \gamma^0 \anticommutator*{\gamma^a}{\gamma^k} \gamma^b + \gamma^a \anticommutator*{\gamma^b}{\gamma^0} \gamma^k - \gamma^a \gamma^0 \anticommutator*{\gamma^b}{\gamma^k}\right)\\
   %                        % &= i g^{ij} \epsilon_{iba} p_k \frac{p_j}{\norm{\vetor{p}}} \left(g^{ak}\gamma^0  \gamma^b + g^{bk}\gamma^a \gamma^0 \right)\\
   %                        % &= \frac{i}{2} g^{ij} \epsilon_{iba} p_k \frac{p_j}{\norm{\vetor{p}}} \left(g^{ak}\gamma^0  \gamma^b + g^{bk}\gamma^a \gamma^0 - g^{bk} \gamma^0 \gamma^a - g^{ak} \gamma^b \gamma^0\right)\\
   %                        % &= \frac{i}{2} g^{ij} \epsilon_{iba} p_k \frac{p_j}{\norm{\vetor{p}}} \left[g^{ak} \commutator*{\gamma^0}{\gamma^b} + g^{bk} \commutator*{\gamma^a}{\gamma^0}\right]\\
   %                        % &= g^{ij} \epsilon_{iba} p_k \frac{p_j}{\norm{\vetor{p}}} \left(g^{ak} \sigma^{0b} - g^{bk} \sigma^{0a}\right)
   % \end{align*}
