\begin{problem}{Conserved quantity due to electromagnetism's Lorentz invariance}{p4}
   Consider the vector field \(A^\mu\) associated to the electromagnetic field. Obtain the conserved quantities due to the Lorentz invariance of the theory. Use the Langrangian density given in the first problem.
\end{problem}
\begin{proof}[Solution]
   We consider the infinitesimal Lorentz transformation
   \begin{equation*}
      x^\mu \mapsto \tilde{x}^\mu = \left(\delta\indices{^\mu_\nu} + \omega\indices{^\mu_\nu}\right)x^\nu
      \quad\text{and}\quad
      A_\mu(x) \mapsto \tilde{A}_\mu(\tilde{x}) = \diffp{x^\nu}{\tilde{x}^\mu} A_\nu(x).
   \end{equation*}
   As the transformation is infinitesimal, we have \(\diffp{x^\nu}{\tilde{x}^\mu} = \delta\indices{^\nu_\mu} - \omega\indices{^\nu_\mu}\), hence
   \begin{align*}
      \delta A_\mu(x) = \tilde{A}_\mu(x) - A_\mu(x) 
      &=\left(\delta\indices{^\nu_\mu} - \omega\indices{^\nu_\mu}\right) A_\nu(x - \omega x) - A_\mu(x)\\
      &=\left(\delta\indices{^\nu_\mu} - \omega\indices{^\nu_\mu}\right) \left[A_\nu(x) - \omega\indices{^\rho_\sigma} x^\sigma \partial_\rho A_\nu(x)\right] - A_\mu(x)\\
      &=  - \omega\indices{^\nu_\mu} A_\nu(x) - \omega\indices{^\lambda_\sigma} x^\sigma \partial_\lambda A_\mu(x)
   \end{align*}
   and \(\Delta x^\rho = \omega\indices{^\rho_\eta}x^\eta\). As we have shown in \cref{prob:p1}, we have
   \begin{equation*}
      \diffp{\mathcal{L}}{(\partial_\rho A_\mu)} = - F^{\rho\mu},
   \end{equation*}
   then the conserved current guaranteed by \nameref{thm:noether} is
   \begin{align*}
      j^\rho = \diffp{L}{(\partial_\rho A_\mu)}\delta A_\mu(x) + \mathcal{L} \Delta x^\rho
      &= F^{\rho\mu} \left[\omega\indices{^\sigma_\mu} A_\sigma + \omega\indices{^\sigma_\lambda} x^\lambda \partial_\sigma A_\mu\right] - \frac14 F^{\alpha \beta}F_{\alpha \beta} \omega\indices{^\rho_\lambda} x^\lambda\\
      &= \omega\indices{^\sigma_\lambda}\left[F^{\rho \mu}\left(\delta\indices{^\lambda_\mu} A_\sigma + x^\lambda \partial_\sigma A_\mu\right) - \frac14 F^{\alpha \beta}F_{\alpha \beta} \delta\indices{^\rho_\sigma}x^\lambda\right]\\
      &= \omega_{\nu \lambda}\left[F^{\rho \lambda} A^\nu + x^\lambda \left(F^{\rho \mu} \partial^\nu A_\mu - \frac14 g^{\rho \nu} F^{\alpha \beta} F_{\alpha \beta}\right)\right]\\
      &= \omega_{\nu \lambda} \left[F^{\rho \lambda} A^\nu + x^\lambda F^{\rho \mu} \partial_\mu A^\nu + x^\lambda T^{\rho \nu}\right]\\
      &= \omega_{\nu \lambda} \left[F^{\rho \mu} \delta\indices{^\lambda_\mu} A^\nu + x^\lambda F^{\rho \mu} \partial_\mu A^\nu + x^\lambda T^{\rho \nu}\right]\\
      &= \omega_{\nu \lambda} \left[F^{\rho \mu} A^\nu \partial_\mu x^\lambda + x^\lambda F^{\rho \mu} \partial_\mu A^\nu + x^\lambda T^{\rho \nu}\right]\\
      &= \omega_{\nu \lambda} \left[F^{\rho \mu} \partial_\mu( x^\lambda A^\nu) + x^\lambda T^{\rho \nu}\right]\\
      &= \frac12 \omega_{\nu \lambda} \left[F^{\rho \mu} \partial_\mu( x^\lambda A^\nu) + x^\lambda T^{\rho \nu}\right] + \frac12 \omega_{\lambda \nu}\left[F^{\rho \mu} \partial_\mu( x^\nu A^\lambda) + x^\nu T^{\rho \lambda}\right]\\
      &= \frac12 \omega_{\nu \lambda} \left[x^{\lambda} T^{\rho \nu} - x^\nu T^{\rho \lambda} + F^{\rho \mu} \partial_\mu (x^\lambda A^\nu - x^\nu A^{\lambda})\right] 
   \end{align*}
   where \(T^{\rho\nu}\) is the symmetric stress-energy tensor defined in \cref{prob:p1}. As the Lorentz transformation components are arbitrary, the tensor
   \begin{equation*}
      \tilde{M}^{\lambda \rho \nu} = x^{\lambda} T^{\rho \nu} - x^\nu T^{\rho \lambda} + F^{\rho \mu} \partial_\mu(x^\lambda A^\nu - x^\nu A^\lambda)
   \end{equation*}
   is conserved with \(\partial_\rho \tilde{M}^{\lambda \rho \nu} = 0\). The equations of motion then yield the conservation of the tensor \(M^{\lambda \rho \nu} = x^{\lambda}T^{\rho \nu} - x^{\nu}T^{\rho \lambda}\) since we have
   \begin{equation*}
      \partial_\rho (\tilde{M}^{\lambda \rho \nu} - M^{\lambda \rho \nu}) = \partial_\rho \left[F^{\rho \mu} \partial_\mu(x^\lambda A^\nu - x^\nu A^\lambda)\right]  = F^{\rho \mu} \partial_\rho \partial_\mu (x^\lambda A^\nu - x^\nu A^\lambda) = 0
   \end{equation*}
   since \(F^{\rho \mu}\) is antisymmetric and the partial derivatives commute.
\end{proof}
