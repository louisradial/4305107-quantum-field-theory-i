% vim: spl=en
\begin{problem}{Gauge invariance of the Compton scattering amplitude}{p5}
   The lowest order contribution to the process in the previous problem possesses two contributions \(\mathcal{M} = \mathcal{M}_1 + \mathcal{M}_2.\) 
   \begin{enumerate}[label=(\alph*)]
      \item Show that the scattering amplitude \(\mathcal{M}_1\) is not gauge invariant, that is, it changes when we change \(\epsilon_\mu(k) \to \epsilon_\mu(k) + \lambda k_\mu\) for both photons.
      \item Repeat the previous item for \(\mathcal{M}_2.\)
      \item Show that \(\mathcal{M}\) is gauge invariant.
   \end{enumerate}
\end{problem}
\begin{proof}[Solution]
   We will consider gauge invariance of \(\mathcal{M} = \mathcal{M}_s + \mathcal{M}_t,\) where
   \begin{equation*}
      i\mathcal{M}_s = i(ie)^2 \bar{u}(k_4) \slashed{\epsilon}^*(k_3) \frac{\slashed{k}_1 + \slashed{k}_2 + m}{s - m^2} \slashed{\epsilon}(k_2) u(k_1)
      \;\text{and}\;
      i\mathcal{M}_t = i (i e)^2\bar{u}(k_4) \slashed{\epsilon}(k_2) \frac{\slashed{k}_1 - \slashed{k}_3 + m}{t - m^2} \slashed{\epsilon}^*(k_3) u(k_1)
   \end{equation*}
   and we will use the gauge transformations
   \begin{equation*}
      \epsilon_\mu(k_3) \to \tilde{\epsilon}_\mu(k_3) = \epsilon_\mu(k_3) + \xi {k_3}_\mu
      \quad\text{and}\quad
      \epsilon_\mu(k_2) \to \tilde{\epsilon}_\mu(k_2) = \epsilon_\mu(k_2) + \zeta {k_2}_\mu.
   \end{equation*}
   To verify gauge invariance, we will consider the following definitions
   \begin{equation*}
      \mathcal{A}_s = \frac{i\tilde{\mathcal{M}}_s - i\mathcal{M}_s}{i(ie)^2}
      \quad\text{and}\quad
      \mathcal{A}_t = \frac{i\tilde{\mathcal{M}}_t - i\mathcal{M}_s}{i(ie)^2},
   \end{equation*}
   check that they are non-vanishing but that \(\mathcal{A}_s + \mathcal{A}_t = 0.\) 

   For the \(s\) channel, we have
   \begin{equation*}
      \mathcal{A}_s = \bar{u}_4 \left[\xi\slashed{k}_3 \frac{\slashed{k}_1 + \slashed{k}_2 + m}{s - m^2} \slashed{\epsilon}_2 + \xi \zeta \slashed{k}_3 \frac{\slashed{k}_1 + \slashed{k}_2 + m}{s - m^2} \slashed{k}_2 + \zeta \slashed{\epsilon}^*_3 \frac{\slashed{k}_1 + \slashed{k}_2 + m}{s - m^2}\slashed{k}_2 \right] u_1,
   \end{equation*}
   where we are using the shorthand \(\slashed{\epsilon}_2 = \slashed{\epsilon}(k_2)\) and similarly for \(u\) and \(\bar{u}.\) Notice
   \begin{equation*}
      \slashed{k}_2 u_1 = \left[(\slashed{k}_1 + \slashed{k}_2 - m) - (\slashed{k}_1 - m)\right]u_1 = (\slashed{k}_1 + \slashed{k}_2 - m) u_1
   \end{equation*}
   by the Dirac equation \((\slashed{p} - m) u(p) = 0,\) then as
   \begin{equation*}
      (\slashed{p} \pm m)(\slashed{p} \mp m) = p_\mu p_\nu \gamma^\mu \gamma^\nu - m^2 = \frac12 p_\mu p_\nu \anticommutator{\gamma^\mu}{\gamma^\nu} - m^2 = p_\mu p^\mu - m^2,
   \end{equation*}
   we have
   \begin{equation*}
      \frac{\slashed{k}_1 + \slashed{k}_2 + m}{s - m^2}\slashed{k}_2 u_1 = \frac{(k_1 + k_2)^2 - m^2}{s - m^2} u_1 = u_1.
   \end{equation*}
   Now, recall that the matrix elements \(i \mathcal{M}\) are accompanied by a delta function that establishes momentum conservation, then \(\slashed{k}_3 = \slashed{k}_1 + \slashed{k}_2 - \slashed{k}_4\) and we have
   \begin{equation*}
       \bar{u}_4 \slashed{k}_3 = \bar{u}_4 (\slashed{k}_1 + \slashed{k}_2 - \slashed{k}_4) = \bar{u}_4(\slashed{k}_1 + \slashed{k}_2 - m),
   \end{equation*}
   where we have used the Dirac equation \(\bar{u}(p) (\slashed{p} - m) = 0.\) After these developments, we get
   \begin{align*}
      \mathcal{A}_s &= \bar{u}_4 \left[\xi \slashed{\epsilon}_2 + \xi \zeta (\slashed{k}_1 + \slashed{k}_2 - m) + \zeta \slashed{\epsilon}^*_3\right]u_1\\
                    &= \bar{u}_4 \left[\xi \slashed{\epsilon}_2 + \xi \zeta \slashed{k}_2 + \zeta \slashed{\epsilon}^*_3\right]u_1,
   \end{align*}
   which is not zero in general.

   We now repeat the procedure for the \(t\) channel
   \begin{equation*}
      \mathcal{A}_t = \bar{u}_4\left[\xi\slashed{\epsilon}_2 \frac{\slashed{k}_1 - \slashed{k}_3 + m}{t - m^2} \slashed{k}_3 + \xi \zeta \slashed{k}_2 \frac{\slashed{k}_1 - \slashed{k}_3 + m}{t - m^2} \slashed{k}_3 + \zeta \slashed{k}_2 \frac{\slashed{k}_1 - \slashed{k}_3 + m}{t - m^2} \slashed{\epsilon}^*_3 \right]u_1
   \end{equation*}
   with the following identities
   \begin{equation*}
       \bar{u}_4 \slashed{k}_2 = \bar{u}_4 (\slashed{k}_3 - \slashed{k}_1 + \slashed{k}_4) = - \bar{u}_4(\slashed{k}_1 - \slashed{k}_3 - m),
   \end{equation*}
   \begin{equation*}
       \slashed{k}_3 u_1 = \slashed{k}_3u_1 - (\slashed{k}_1 - m)u_1 = -(\slashed{k}_1 - \slashed{k}_3 - m)u_1,
   \end{equation*}
   then
   \begin{align*}
      \mathcal{A}_t &= \bar{u}_4\left[\xi\slashed{\epsilon}_2 \frac{\slashed{k}_1 - \slashed{k}_3 + m}{t - m^2} \slashed{k}_3 + \xi \zeta \slashed{k}_2 \frac{\slashed{k}_1 - \slashed{k}_3 + m}{t - m^2} \slashed{k}_3 + \zeta \slashed{k}_2 \frac{\slashed{k}_1 - \slashed{k}_3 + m}{t - m^2} \slashed{\epsilon}^*_3 \right]u_1\\
                    &= \bar{u}_4\left[-\xi \slashed{\epsilon}_2 + \xi \zeta (\slashed{k}_1 - \slashed{k}_3 - m) - \zeta \slashed{\epsilon}_3^*\right]u_1\\
                    &= -\bar{u}_4\left[\xi \slashed{\epsilon}_2 + \xi \zeta \slashed{k}_3 + \zeta \slashed{\epsilon}_3^*\right]u_1,
   \end{align*}
   which is also non-zero.

   The difference for the process amplitude is
   \begin{align*}
      i \tilde{\mathcal{M}} - i \mathcal{M} &= i (ie)^2 (\mathcal{A}_s + \mathcal{A}_t)\\
                                            &= i(ie)^2 \bar{u}_4 \left[\left(\xi \slashed{\epsilon}_2 + \xi \zeta \slashed{k}_2 + \zeta \epsilon_3^*\right) - \left(\xi \slashed{\epsilon}_2 + \xi \zeta \slashed{k}_3 + \zeta \slashed{\epsilon}_3^*\right)\right]\\
                                            &= i(i e)^2 \xi \zeta \bar{u}_4\left(\slashed{k}_2 - \slashed{k}_3\right) u_1\\
                                            &= i(ie)^2 \xi \zeta \bar{u}_4 \left(\slashed{k}_4 - \slashed{k}_1\right) u_1\\
                                            &= i(ie)^2 \xi \zeta (m - \slashed{k}_1) u_1\\
                                            &= 0,
   \end{align*}
   hence gauge invariant.
\end{proof}
