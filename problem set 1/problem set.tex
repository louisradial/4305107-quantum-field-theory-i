\documentclass[english]{artigo}

\author{Louis Bergamo Radial\\8992822}
\title{Quantum Field Theory I\\Problem set I}
\begin{document}
   \maketitle
   \begin{theorem}{Noether's theorem}{noether}
      Let \(x^\mu \mapsto \tilde{x}^\mu = x^\mu + \Delta x^\mu\) and \(\phi_j(x) \mapsto \tilde{\phi}_j(\tilde{x}) = \phi_j(x) + \Delta\phi_j(x)\) be an infinitesimal transformation that leaves the action \(S = \int \dln4x \mathcal{L}\) invariant. Then the current \(j^\mu = \diffp{\mathcal{L}}{(\partial_\mu \phi_j)} \delta \phi_j + \mathcal{L} \Delta x^\mu\) satisfies the continuity equation \(\partial_\mu j^\mu = 0\), where \(\delta \phi_j(x) = \tilde{\phi}_j(x) - \phi_j(x)\).
   \end{theorem}
   \begin{proof}
      Notice we have
      \begin{align*}
         \Delta\phi_j(x) = \tilde{\phi_j}(\tilde{x}) - \phi_j(x) &= \tilde{\phi_j}(x) + \Delta x^\mu \partial_\mu \tilde{\phi_j}(x) - \phi_j(x) = \Delta x^\mu \partial_\mu \tilde{\phi_j}(x) + \delta \phi_j(x)\\
                                                                 &= \Delta x^\mu \partial_\mu \left[\phi_j(x) + \Delta \phi_j(x)\right] + \delta\phi_j(x)\\
                                                                 &= \Delta x^\mu \partial_\mu \phi_j(x) + \delta\phi_j(x)
      \end{align*}
      and
      \begin{equation*}
         \tilde{\partial}_\mu = \diffp{x^\nu}{\tilde{x}^\mu} \partial_\nu = \left(\delta\indices{^\nu_\mu} - \partial_\mu \Delta x^\nu\right)\partial_\nu
      \end{equation*}
      in first order of \(\Delta x^\mu\) and \(\Delta \phi_j(x)\). This yields
      \begin{align*}
         \Delta (\partial_\mu \phi_j)(x) = \tilde{\partial}_\mu \tilde{\phi_j}(\tilde{x}) - \partial_\mu \phi_j(x) 
         &= \left(\delta\indices{^\nu_\mu} - \partial_\mu \Delta x^\nu\right) \partial_\nu \left[\phi_j(x) + \Delta \phi_j(x)\right] - \partial_\mu \phi_j(x)\\
         &= \partial_\mu \Delta\phi_j(x) - (\partial_\mu \Delta x^\nu) \partial_\nu \phi_j(x)\\
         &= \partial_\mu \left[\delta \phi_j(x)\right] + \partial_\mu \left[\Delta x^\nu \partial_\nu \phi_j(x)\right] - (\partial_\mu \Delta x^\nu) \partial_\nu \phi_j(x)\\
         &= \partial_\mu \left[\delta \phi_j(x)\right] + \Delta x^\nu \partial_\mu \partial_\nu \phi_j(x)\\
         &= \partial_\mu \left[\delta \phi_j(x)\right] + \Delta x^\nu \partial_\nu \partial_\mu \phi_j(x)
      \end{align*}
      in the same first order approximation. Let us denote 
      \begin{equation*}
         \tilde{\mathcal{L}} = \mathcal{L}[\tilde{\phi}(\tilde{x}), \tilde{\partial}\tilde{\phi}(\tilde{x}), \tilde{x}]\quad\text{and}\quad
         \mathcal{L} = \mathcal{L}[\phi(x), \partial \phi(x), x], 
      \end{equation*}
      then
      \begin{align*}
         \tilde{\mathcal{L}} &= \mathcal{L}\left[\phi(x) + \Delta \phi(x), \partial \phi(x) + \Delta \partial\phi(x), x + \Delta x\right]\\
                             &= \mathcal{L} + \diffp{\mathcal{L}}{\phi_j} \Delta \phi_j(x) + \diffp{\mathcal{L}}{(\partial_\mu \phi_j)} \Delta \partial_\mu \phi_j(x) + \diffp{\mathcal{L}}{x^\mu} \Delta x^\mu\\
                             &= \mathcal{L} + \diffp{\mathcal{L}}{\phi_j} \left[\Delta x^\mu \partial_\mu \phi_j(x) + \delta\phi_j(x)\right] + 
                             \diffp{\mathcal{L}}{(\partial_\nu \phi_j)} \left\{\partial_\nu \left[\delta \phi_j(x)\right] + \Delta x^\mu \partial_\mu \partial_\nu \phi_j(x)\right\}
                             + \diffp{L}{x^\mu} \Delta x^\mu\\
                             &=  \mathcal{L} + \diff{\mathcal{L}}{x^\mu} \Delta x^\mu + \diffp{L}{\phi_j} \delta \phi_j(x) + \diffp{\mathcal{L}}{(\partial_\nu \phi_j)} \partial_\nu\left[\delta \phi_j(x)\right]\\
                             &=  \mathcal{L} + \diff{\mathcal{L}}{x^\mu} \Delta x^\mu + \delta \mathcal{L},
      \end{align*}
      where we defined \(\delta \mathcal{L} = \diffp{\mathcal{L}}{\phi_j} \delta \phi_j(x) + \diffp{\mathcal{L}}{(\partial_\nu \phi_j)} \partial_\nu\left[\delta \phi_j(x)\right]\). Notice if \(\phi_j\) satisfies the equations of motion,
      \begin{equation*}
         \diffp{\mathcal{L}}{\phi_j} = \partial_\sigma \diffp{\mathcal{L}}{(\partial_\sigma \phi_j)},
      \end{equation*}
      we have
      \begin{equation*}
         \delta \mathcal{L} = \partial_\nu \left(\diffp{\mathcal{L}}{(\partial_\nu \phi_j)}\right) \delta \phi_j(x) + \diffp{\mathcal{L}}{(\partial_\nu \phi_j)} \partial_\nu \left[\delta \phi_j(x)\right] = \partial_\nu\left[\diffp{\mathcal{L}}{(\partial_\nu \phi_j)} \delta\phi_j(x)\right] = \diff*{\left[\diffp{\mathcal{L}}{(\partial_\nu \phi_j)} \delta\phi_j(x)\right]}{x^\nu}.
      \end{equation*}

      Recall that if a matrix \(A\) has (complex) eigenvalues \(\sigma(A),\) then the matrix \(\unity + A\) has eigenvalues \(\sigma(\unity + A) = \setc{1 + a}{a \in \sigma(A)}\). Therefore, if \(\varepsilon\) is an infinitesimal parameter, we have
      \begin{equation*}
         \det(\unity + \varepsilon A) = \prod_{\lambda \in \sigma(\unity + \varepsilon A)} \lambda = \prod_{a \in \sigma(A)} (1 + \varepsilon a) = 1 + \varepsilon \sum_{a \in \sigma(A)} a = 1 + \varepsilon \Tr(A)
      \end{equation*}
      up to linear order in \(\varepsilon\). As a result, the Jacobian of the infinitesimal transformation is such that
      \begin{equation*}
         \dln4{\tilde{x}} = \left(1 + \partial_\mu\Delta x^\mu\right) \dln4x,
      \end{equation*}
      thus the difference in action is given by
      \begin{align*}
         \Delta S &= \int \dln4{\tilde{x}} \tilde{\mathcal{L}} - \int \dln4x \mathcal{L}\\
                  &= \int \dln4x \left\{\left(1 + \partial_\sigma \Delta x^\sigma\right)\left[\mathcal{L} + \diff{\mathcal{L}}{x^\mu}\Delta x^\mu + \delta \mathcal{L}\right] - \mathcal{L}\right\}\\
                  &= \int \dln4x \left(\mathcal{L} \partial_\sigma \Delta x^\sigma + \diff{\mathcal{L}}{x^\mu}\Delta x^\mu + \delta \mathcal{L}\right)\\
                  &= \int \dln4x \left[\diff*{\left(\mathcal{L} \Delta x^\sigma\right)}{x^\sigma} + \delta \mathcal{L}\right].
      \end{align*}
      Then, for a solution of the equations of motion, this yields
      \begin{equation*}
         \Delta S = \int \dln4x \diff*{\left[\diffp{\mathcal{L}}{(\partial_\sigma \phi_j)} \delta \phi_j(x) + \mathcal{L} \Delta x^\sigma\right]}{x^\sigma},
      \end{equation*}
      therefore the integrand must be equal to zero if the infinitesimal transformation is a symmetry, as desired.
   \end{proof}

   % vim: spl=en
\begin{lemma}{Eigenvectors of a real linear combination of Pauli matrices}{pauli}
   Let \(\vetor{n} = n^i \vetor{e}_i \in \mathbb{R}^3\) be a unit vector. Then the eigenvalues of \(\vetor{n}\cdot \vetor{\sigma}\) are \(\set{-1,1}\) and its eigenvectors are
   \begin{equation*}
      \psi_{s} = \begin{pmatrix}
          s( n^1 - i n^2)\\
          1 - s n^3
      \end{pmatrix},
   \end{equation*}
   where \(s = \pm1.\)
\end{lemma}
\begin{proof}
   Since \(\vetor{n} \cdot \vetor{\sigma} = \unity,\) it is clear its eigenvalues are either \(1\) or \(-1\) as we have
   \begin{equation*}
      \vetor{n} \cdot \vetor{\sigma} \psi_s = s \psi_s \implies \psi_s = s^2 \psi_s.
   \end{equation*}
   For \(s \in \set{-1,1},\) we have
   \begin{align*}
      \vetor{n} \cdot \vetor{\sigma} \psi_s &= \begin{pmatrix}
         n^3 && n^1 - i n^2\\
         n^1 + i n^2 && -n^3
      \end{pmatrix}
      \begin{pmatrix}
          s( n^1 - i n^2)\\
          1 - s n^3
      \end{pmatrix}\\
      &= 
      \begin{pmatrix}
         sn^3 (n^1 - in^2) + (n^1 - i n^2) (1 - sn^3)\\
         s(n^1 - in^2)(n^1 + in^2) - n^3(1 - sn^3)
      \end{pmatrix}\\
      &= \begin{pmatrix}
          n^1 - in^2\\
          s\vetor{n}\cdot\vetor{n} - n^3
      \end{pmatrix}\\
      &= s \begin{pmatrix}
          s(n^1 - in^2)\\
          1 - sn^3
      \end{pmatrix}\\
      &= s \psi_s,
   \end{align*}
   that is, \(\psi_s\) are the eigenvectors as claimed.
\end{proof}
\begin{problem}{Simultaneous eigenstates of the helicity operator and free Dirac Hamiltonian}{p1}
   The helicity operator is defined as
   \begin{equation*}
      h = \frac{\vetor{\Sigma}\cdot \vetor{p}}{\norm{\vetor{p}}}.
   \end{equation*}
   \begin{enumerate}[label=(\alph*)]
       \item Show that his operator can be diagonalized simultaneously with the free Dirac Hamiltonian (\(H_D\)).
       \item Obtain the positive and negative energy eigenstates of \(h\) and \(H_D\).
   \end{enumerate}
\end{problem}
\begin{proof}[Solution]
   We consider the Lagrangian density \(\mathcal{L} = \bar{\Psi} (i \slashed{\partial} - m) \Psi\) and its invariance under the infinitesimal Lorentz transformation 
   \begin{equation*}
      x^\mu \mapsto x'^\mu = x^\mu + \omega\indices{^\mu_\nu}x^\nu\quad\text{and}\quad\Psi(x) \mapsto \Psi'(x') = \left(\unity - \frac{i}{4} \omega_{\mu\nu} \sigma^{\mu\nu}\right)\Psi(x).
   \end{equation*}
   As
   \begin{align*}
      \delta \Psi(x) = \Psi'(x) - \Psi(x) &= \left(\unity - \frac{i}{4} \omega_{\mu\nu} \sigma^{\mu\nu}\right)\Psi(x - \omega x) - \Psi(x)\\
                                          &= -\frac{i}{4} \omega_{\mu\nu} \sigma^{\mu\nu}\Psi(x) - \omega\indices{^\mu_\nu} x^\nu\partial_\mu \Psi(x)\\
                                          &= -i \omega_{\mu\nu} \left(\frac14 \omega^{\mu\nu} + \frac1i x^\nu \partial^\mu\right)\Psi(x)
   \end{align*}
   and as the conserved current is \(J^\rho = i \bar{\Psi} \gamma^\rho \delta \Psi,\) the conserved charge for a rotation about the \(\vetor{n}\) axis by the infinitesimal angle \(\vartheta\), where \(\omega_{jk} = \vartheta n^i \epsilon_{ijk},\) is
   \begin{equation*}
      \int_{\mathbb{R}^3} \dln3x J^0 = \int_{\mathbb{R}^3} \dln3x \herm{\Psi} \vartheta n^i \epsilon_{ijk} \left(\frac14 \sigma^{jk} + \frac{1}{i} x^{k} \partial^j\right) \Psi(x) = \int_{\mathbb{R}^3} \dln3x \herm{\Psi} \left(\frac{\vartheta}2 n^i \Sigma_i + \vartheta n^i L_i\right)\Psi(x),
   \end{equation*}
   where \(L_i\) is the \(i\) component of the orbital angular momentum and 
   \begin{equation*}
      \Sigma_i = \frac12 \epsilon_{ijk} \sigma^{jk} = \frac{i}{4} \epsilon_{ijk} \commutator{\gamma^j}{\gamma^k} = \frac{i}{4} \epsilon_{ijk} \left(\anticommutator{\gamma^j}{\gamma^k} - 2\gamma^{k} \gamma^{j}\right) = -\frac{i}{2} \epsilon_{ijk} \gamma^{k} \gamma^{j} = \frac{i}{2} \epsilon_{ijk} \gamma^{j} \gamma^{k}
   \end{equation*}
   is the \(i\) component of the spin angular momentum, up to a factor of two. In the Weyl representation, where
   \begin{equation*}
      \gamma^0 = \begin{pmatrix}
           && \unity\\
         \unity && 
      \end{pmatrix}
      \quad\text{and}\quad
      \gamma^j = \begin{pmatrix}
           && \sigma^j\\
         -\sigma^j &&
      \end{pmatrix},
   \end{equation*}
   we have
   \begin{equation*}
      \Sigma_i = \frac12 \epsilon_{ijk} \gamma^j \gamma^k = -\frac12 \epsilon_{ijk} (\delta_{jk} + i \epsilon_{jk\ell} \sigma^{\ell}) \unity = -\delta_{i \ell} \sigma_\ell \unity \implies \Sigma^i = \begin{pmatrix}
         \sigma^i &&\\
                  && \sigma^i
      \end{pmatrix}.
   \end{equation*}

   From the Dirac equation we have
   \begin{equation*}
      i\slashed{\partial}\Psi = m \Psi \implies i \gamma^0\partial_0 \Psi = (m - i \gamma^j \partial_j)\Psi \implies i \partial_0 \Psi = \gamma^0 (m - i \gamma^j \partial_j) \Psi,
   \end{equation*}
   that is, 
   \begin{equation*}
      H_D = \gamma^0 (m + \gamma^j p_j)
   \end{equation*}
   is the free Dirac Hamiltonian. In the Weyl representation, we have
   \begin{equation*}
      H_D = \begin{pmatrix}
         -\sigma^j p_j && m\\
         m && \sigma^j p_j
         \end{pmatrix} = \begin{pmatrix}
         \vetor{\sigma} \cdot \vetor{p} && m\\
         m && -\vetor{\sigma} \cdot \vetor{p}
      \end{pmatrix},
   \end{equation*}
   where \(\vetor{\sigma} \cdot \vetor{p} = \delta_{ij} \sigma^i p^j.\) Notice the helicity, \(h = \frac{\vetor{\Sigma} \cdot \vetor{p}}{\norm{\vetor{p}}},\) satisfies
   \begin{align*}
      \commutator{h}{H_D} 
      &= \commutator*{\begin{pmatrix}
            \vetor{\sigma} \cdot \frac{\vetor{p}}{\norm{\vetor{p}}} &&\\
                                                                    && \vetor{\sigma} \cdot \frac{\vetor{p}}{\norm{\vetor{p}}}
      \end{pmatrix}}
      {\begin{pmatrix}
            \vetor{\sigma} \cdot \vetor{p} && m\\
            m && -\vetor{\sigma} \cdot \vetor{p}
      \end{pmatrix}}\\
      &=
      \begin{pmatrix}
         \left(\vetor{\sigma}\cdot\frac{\vetor{p}}{\norm{\vetor{p}}}\right)(\vetor{\sigma}\cdot \vetor{p}) && \left(\vetor{\sigma}\cdot\frac{\vetor{p}}{\norm{\vetor{p}}}\right)m\\
         \left(\vetor{\sigma}\cdot\frac{\vetor{p}}{\norm{\vetor{p}}}\right)m && -\left(\vetor{\sigma}\cdot\frac{\vetor{p}}{\norm{\vetor{p}}}\right)(\vetor{\sigma} \cdot \vetor{p})
      \end{pmatrix}-
      \begin{pmatrix}
         (\vetor{\sigma}\cdot \vetor{p})\left(\vetor{\sigma}\cdot\frac{\vetor{p}}{\norm{\vetor{p}}}\right) && m \left(\vetor{\sigma}\cdot\frac{\vetor{p}}{\norm{\vetor{p}}}\right)\\
         m \left(\vetor{\sigma}\cdot\frac{\vetor{p}}{\norm{\vetor{p}}}\right) && - (\vetor{\sigma} \cdot \vetor{p})\left(\vetor{\sigma}\cdot\frac{\vetor{p}}{\norm{\vetor{p}}}\right)
      \end{pmatrix}\\
      &= 0,
   \end{align*}
   then \(h\) and \(H_D\) are compatible, that is, there exists a basis with which both operators can be diagonalized simultaneously.

   Let \(\Psi_s\) be an element of such basis, that is, \(h \Psi_s = s \Psi_s\) and \(H_D \Psi_s = E \Psi_s.\) Notice \(h^2 = \unity,\) then \(s \in \set{-1,1}.\) If we write \(\vetor{p} = \norm{\vetor{p}} \left(\cos\phi \sin\theta \vetor{e}_1 + \sin\phi\cos\theta \vetor{e}_2 + \cos\theta \vetor{e}_3\right),\) we learn from \cref{lem:pauli} that
   \begin{equation*}
      \Psi_s = \begin{pmatrix}
         \lambda_s \psi_s\\
         \mu_s \psi_s
      \end{pmatrix},
      \quad\text{where}\quad
      \psi_s = \begin{pmatrix}
         se^{-i\phi} \sin\theta\\
         1 - s \cos\theta
      \end{pmatrix},
   \end{equation*}
   with \(\lambda_s, \mu_s\) depending on the energy eigenvalue. Using this on the energy eigenvalue problem yields
   \begin{equation*}
      \begin{cases}
         \lambda_s s \norm{\vetor{p}} \psi_s + m \mu_s \psi_s = \lambda_s E \psi_s\\
         m \lambda_s  \psi_s  - \mu_s s  \norm{\vetor{p}} \psi_s =  \mu_s E \psi_s
      \end{cases}
      \implies
      \begin{cases}
         \lambda_s (E - s \norm{\vetor{p}}) = m \mu_s\\
         \mu_s (E + s \norm{\vetor{p}}) = m \lambda_s
      \end{cases}
      % \implies
      % \begin{cases}
      %    \lambda_s (E - s \sqrt{E^2 - m^2}) = m \mu_s\\
      %    \mu_s (E + s \sqrt{E^2 - m^2}) = m \lambda_s
      % \end{cases}
   \end{equation*}
   as \(\psi_s\) is not a null spinor. For a massive particle, we have \(E \pm s \norm{\vetor{p}} \neq 0,\) then
   \begin{equation*}
      \Psi_s = \mu_s
      \begin{pmatrix}
         \frac{m}{E - s\norm{\vetor{p}}} \psi_s\\
         \psi_s
      \end{pmatrix}.
   \end{equation*}
   Imposing the normalization condition \(\herm{\Psi_s}\Psi_s = 2 E\) yields
   \begin{align*}
      \abs{\mu_s}^2 \left[\left(\frac{m}{E - s \norm{\vetor{p}}}\right)^2 + 1\right]\herm{\psi_s}\psi_s = 2E
      &\implies \abs{\mu_s}^2 \frac{2E^2 - 2s E\norm{\vetor{p}}}{(E - s \norm{\vetor{p}})^2}(1 - s\cos\theta) = E\\
      &\implies (1 - s\cos\theta)\abs{\mu_s}^2 = \frac12 \abs*{E - s \norm{\vetor{p}}},
   \end{align*}
   and we choose the phase of \(\mu_s\) such that
   \begin{equation*}
      \mu_s = \sqrt{\frac{E - s \norm{\vetor{p}}}{2 (1 - s \cos\theta)}},
   \end{equation*}
   hence
   \begin{equation*}
      \Psi_s = \sqrt{E - s \norm{\vetor{p}}} \begin{pmatrix}
         \frac{sm}{E - s \norm{\vetor{p}}} e^{-i\phi} \frac{\sin\frac{\theta}{2} \cos\frac\theta2}{\sqrt{\frac{1 - s \cos\theta}2}}\\
         \frac{m}{E - s \norm{\vetor{p}}} \sqrt{\frac{1 - s \cos\theta}{2}}\\
         s e^{-i\phi} \frac{\sin\frac{\theta}{2} \cos\frac\theta2}{\sqrt{\frac{1 - s \cos\theta}2}}\\
         \sqrt{\frac{1 - s \cos\theta}{2}}
      \end{pmatrix},
   \end{equation*}
   that is,
   \begin{equation*}
      \Psi_\uparrow =  \begin{pmatrix}
         \frac{m}{\sqrt{E - \norm{\vetor{p}}}} e^{-i\phi} \cos\frac\theta2\\
         \frac{m}{\sqrt{E - \norm{\vetor{p}}}} \sin\frac\theta2\\
         \sqrt{E - \norm{\vetor{p}}}e^{-i\phi} \cos\frac\theta2\\
         \sqrt{E - \norm{\vetor{p}}}\sin\frac\theta2
      \end{pmatrix}
      =
      \begin{pmatrix}
         \sqrt{E + \norm{\vetor{p}}} e^{-i\phi} \cos\frac\theta2\\
         \sqrt{E + \norm{\vetor{p}}} \sin\frac\theta2\\
         \sqrt{E - \norm{\vetor{p}}}e^{-i\phi} \cos\frac\theta2\\
         \sqrt{E - \norm{\vetor{p}}}\sin\frac\theta2
      \end{pmatrix}
   \end{equation*}
   and
   \begin{equation*}
      \Psi_\downarrow =  \begin{pmatrix}
         -\frac{m}{\sqrt{E + \norm{\vetor{p}}}} e^{-i\phi} \sin\frac\theta2\\
         \frac{m}{\sqrt{E + \norm{\vetor{p}}}} \cos\frac\theta2\\
         -\sqrt{E + \norm{\vetor{p}}}e^{-i\phi} \sin\frac\theta2\\
         \sqrt{E + \norm{\vetor{p}}}\cos\frac\theta2
      \end{pmatrix}
      =
      \begin{pmatrix}
         -\sqrt{E - \norm{\vetor{p}}} e^{-i\phi} \sin\frac\theta2\\
         \sqrt{E - \norm{\vetor{p}}} \cos\frac\theta2\\
         -\sqrt{E + \norm{\vetor{p}}}e^{-i\phi} \sin\frac\theta2\\
         \sqrt{E + \norm{\vetor{p}}}\cos\frac\theta2
      \end{pmatrix},
   \end{equation*}
   where we have used that \(m = \sqrt{E^2 - \norm{\vetor{p}}^2} = \sqrt{E - \norm{\vetor{p}}} \sqrt{E + \norm{\vetor{p}}}.\) Finally, the positive energy solutions, \(H_D u_s = E u_s\) with \(E > 0,\) are
   \begin{equation*}
      u_\uparrow = \begin{pmatrix}
         \sqrt{E + \norm{\vetor{p}}} e^{-i\phi} \cos\frac\theta2\\
         \sqrt{E + \norm{\vetor{p}}} \sin\frac\theta2\\
         \sqrt{E - \norm{\vetor{p}}}e^{-i\phi} \cos\frac\theta2\\
         \sqrt{E - \norm{\vetor{p}}}\sin\frac\theta2
      \end{pmatrix},
      \quad\text{and}\quad
      u_\downarrow =  \begin{pmatrix}
         -\sqrt{E - \norm{\vetor{p}}} e^{-i\phi} \sin\frac\theta2\\
         \sqrt{E - \norm{\vetor{p}}} \cos\frac\theta2\\
         -\sqrt{E + \norm{\vetor{p}}}e^{-i\phi} \sin\frac\theta2\\
         \sqrt{E + \norm{\vetor{p}}}\cos\frac\theta2
      \end{pmatrix},
   \end{equation*}
   and the negative energy solutions, \(H_D v_s = - E v_s\) with \(E > 0\), are
   \begin{equation*}
      v_\uparrow = \begin{pmatrix}
         i\sqrt{E - \norm{\vetor{p}}} e^{-i\phi} \cos\frac\theta2\\
         i\sqrt{E - \norm{\vetor{p}}} \sin\frac\theta2\\
         i\sqrt{E + \norm{\vetor{p}}}e^{-i\phi} \cos\frac\theta2\\
         i\sqrt{E + \norm{\vetor{p}}}\sin\frac\theta2
      \end{pmatrix},
      \quad\text{and}\quad
      v_\downarrow =  \begin{pmatrix}
         -i\sqrt{E + \norm{\vetor{p}}} e^{-i\phi} \sin\frac\theta2\\
         i\sqrt{E + \norm{\vetor{p}}} \cos\frac\theta2\\
         -i\sqrt{E - \norm{\vetor{p}}}e^{-i\phi} \sin\frac\theta2\\
         i\sqrt{E - \norm{\vetor{p}}}\cos\frac\theta2
      \end{pmatrix}.
   \end{equation*}
   Notice we may take the \(m \to 0\) limit without problems yielding
   \begin{equation*}
      u_\uparrow = \begin{pmatrix}
         \sqrt{2E} e^{-i\phi} \cos\frac\theta2\\
         \sqrt{2E} \sin\frac\theta2\\
         0\\
         0
      \end{pmatrix},
      \quad\text{and}\quad
      u_\downarrow =  \begin{pmatrix}
         0\\
         0\\
         -\sqrt{2E}e^{-i\phi} \sin\frac\theta2\\
         \sqrt{2E}\cos\frac\theta2
      \end{pmatrix},
   \end{equation*}
   \begin{equation*}
      v_\uparrow = \begin{pmatrix}
         0\\
         0\\
         i\sqrt{2E}e^{-i\phi} \cos\frac\theta2\\
         i\sqrt{2E}\sin\frac\theta2
      \end{pmatrix},
      \quad\text{and}\quad
      v_\downarrow =  \begin{pmatrix}
         -i\sqrt{2E} e^{-i\phi} \sin\frac\theta2\\
         i\sqrt{2E} \cos\frac\theta2\\
         0\\
         0\\
      \end{pmatrix},
   \end{equation*}
   which are the massless solutions.
\end{proof}
\begin{proof}[Alternative solution for the eigenstates up to a rotation]
   Let \(\Psi\) be an element of such basis, where we have \(h \Psi = s \Psi\) and \(H_D \Psi = E \Psi.\) In the Weyl representation, let us write \(\Psi = \left(\begin{smallmatrix}
         \Psi_1\\
         \Psi_2
   \end{smallmatrix}\right),\) then,
   \begin{equation*}
      \begin{cases}
         \vetor{\sigma}\cdot \vetor{p} \Psi_i = s \norm{\vetor{p}} \Psi_i\\
         \vetor{\sigma} \cdot \vetor{p} \Psi_1 + m \Psi_2 = E \Psi_1\\
         m \Psi_1 - \vetor{\sigma} \cdot \vetor{p} \Psi_2 = E \Psi_2
      \end{cases}
      \implies
      \begin{cases}
         \vetor{\sigma}\cdot \vetor{p} \Psi_i = s \norm{\vetor{p}} \Psi_i\\
         s \norm{\vetor{p}} \Psi_1 + m \Psi_2 = E \Psi_1\\
         m \Psi_1 - s \norm{\vetor{p}} \Psi_2 = E \Psi_2.
      \end{cases}
   \end{equation*}
   We may use a rotated frame for which \(\frac{\vetor{p}}{\norm{\vetor{p}}} = \vetor{e}_z,\) then the eigenstate equation for the helicity reads as \(\sigma_z \Psi_i = s \Psi_i,\) that is, \(s = \pm 1\) and we see \(\Psi_i\) is an eigenvector of \(\sigma_z\) in this frame. We'll first consider massive particles, and after a boost to its rest frame, we have \(E = \pm m\), hence
   \begin{equation*}
      \begin{cases}
         m \Psi_2 = \pm m \Psi_1\\
         m \Psi_1 = \pm m \Psi_2
      \end{cases} \implies \Psi_1 = \pm \Psi_2.
   \end{equation*}
   We have thus the following set of linearly independent solutions
   \begin{equation*}
      \Psi^+_\uparrow = \sqrt{m} \begin{pmatrix}
         1 \\ 0\\1 \\ 0
      \end{pmatrix},
      \quad
      \Psi^+_\downarrow = \sqrt{m} \begin{pmatrix}
         0 \\ 1\\0 \\ 1
      \end{pmatrix},
      \quad
      \Psi^-_\uparrow = \sqrt{m} \begin{pmatrix}
         1 \\ 0\\-1 \\ 0
      \end{pmatrix},
      \quad\text{and}\quad
      \Psi^-_\downarrow = \sqrt{m} \begin{pmatrix}
         0 \\ 1\\0 \\-1
      \end{pmatrix},
   \end{equation*}
   where the sign reflects that of the energy and the arrow denotes the helicity. We'll write
   \begin{equation*}
      \Psi^\pm_{s} = \sqrt{m}\begin{pmatrix} \xi_s\\\pm\xi_s \end{pmatrix},\quad\text{where}\quad \xi_s \in \set*{\begin{pmatrix} 1 \\0 \end{pmatrix}, \begin{pmatrix} 0\\1 \end{pmatrix}},
   \end{equation*}
   and boost each rest frame solution back to the frame where \(p^\mu = (E, \norm{\vetor{p}} \vetor{e}_z)\). The rapidity \(\phi\) for the boost is such that \(\norm{\vetor{p}} = \pm m \sinh\phi\) and \(E = m \cosh\phi\) then the spinor transformation is
   \begin{align*}
      \exp\left(\frac14 \phi \sigma^{03}\right) 
      &= \exp\left[\frac12 \phi \begin{pmatrix}
            - \sigma_z && \\
                       && \sigma_z
      \end{pmatrix}\right]\\
      &= \left[\cosh\left(\frac12 \phi\right) \unity + \sinh\left(\frac12 \phi\right)\begin{pmatrix}
            - \sigma_z && \\
                       && \sigma_z
      \end{pmatrix}\right]\\
      &= \begin{pmatrix}
         e^{\frac{\phi}2} \left(\frac{\unity - \sigma_z}{2}\right) + e^{-\frac{\phi}2} \left(\frac{\unity + \sigma_z}{2}\right) &&\\ && e^{\frac{\phi}2} \left(\frac{\unity + \sigma_z}2\right) + e^{-\frac{\phi}2} \left(\frac{\unity - \sigma_z}{2}\right)
      \end{pmatrix}\\
      &= \frac{1}{\sqrt{m}}\begin{pmatrix}
         \sqrt{E \pm \norm{\vetor{p}}} \left(\frac{\unity - \sigma_z}{2}\right) + \sqrt{E \mp \norm{\vetor{p}}} \left(\frac{\unity + \sigma_z}2\right)&&\\ &&\sqrt{E \pm \norm{\vetor{p}}} \left(\frac{\unity + \sigma_z}{2}\right) + \sqrt{E \mp \norm{\vetor{p}}} \left(\frac{\unity - \sigma_z}2\right)
      \end{pmatrix}.
   \end{align*}
   The boosted solutions are
   \begin{equation*}
      \Psi_s^{\pm} = \begin{pmatrix}
         \left[\sqrt{E \pm \norm{\vetor{p}}} \left(\frac{1 - s}{2}\right) + 
         \sqrt{E \mp \norm{\vetor{p}}} \left(\frac{1 + s}{2}\right)\right] \xi_s\\
         \pm\left[\sqrt{E \pm \norm{\vetor{p}}} \left(\frac{1 + s}{2}\right) + 
         \sqrt{E \mp \norm{\vetor{p}}} \left(\frac{1 - s}{2}\right)\right]\xi_s
      \end{pmatrix},
   \end{equation*}
   or
   \begin{equation*}
      \Psi^+_\uparrow = \begin{pmatrix}
         \sqrt{E - \norm{\vetor{p}}}\\
         0\\
         \sqrt{E + \norm{\vetor{p}}}\\
         0
      \end{pmatrix},\quad
      \Psi^+_\downarrow = \begin{pmatrix}
         0\\
         \sqrt{E + \norm{\vetor{p}}}\\
         0\\
         \sqrt{E - \norm{\vetor{p}}}
      \end{pmatrix},\quad
      \Psi^-_\uparrow = \begin{pmatrix}
         \sqrt{E + \norm{\vetor{p}}}\\
         0\\
         -\sqrt{E - \norm{\vetor{p}}}\\
         0
      \end{pmatrix},\quad\text{and}\quad
      \Psi^-_\downarrow = \begin{pmatrix}
         0\\
         \sqrt{E - \norm{\vetor{p}}}\\
         0\\
         -\sqrt{E + \norm{\vetor{p}}}
      \end{pmatrix}.
   \end{equation*}
   We now consider the Hamiltonian for a massless particle and the solutions above. For a massless particle we have \(E = \pm \norm{\vetor{p}},\) which means
   \begin{equation*}
      \Psi^+_\uparrow = \begin{pmatrix}
         0\\
         0\\
         \sqrt{2E}\\
         0
      \end{pmatrix},\quad
      \Psi^+_\downarrow = \begin{pmatrix}
         0\\
         \sqrt{2E}\\
         0\\
         0
      \end{pmatrix},\quad
      \Psi^-_\uparrow = \begin{pmatrix}
         0\\
         0\\
         -\sqrt{2E}\\
         0
      \end{pmatrix},\quad\text{and}\quad
      \Psi^-_\downarrow = \begin{pmatrix}
         0\\
         \sqrt{2E}\\
         0\\
         0
      \end{pmatrix},
   \end{equation*}
   are the independent energy eigenstates.
\end{proof}

   % We'll show the Hamiltonian and the helicity, \(h = \frac{\vetor{\Sigma} \cdot \vetor{p}}{\norm{\vetor{p}}} = g^{ij}\frac{\Sigma_i p_j}{\norm{\vetor{p}}},\) are compatible operators
   % \begin{align*}
   %    \commutator*{h}{H_D} %&= \commutator*{\frac{\vetor{\Sigma}\cdot\vetor{p}}{\norm{\vetor{p}}}}{\gamma^0 (m - \vetor{\gamma}\cdot \vetor{p})}\\
   %                        &= g^{ij}\commutator*{\frac{\Sigma_i p_j}{\norm{\vetor{p}}}}{\gamma^0\left(m + \gamma^k p_k\right)}\\
   %                        &= \frac{i}{2} g^{ij} \epsilon_{iab} \frac{p_j}{\norm{\vetor{p}}}\commutator*{\gamma^a \gamma^b}{\gamma^0\left(m + \gamma^kp_k\right)}\\
   %                        &= \frac{i}{2} g^{ij} \epsilon_{iab} \frac{p_j}{\norm{\vetor{p}}}\left(m\commutator*{\gamma^a \gamma^b}{\gamma^0} + \commutator*{\gamma^a \gamma^b}{\gamma^0 \gamma^k}p_k\right)\\
   %                        &= \frac{i}{2} g^{ij} \epsilon_{iab} \frac{p_j}{\norm{\vetor{p}}}p_k\commutator*{\gamma^a \gamma^b}{\gamma^0 \gamma^k}\\
   %                        &= \frac{i}{2} g^{ij} \epsilon_{iab} \frac{p_j}{\norm{\vetor{p}}}p_k\left(\commutator*{\gamma^a \gamma^b}{\gamma^0}\gamma^k + \gamma^0\commutator*{\gamma^a \gamma^b}{\gamma^k}\right)\\
   %                        &= \frac{i}{2} g^{ij} \epsilon_{iab} \frac{p_j}{\norm{\vetor{p}}}p_k\gamma^0\commutator*{\gamma^a \gamma^b}{\gamma^k}\\
   %                        &= \frac{i}{2} g^{ij} \epsilon_{iba} \frac{p_j}{\norm{\vetor{p}}}p_k\gamma^0\left(\anticommutator*{\gamma^k}{\gamma^a}\gamma^b - \gamma^a\anticommutator*{\gamma^k}{\gamma^b}\right)\\
   %                        &= i g^{ij} \epsilon_{iba} \frac{p_j}{\norm{\vetor{p}}}p_k\gamma^0\left(g^{ka}\gamma^b - \gamma^ag^{kb}\right)\\
   %                        &= \\
   %                        % &= \frac{i}{2} g^{ij} \epsilon_{iba} \frac{p_j}{\norm{\vetor{p}}} \left[\anticommutator*{\gamma^0\left(m + \gamma^k p_k\right)}{\gamma^a} \gamma^b - \gamma^a \anticommutator*{\gamma^0\left(m + \gamma^k p_k\right)}{\gamma^b}\right]\\
   %                        % &= \frac{i}{2} g^{ij} \epsilon_{iba} p_k \frac{p_j}{\norm{\vetor{p}}} \left(\anticommutator*{\gamma^0\gamma^k}{\gamma^a} \gamma^b - \gamma^a \anticommutator*{\gamma^0\gamma^k}{\gamma^b} + mg^{0a} \gamma^b - m g^{0b}\gamma^a\right)\\
   %                        % &= \frac{i}{2} g^{ij} \epsilon_{iba} p_k \frac{p_j}{\norm{\vetor{p}}} \left(\commutator*{\gamma^0\gamma^k}{\gamma^a} \gamma^b + \gamma^a \commutator*{\gamma^0\gamma^k}{\gamma^b}\right)\\
   %                        % &= \frac{i}{2} g^{ij} \epsilon_{iab} p_k \frac{p_j}{\norm{\vetor{p}}} \left(\anticommutator*{\gamma^a}{\gamma^0} \gamma^k \gamma^b - \gamma^0 \anticommutator*{\gamma^a}{\gamma^k} \gamma^b + \gamma^a \anticommutator*{\gamma^b}{\gamma^0} \gamma^k - \gamma^a \gamma^0 \anticommutator*{\gamma^b}{\gamma^k}\right)\\
   %                        % &= i g^{ij} \epsilon_{iba} p_k \frac{p_j}{\norm{\vetor{p}}} \left(g^{ak}\gamma^0  \gamma^b + g^{bk}\gamma^a \gamma^0 \right)\\
   %                        % &= \frac{i}{2} g^{ij} \epsilon_{iba} p_k \frac{p_j}{\norm{\vetor{p}}} \left(g^{ak}\gamma^0  \gamma^b + g^{bk}\gamma^a \gamma^0 - g^{bk} \gamma^0 \gamma^a - g^{ak} \gamma^b \gamma^0\right)\\
   %                        % &= \frac{i}{2} g^{ij} \epsilon_{iba} p_k \frac{p_j}{\norm{\vetor{p}}} \left[g^{ak} \commutator*{\gamma^0}{\gamma^b} + g^{bk} \commutator*{\gamma^a}{\gamma^0}\right]\\
   %                        % &= g^{ij} \epsilon_{iba} p_k \frac{p_j}{\norm{\vetor{p}}} \left(g^{ak} \sigma^{0b} - g^{bk} \sigma^{0a}\right)
   % \end{align*}

   % vim: spl=en
\begin{problem}{\(\gamma\) matrices properties}{p2}
   Prove the following relations for \(\gamma\) matrices
   \begin{enumerate}[label=(\alph*)]
       \item \(\gamma^\alpha \gamma^\beta \gamma_\alpha = - 2\gamma^\beta\);
       \item \(\gamma^\alpha \gamma^\beta \gamma^\nu \gamma_\alpha = 4 g^{\nu \beta}\);
       \item \(\gamma^\alpha \gamma^\beta \gamma^\nu \gamma^\mu \gamma_\alpha = -2 \gamma^\mu \gamma^\nu \gamma^\beta\); and
       \item \(\Tr\left(\gamma^\alpha \gamma^\beta \gamma^\nu \gamma^\mu\right) = 4(g^{\alpha \beta} g^{\nu \mu} - g^{\alpha \nu} g^{\beta \mu} + g^{\alpha \mu} g^{\beta \nu})\).
   \end{enumerate}
\end{problem}
\begin{proof}[Solution]
   Notice
   \begin{equation*}
      \gamma^\alpha\gamma_\alpha = g_{\alpha \beta} \gamma^\alpha \gamma^\beta = - g_{\alpha \beta} \gamma^{\beta}\gamma^{\alpha} + 2g_{\alpha \beta}g^{\alpha \beta} = - \gamma^\alpha \gamma_\alpha + 8 \implies \gamma^\alpha \gamma_\alpha = 4,
   \end{equation*}
   then
   \begin{equation*}
      \gamma^\alpha \gamma^\beta \gamma_\alpha = - 4\gamma^\beta + 2g^{\alpha \beta} \gamma_{\alpha} = -2 \gamma^\beta.
   \end{equation*}
   We also have
   \begin{equation*}
      \anticommutator{\gamma^\nu}{\gamma_{\alpha}} = g_{\alpha \beta}\anticommutator{\gamma^\nu}{\gamma^\beta} = 2g_{\alpha \beta} g^{\beta \nu} = 2 \delta\indices{^\nu_\alpha},
   \end{equation*}
   then
   \begin{equation*}
      \gamma^\alpha \gamma^\beta \gamma^\nu \gamma_{\alpha} = - \gamma^\alpha \gamma^\beta \gamma_{\alpha} \gamma^\nu + 2 \gamma^\alpha \gamma^\beta \delta\indices{^\nu_\alpha} = 2 \gamma^\beta \gamma^\nu + 2 \gamma^{\nu}\gamma^\beta = 2 \anticommutator{\gamma^\nu}{\gamma^\beta} = 4 g^{\nu \beta}
   \end{equation*}
   and
   \begin{equation*}
      \gamma^\alpha \gamma^\beta \gamma^\nu \gamma^\mu \gamma_{\alpha} = - 4g^{\nu \beta} \gamma^\mu + 2 \gamma^\mu \gamma^\beta \gamma^\nu = -4 g^{\nu \beta} \gamma^\mu - 2 \gamma^\mu \gamma^\beta \gamma^\nu + 4 \gamma^\mu g^{\beta\nu} = -2 \gamma^\mu \gamma^\beta \gamma^\nu.
   \end{equation*}

   For the trace, we build it up from \(\Tr(\gamma^\alpha) = 0,\) the anticommutation relations and the cyclic property of trace. We have
   \begin{equation*}
      \Tr(\gamma^\alpha \gamma^\beta) = -\Tr(\gamma^\beta \gamma^\alpha) + 8g^{\alpha \beta} = -\Tr(\gamma^\alpha \gamma^\beta) + 8g^{\alpha \beta} \implies \Tr(\gamma^\alpha \gamma^\beta) = 4g^{\alpha \beta},
   \end{equation*}
   then
   \begin{align*}
      \Tr(\gamma^\alpha \gamma^\beta \gamma^\nu \gamma^\mu) &= -\Tr(\gamma^\alpha \gamma^\beta \gamma^\mu \gamma^\nu) + 2g^{\mu\nu}\Tr(\gamma^\alpha \gamma^\beta)\\
                                                            &= \Tr(\gamma^\alpha \gamma^\mu\gamma^\beta \gamma^\nu) - 8g^{\mu \beta} g^{\alpha \nu} + 8 g^{\mu\nu} g^{\alpha \beta}\\
                                                            &= -\Tr(\gamma^\mu \gamma^\alpha \gamma^\beta \gamma^\nu) - 8 g^{\mu\beta}g^{\alpha \nu} + 8 g^{\mu \nu}g^{\alpha \beta} + 8 g^{\mu \alpha} g^{\nu \beta}\\
                                                            &= -\Tr(\gamma^\alpha \gamma^\beta \gamma^\nu \gamma^\mu) + 8(g^{\alpha \beta}g^{\nu\mu} - g^{\alpha \nu}g^{\mu \beta} + g^{\alpha\mu}g^{\beta\nu})
   \end{align*}
   hence \(\Tr(\gamma^\alpha \gamma^\beta \gamma^\nu) = 4(g^{\alpha \beta} g^{\nu\mu}-g^{\alpha\nu}g^{\beta\mu} + g^{\alpha \mu}g^{\beta \nu}).\) Another trace property we use is that the trace of three gamma matrices vanishes
   \begin{equation*}
      \Tr(\gamma^\alpha \gamma^\beta \gamma^\nu) = - \Tr(\gamma^\alpha \gamma^\nu \gamma^\beta) + 2g^{\beta \nu} \Tr(\gamma^\alpha) = - \Tr(\gamma^\alpha \gamma^\nu \gamma^\beta) \implies \Tr(\gamma^\alpha \gamma^\beta \gamma^\nu) = 0
   \end{equation*}
   and it generalizes to the vanishing of the trace of any odd number of gamma matrices.
\end{proof}

   % vim: spl=en
\begin{problem}{Cross section for annihilation of electron-positron in QED}{p3}
    In QED, evaluate the unpolarized cross section for \(e^+ e^- \to \gamma \gamma\) as a function of the final state photon polarizations\footnote{We'll ignore this and sum over photon polarizations.}. Take into account the non-vanishing electron mass.
\end{problem}
\begin{proof}[Solution]
    For the annihilation \(e^+ e^- \to \gamma \gamma\) we consider the lowest order diagrams
    \begin{center}
        \feynmandiagram[inline=(a.base),medium,vertical=a to b]{
            i1 [particle=\(e^-\)] -- [fermion, momentum'=\(p_1\)] a -- [photon, momentum'=\(p_3\)] f1 [particle=\(\gamma\)];
            a -- [fermion, momentum=\(p_1 - p_3\)] b;
            b -- [photon, momentum=\(p_4\)] f2 [particle=\(\gamma\)];
            i2 [particle=\(e^+\)] -- [anti fermion, momentum=\(p_2\)] b;
        };
        and
        \feynmandiagram[inline=(a.base),medium,vertical=a to b]{
            i1 [particle=\(e^-\)] -- [fermion, momentum'=\(p_1\)] a -- [photon, momentum'=\(p_4\)] f1 [particle=\(\gamma\)];
            a -- [fermion, momentum=\(p_1 - p_4\)] b;
            b -- [photon, momentum=\(p_3\)] f2 [particle=\(\gamma\)];
            i2 [particle=\(e^+\)] -- [anti fermion, momentum=\(p_2\)] b;
        };
    \end{center}
    with amplitudes
    \begin{align*}
        i \mathcal{M}_t &= \bar{v}(p_2) (i e \gamma^\nu) \epsilon^*_{\nu}(p_4) i\frac{\slashed{p}_1 - \slashed{p}_3 + m}{t - m^2} \epsilon^*_{\mu}(p_3) (i e \gamma^\mu) u(p_1) \\
                        &= i(ie)^2 \bar{v}(p_2) \slashed{\epsilon}^*(p_4) \frac{\slashed{p}_1 - \slashed{p}_3 + m}{t - m^2} \slashed{\epsilon}^*(p_3) u(p_1)
    \end{align*}
    and
    \begin{equation*}
        i\mathcal{M}_u = i(ie)^2 \bar{v}(p_2) \slashed{\epsilon}^*(p_3) \frac{\slashed{p}_1 - \slashed{p}_4 + m}{u - m^2} \slashed{\epsilon}^*(p_4) u(p_1).
    \end{equation*}
    In order to compute the cross section, we'll compute the amplitude
    \begin{equation*}
        i \mathcal{M} = i\mathcal{M}_t +  i\mathcal{M}_u,
    \end{equation*}
    sum over photon polarizations, and average over the initial spins
    \begin{equation*}
        \overline{\abs{i\mathcal{M}}^2} = \frac14\sum_{\lambda,\lambda'}\sum_{\mathrm{spin}}\abs{i \mathcal{M}}^2 = \underbrace{\frac14\sum_{\lambda,\lambda'}\sum_{\mathrm{spin}}\abs{i \mathcal{M}_t}^2}_{\overline{\abs{i\mathcal{M}_t}^2}} + \underbrace{\frac14\sum_{\lambda,\lambda'}\sum_{\mathrm{spin}} \abs{i\mathcal{M}_u}^2}_{\overline{\abs{i\mathcal{M}_u}^2}} + \underbrace{\frac12\sum_{\lambda,\lambda'}\sum_{\mathrm{spin}} \Re\left(\mathcal{M}_t^*\mathcal{M}_u\right)}_{2\overline{\Re\left(\mathcal{M}_t^*\mathcal{M}_u\right)}}.
    \end{equation*}
    As we have
    \begin{equation*}
        (\bar{v} \Lambda u)^* = \bar{u} \gamma^0 \herm{\Lambda} \gamma^0 v,
    \end{equation*}
    we compute
    \begin{equation*}
        [\slashed{a}^* (\slashed{b} + c) \slashed{d}^*]^\dag = \gamma^0 \slashed{d} \gamma^0 \gamma^0 (\slashed{b} + c) \gamma^0 \gamma^0 \slashed{a} \gamma^0 = \gamma^0 \slashed{d} (\slashed{b} + c) \slashed{a} \gamma^0,
    \end{equation*}
    then
    \begin{equation*}
        (i \mathcal{M}_t)^* = -i(ie)^2 \bar{u}(p_1) \slashed{\epsilon}(p_3) \frac{\slashed{p}_1 - \slashed{p}_3 + m}{t - m^2} \slashed{\epsilon}(p_4) v(p_2).
    \end{equation*}
    % and
    % \begin{equation*}
    %     (i \mathcal{M}_u)^* = -i(ie)^2 \bar{u}(p_1) \slashed{\epsilon}(p_4) \frac{\slashed{p}_1 - \slashed{p}_4 + m}{u - m^2} \slashed{\epsilon}(p_3) v(p_2) 
    % \end{equation*}
    For the \(t\)-channel we have
    \begin{equation*}
        \abs{i\mathcal{M}_t}^2 = e^4 \bar{u}_a^{s_1}(p_1) \left[\slashed{\epsilon}(p_3) \frac{\slashed{p}_1 - \slashed{p}_3 + m}{t - m^2}\slashed{\epsilon}(p_4)\right]_{ab} v^{s_2}_b(p_2)\bar{v}_c^{s_2}(p_2)\left[\slashed{\epsilon}^*(p_4) \frac{\slashed{p}_1 - \slashed{p}_3 + m}{t - m^2}\slashed{\epsilon}^*(p_3)\right]_{cd}u^{s_1}_d(p_1)
    \end{equation*}
    then
    \begin{align*}
        \frac14 \sum_{\mathrm{spin}}{\abs{i\mathcal{M}_t}^2} &= \frac{e^4}{4(t - m^2)^2} \Tr\left[(\slashed{p}_1 + m)\slashed{\epsilon}(p_3)(\slashed{p}_1 - \slashed{p}_3 + m) \slashed{\epsilon}(p_4)(\slashed{p}_2 - m)\slashed{\epsilon}^*(p_4) (\slashed{p}_1 - \slashed{p}_3 + m) \slashed{\epsilon}^*(p_3)\right]\\
                                                             &= \frac{e^4}{4(t - m^2)^2} \Tr\left[\slashed{\epsilon}^*(p_3)(\slashed{p}_1 + m)\slashed{\epsilon}(p_3)(\slashed{p}_1 - \slashed{p}_3 + m) \slashed{\epsilon}(p_4)(\slashed{p}_2 - m)\slashed{\epsilon}^*(p_4) (\slashed{p}_1 - \slashed{p}_3 + m)\right].
    \end{align*}
    When we sum over photon polarizations, we will get
    \begin{equation*}
        \sum_{\lambda} \sum_{\lambda'} \slashed{\epsilon}^*(p_3, \lambda) \dots \slashed{\epsilon}(p_3,\lambda) \dots \slashed{\epsilon}(p_4,\lambda') \dots \slashed{\epsilon}^*(p_4, \lambda') \dots  = \gamma^\mu \dots \gamma_\mu \dots \gamma^\nu \dots \gamma_\nu \dots,
    \end{equation*}
    then we may use \cref{prob:p2} to obtain
    \begin{equation*}
        \gamma^\nu (\slashed{p}_2 - m) \gamma_\nu = - 2(\slashed{p}_2 + 2m)\quad\text{and}\quad
        \gamma^\mu (\slashed{p}_1 + m) \gamma_\mu = -2(\slashed{p}_1 - 2m),
    \end{equation*}
    hence
    \begin{equation*}
        \overline{\abs{i\mathcal{M}_t}^2} = \frac{e^4}{(t - m^2)^2} \Tr\left[(\slashed{p}_1 - 2m)(\slashed{p}_1 - \slashed{p}_3 + m) (\slashed{p}_2 + 2m)(\slashed{p}_1 - \slashed{p}_3 + m)\right].
    \end{equation*}
    Henceforth we will use the following results concerning Mandelstam variables: first, their sum,
    \begin{equation*}
        s + t + u = p_1^2 + p_2^2 + p_3^2 + p_4^2 = 2m^2,
    \end{equation*}
    then the momentum product identities, for \(s\)
    \begin{equation*}
        2 p_1 p_2 = s - p_1^2 - p_2^2 = s - 2m^2,\quad\text{and}\quad
        2 p_3 p_4 = s - p_3^2 - p_4^2 = s
    \end{equation*}
    for \(t\)
    \begin{equation*}
        2 p_1 p_3 = p_1^2 + p_3^2 - t = m^2 - t,\quad\text{and}\quad
        2 p_2 p_4 = p_2^2 + p_4^2 - t = m^2 - t
    \end{equation*}
    and for \(u\)
    \begin{equation*}
        2 p_1 p_4 = p_1^2 + p_4^2 - u = m^2 - u,\quad\text{and}\quad
        2 p_2 p_3 = p_2^2 + p_3^2 - u = m^2 - u.
    \end{equation*}
    As the product of an odd number of \(\gamma\) matrices is traceless, we need only collect the terms of the \(m\) polynomial with even degree. For \(m^0\) we have
    \begin{align*}
        \Tr[\slashed{p}_1 (\slashed{p}_1 - \slashed{p}_3) \slashed{p}_2 (\slashed{p}_1 - \slashed{p}_3)] 
        &= 8(p_1^2 - p_1p_3)(p_2 p_1 - p_2 p_3) - 4 p_1 p_2 t\\
        &= 2(2 m^2 - 2p_1 p_3)(2p_2 p_1 - 2p_2 p_3) - 2 (s - 2m^2)t\\
        &= 2(m^2 + t)(s - 3m^2 + u) - 2(s - 2m^2)t\\
        &= 2(m^2 + t)(s + u + t - 3m^2 - t) - 2(s - 2m^2)t\\
        &= - 2(m^4 + 2m^2 t + t^2) - 2(s - 2m^2)t\\
        &= -2m^4 -2t^2 -2st,
    \end{align*}
    for \(m^2\) we have
    \begin{align*}
        \Tr[\slashed{p}_1 \slashed{p}_2 + 4(\slashed{p}_1 - \slashed{p}_2) (\slashed{p}_1 - \slashed{p}_3) - 4 (\slashed{p}_1 - \slashed{p}_3)^2] 
        &= 4 p_1 p_2 + 16 (p_1 - p_2)(p_1 - p_3) - 16 t\\
        &= 2 s - 4m^2 +  8(2m^2 - 2p_1 p_3 - 2p_1 p_2 + 2p_2 p_3) - 16t\\
        &= 2s - 4m^2 + 8(4m^2 + t - s - u) - 16t\\
        &= 2s - 4m^2 + 8(2t + 2m^2) - 16t\\
        &= 2s + 12m^2
    \end{align*}
    then
    \begin{align*}
        \overline{\abs{i\mathcal{M}_t}^2} &= \frac{e^4\left[(-2m^4 - 2t^2-2st)m^0 + (2s + 12m^2)m^2 - 16m^4\right]}{(t - m^2)^2}\\
                                          &= \frac{2e^4\left(m^2s-3m^4 - t^2 - st\right)}{(t - m^2)^2}.
    \end{align*}
    Replacing \(t \to u\) we obtain the \(u\) channel amplitude,
    \begin{align*}
        \overline{\abs{i\mathcal{M}_u}^2} = \frac{2e^4\left(m^2s-3m^4 - u^2 - su\right)}{(u - m^2)^2}.
    \end{align*}
    For the interference term, we have
    \begin{equation*}
        \mathcal{M}_t^*\mathcal{M}_u = 
        e^4 \bar{u}^{s_1}_a(p_1) \left[\slashed{\epsilon}(p_3) \frac{\slashed{p}_1 - \slashed{p}_3 + m}{t - m^2} \slashed{\epsilon}(p_4)\right]_{ab} v^{s_2}_b(p_2)\bar{v}^{s_2}_c(p_2) \left[\slashed{\epsilon}^*(p_3) \frac{\slashed{p}_1 - \slashed{p}_4 + m}{u - m^2} \slashed{\epsilon}^*(p_4)\right]_{cd} u^{s_1}_d(p_1),
    \end{equation*}
    then
    \begin{equation*}
        \frac12 \sum_{\mathrm{spin}} \mathcal{M}_t^* \mathcal{M}_u = \frac{e^4\Tr\left[(\slashed{p}_1 + m) \slashed{\epsilon}(p_3)(\slashed{p}_1 - \slashed{p}_3 + m) \slashed{\epsilon}(p_4) (\slashed{p}_2 - m) \slashed{\epsilon}^*(p_3) (\slashed{p}_1 - \slashed{p}_4 + m) \slashed{\epsilon}^*(p_4)\right]}{2(t - m^2)(u - m^2)}.
    \end{equation*}
    Before summing over polarizations, we consider the term between the \(\slashed{\epsilon}(p_3, \lambda)\) polarizations
    \begin{align*}
        (\slashed{p}_1 - \slashed{p}_3 + m) \slashed{\epsilon}(p_4) (\slashed{p}_2 - m) 
        &= (\slashed{p}_1 - \slashed{p}_3 + m)( \slashed{\epsilon}(p_4) \slashed{p}_2 - m \slashed{\epsilon}(p_4))\\
        &= (\slashed{p}_1 - \slashed{p}_3)\slashed{\epsilon}(p_4) \slashed{p}_2 - m(\slashed{p}_1 - \slashed{p}_3)\slashed{\epsilon}(p_4) + m \slashed{\epsilon}(p_4) \slashed{p}_2 - m^2 \slashed{\epsilon}(p_4)\\
        &=(\slashed{p}_1 - \slashed{p}_3)\slashed{\epsilon}(p_4) \slashed{p}_2 + m \left[\slashed{\epsilon}(p_4) \slashed{p}_2 - (\slashed{p}_1 - \slashed{p}_3) \slashed{\epsilon}(p_4)\right] - m^2 \slashed{\epsilon}(p_4)
    \end{align*}
    then when we sum over \(\lambda\) this term becomes
    \begin{equation*}
        \sum_{\lambda} \slashed{\epsilon}(p_3)(\slashed{p}_1 - \slashed{p}_3 + m) \slashed{\epsilon}(p_4) (\slashed{p}_2 - m) \slashed{\epsilon}^*(p_3) = 2 \slashed{p}_2 \slashed{\epsilon}(p_4) (\slashed{p}_1 - \slashed{p}_3) - 4m(p_2 + p_3 - p_1) \epsilon(p_4) - 2m^2 \slashed{\epsilon}(p_4)
    \end{equation*}
    and hence 
    \begin{align*}
        \frac12 \sum_{\lambda} \sum_{\mathrm{spin}} \mathcal{M}_t^*\mathcal{M}_u 
        &= -\frac{e^4}{(t - m^2)(u - m^2)}\left\{m^2\Tr\left[(\slashed{p}_1 + m) \slashed{\epsilon}(p_4) (\slashed{p}_1 - \slashed{p}_4 + m) \slashed{\epsilon}^*(p_4)\right]\right. + \\
        &{}\phantom{=-{(t - m^2)(u - m^2)}} {}+ 2m(p_2 + p_3 - p_1) \epsilon(p_4) \Tr\left[(\slashed{p}_1 + m) (\slashed{p}_1 - \slashed{p}_4 + m) \slashed{\epsilon}^*(p_4)\right]+\\
        &{}\phantom{=-{(t - m^2)(u - m^2)}} \left.{}- \Tr\left[(\slashed{p}_1 + m) \slashed{p}_2 \slashed{\epsilon}(p_4) (\slashed{p}_1 - \slashed{p}_3) (\slashed{p}_1 - \slashed{p}_4 + m) \slashed{\epsilon}^*(p_4)\right]\right\}.
    \end{align*}
    When we sum over \(\lambda'\) we will get the following developments
    \begin{equation*}
        \sum_{\lambda'}\slashed{\epsilon}(p_4)(\slashed{p}_1 - \slashed{p}_4 + m) \slashed{\epsilon}^*(p_4) = 2(\slashed{p}_1 - \slashed{p}_4 - 2m),
    \end{equation*}
    \begin{equation*}
        \sum_{\lambda'} \epsilon_\mu(p_4) \Tr\left[(\slashed{p}_1 + m) (\slashed{p}_1 - \slashed{p}_4 + m) \slashed{\epsilon}^*(p_4)\right] = - g_{\mu\nu} \Tr\left[(\slashed{p}_1 + m)(\slashed{p}_1 - \slashed{p}_4 + m) \gamma^\nu\right],
    \end{equation*}
    and
    \begin{equation*}
        \sum_{\lambda'} \slashed{\epsilon}(p_4) (\slashed{p}_1 - \slashed{p}_3) (\slashed{p}_1 - \slashed{p}_4 + m) \slashed{\epsilon}^*(p_4) = - 4(p_1 - p_3)(p_1 - p_4) + 2m (\slashed{p}_1 - \slashed{p}_3)
    \end{equation*}
    then
    \begin{align*}
        \frac12 \sum_{\lambda, \lambda'} \sum_{\mathrm{spin}} \mathcal{M}_t^* \mathcal{M}_u 
        &= - \frac{e^4}{(t - m^2)(u - m^2)} \left\{2m^2 \Tr\left[(\slashed{p}_1 + m)(\slashed{p}_1 - \slashed{p}_4 - 2m)\right]+\right.\\
        &{}\phantom{=-(t - m^2)(u - m^2)} -2m \Tr\left[(\slashed{p}_1 + m)(\slashed{p}_1 - \slashed{p}_4 + m)(\slashed{p}_2 + \slashed{p}_3 - \slashed{p}_1)\right]\\
        &{}\phantom{=-(t - m^2)} \left.+ 4(p_1 - p_3)(p_1 - p_4) \Tr\left[(\slashed{p}_1 + m)\slashed{p}_2\right] - 2m \Tr\left[(\slashed{p}_1 + m)\slashed{p}_2(\slashed{p}_1 - \slashed{p}_3)\right]\right\}.
    \end{align*}
    We compute the traces separately,
    \begin{align*}
        \Tr\left[(\slashed{p}_1 + m)(\slashed{p}_1 - \slashed{p}_4 - 2m)\right] 
        &= \Tr\left[\slashed{p}_1(\slashed{p}_1 - \slashed{p}_4)\right] - 8m^2\\
        &= 2(2p_1^2 - 2p_1 p_4) - 8m^2\\
        &= 2(m^2 + u) - 8m^2\\
        &= 2u - 6m^2,
    \end{align*}
    \begin{align*}
        \Tr\left[(\slashed{p}_1 + m)(\slashed{p}_1 - \slashed{p}_4 + m)(\slashed{p}_2 + \slashed{p}_3 - \slashed{p}_1)\right] 
        &= m\Tr\left[(2\slashed{p}_1 - \slashed{p}_4)(\slashed{p}_2 + \slashed{p}_3 - \slashed{p}_1)\right]\\
        &= 2m (4p_1p_2 + 4p_1p_3 - 4p_1^2 - 2p_4p_2 - 2p_4p_3 + 2p_4p_1)\\
        &= 2m[(4p_1 p_2 - 2p_3 p_4) + (4p_1 p_3 - 2p_2 p_4) + 2p_1 p_4 - 4p_1^2]\\
        &= 2m(s - 4m^2 + m^2 - t + m^2 - u - 4m^2)\\
        &= 2m[2s - (s + t + u) - 6m^2]\\
        &= 4m(s - 4m^2)
    \end{align*}
    \begin{align*}
        (p_1 - p_3)(p_1 - p_4)\Tr\left[(\slashed{p}_1 + m) \slashed{p}_2\right] 
        &= (2p_1^2 - 2p_1 p_4 - 2p_1 p_3 + 2p_3 p_4)(2p_1 p_2)\\
        &= (2m^2 - m^2 + u - m^2 + t + s)(s - 2m^2)\\
        &= 2m^2 (s - 2m^2)
    \end{align*}
    and
    \begin{align*}
        \Tr\left[(\slashed{p}_1 + m)\slashed{p}_2(\slashed{p}_1 - \slashed{p}_3)\right]
        &= m \Tr\left[\slashed{p}_2 (\slashed{p}_1 - \slashed{p}_3)\right]\\
        &= 2m (2p_1 p_2 -2 p_2 p_3)\\
        &= 2m (s - 3m^2 + u)\\
        &= -2m (m^2 + t)
    \end{align*}
    then noticing all of these results are real, we have
    \begin{align*}
        2\overline{\Re(\mathcal{M}_t^*\mathcal{M}_u)} &= -\frac{e^4}{(t - m^2)(u - m^2)} \left[2m^2(2u - 6m^2) - 8m^2(s - 4m^2) + 8m^2(s - 2m^2) + 4m^2(m^2 + t)\right]\\
                                                      &= -\frac{4m^2e^4\left[(u - 3m^2) - 2(s - 4m^2) + 2(s - 2m^2) + m^2 + t\right]}{(t - m^2)(u - m^2)}\\
                                                      &= - \frac{4m^2 e^4 (2m^2 + u + t)}{(t - m^2)(u - m^2)}
    \end{align*}
    Then, the amplitude is
    \begin{equation*}
        \overline{\abs{i\mathcal{M}}^2} = \frac{2e^4\left(m^2s-3m^4 - t^2 - st\right)}{(t - m^2)^2} + \frac{2e^4\left(m^2s-3m^4 - u^2 - su\right)}{(u - m^2)^2}- \frac{4m^2 e^4(t + u + 2m^2)}{(t - m^2)(u - m^2)},
    \end{equation*}
    which yields
    \begin{equation*}
        \overline{\abs{i\mathcal{M}}^2} \to -2e^4\left(\frac{t + s}{t} + \frac{u + s}{u}\right) = 2e^4 \left(\frac{u}{t} + \frac{t}{u}\right)
    \end{equation*}
    in the high energy limit \(m \to 0.\)

    The differential cross section of the pair annihilation is
    \begin{equation*}
        \dl{\sigma} = \dli{\Pi_2} \frac{\overline{\abs{i\mathcal{M}}^2}}{F},
    \end{equation*}
    where the differential phase space is, by \href{https://github.com/louisradial/4305107-quantum-field-theory-i/releases/tag/pset8}{Problem 3 of Problem Set VIII},
    \begin{equation*}
        \dl{\Pi_2} = \dli{\Omega} \frac{1}{32\pi^2},
    \end{equation*}
    and, as we've shown in \href{https://github.com/louisradial/4305107-quantum-field-theory-i/releases/tag/pset7}{Problem 4 of Problem Set VII}, the flux factor in the center of momentum frame is
    \begin{equation*}
    F = (2E_1)(2E_2) \sqrt{(\vetor{v}_1 - \vetor{v}_2)^2 - (\vetor{v}_1 \times \vetor{v}_2)^2} = 2 s\sqrt{1 - \frac{4m^2}{s}},
    \end{equation*}
    hence
    \begin{equation*}
        \diff{\sigma}{\Omega} = \frac{e^4}{32\pi^2 s\sqrt{1 - \frac{4m^2}{s}}}\left[\frac{m^2s-3m^4 - t^2 - st}{(t - m^2)^2} + \frac{m^2s-3m^4 - u^2 - su}{(u - m^2)^2}- \frac{2m^2 (t + u + 2m^2)}{(t - m^2)(u - m^2)}\right]
    \end{equation*}
    is the cross section for \(e^+ e^- \to \gamma \gamma.\)
\end{proof}

   \begin{problem}{Conserved quantity due to electromagnetism's Lorentz invariance}{p4}
   Consider the vector field \(A^\mu\) associated to the electromagnetic field. Obtain the conserved quantities due to the Lorentz invariance of the theory. Use the Langrangian density given in the first problem.
\end{problem}
\begin{proof}[Solution]
   We consider the infinitesimal Lorentz transformation
   \begin{equation*}
      x^\mu \mapsto \tilde{x}^\mu = \left(\delta\indices{^\mu_\nu} + \omega\indices{^\mu_\nu}\right)x^\nu
      \quad\text{and}\quad
      A_\mu(x) \mapsto \tilde{A}_\mu(\tilde{x}) = \diffp{x^\nu}{\tilde{x}^\mu} A_\nu(x).
   \end{equation*}
   As the transformation is infinitesimal, we have \(\diffp{x^\nu}{\tilde{x}^\mu} = \delta\indices{^\nu_\mu} - \omega\indices{^\nu_\mu}\), hence
   \begin{align*}
      \delta A_\mu(x) = \tilde{A}_\mu(x) - A_\mu(x) 
      &=\left(\delta\indices{^\nu_\mu} - \omega\indices{^\nu_\mu}\right) A_\nu(x - \omega x) - A_\mu(x)\\
      &=\left(\delta\indices{^\nu_\mu} - \omega\indices{^\nu_\mu}\right) \left[A_\nu(x) - \omega\indices{^\rho_\sigma} x^\sigma \partial_\rho A_\nu(x)\right] - A_\mu(x)\\
      &=  - \omega\indices{^\nu_\mu} A_\nu(x) - \omega\indices{^\lambda_\sigma} x^\sigma \partial_\lambda A_\mu(x)
   \end{align*}
   and \(\Delta x^\rho = \omega\indices{^\rho_\eta}x^\eta\). As we have shown in \cref{prob:p1}, we have
   \begin{equation*}
      \diffp{\mathcal{L}}{(\partial_\rho A_\mu)} = - F^{\rho\mu},
   \end{equation*}
   then the conserved current guaranteed by \nameref{thm:noether} is
   \begin{align*}
      j^\rho = \diffp{L}{(\partial_\rho A_\mu)}\delta A_\mu(x) + \mathcal{L} \Delta x^\rho
      &= F^{\rho\mu} \left[\omega\indices{^\sigma_\mu} A_\sigma + \omega\indices{^\sigma_\lambda} x^\lambda \partial_\sigma A_\mu\right] - \frac14 F^{\alpha \beta}F_{\alpha \beta} \omega\indices{^\rho_\lambda} x^\lambda\\
      &= \omega\indices{^\sigma_\lambda}\left[F^{\rho \mu}\left(\delta\indices{^\lambda_\mu} A_\sigma + x^\lambda \partial_\sigma A_\mu\right) - \frac14 F^{\alpha \beta}F_{\alpha \beta} \delta\indices{^\rho_\sigma}x^\lambda\right]\\
      &= \omega_{\nu \lambda}\left[F^{\rho \lambda} A^\nu + x^\lambda \left(F^{\rho \mu} \partial^\nu A_\mu - \frac14 g^{\rho \nu} F^{\alpha \beta} F_{\alpha \beta}\right)\right]\\
      &= \omega_{\nu \lambda} \left[F^{\rho \lambda} A^\nu + x^\lambda F^{\rho \mu} \partial_\mu A^\nu + x^\lambda T^{\rho \nu}\right]\\
      &= \omega_{\nu \lambda} \left[F^{\rho \mu} \delta\indices{^\lambda_\mu} A^\nu + x^\lambda F^{\rho \mu} \partial_\mu A^\nu + x^\lambda T^{\rho \nu}\right]\\
      &= \omega_{\nu \lambda} \left[F^{\rho \mu} A^\nu \partial_\mu x^\lambda + x^\lambda F^{\rho \mu} \partial_\mu A^\nu + x^\lambda T^{\rho \nu}\right]\\
      &= \omega_{\nu \lambda} \left[F^{\rho \mu} \partial_\mu( x^\lambda A^\nu) + x^\lambda T^{\rho \nu}\right]\\
      &= \frac12 \omega_{\nu \lambda} \left[F^{\rho \mu} \partial_\mu( x^\lambda A^\nu) + x^\lambda T^{\rho \nu}\right] + \frac12 \omega_{\lambda \nu}\left[F^{\rho \mu} \partial_\mu( x^\nu A^\lambda) + x^\nu T^{\rho \lambda}\right]\\
      &= \frac12 \omega_{\nu \lambda} \left[x^{\lambda} T^{\rho \nu} - x^\nu T^{\rho \lambda} + F^{\rho \mu} \partial_\mu (x^\lambda A^\nu - x^\nu A^{\lambda})\right] 
   \end{align*}
   where \(T^{\rho\nu}\) is the symmetric stress-energy tensor defined in \cref{prob:p1}. As the Lorentz transformation components are arbitrary, the tensor
   \begin{equation*}
      \tilde{M}^{\lambda \rho \nu} = x^{\lambda} T^{\rho \nu} - x^\nu T^{\rho \lambda} + F^{\rho \mu} \partial_\mu(x^\lambda A^\nu - x^\nu A^\lambda)
   \end{equation*}
   is conserved with \(\partial_\rho \tilde{M}^{\lambda \rho \nu} = 0\). The equations of motion then yield the conservation of the tensor \(M^{\lambda \rho \nu} = x^{\lambda}T^{\rho \nu} - x^{\nu}T^{\rho \lambda}\) since we have
   \begin{equation*}
      \partial_\rho (\tilde{M}^{\lambda \rho \nu} - M^{\lambda \rho \nu}) = \partial_\rho \left[F^{\rho \mu} \partial_\mu(x^\lambda A^\nu - x^\nu A^\lambda)\right]  = F^{\rho \mu} \partial_\rho \partial_\mu (x^\lambda A^\nu - x^\nu A^\lambda) = 0
   \end{equation*}
   since \(F^{\rho \mu}\) is antisymmetric and the partial derivatives commute.
\end{proof}

\end{document}
