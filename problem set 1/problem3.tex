\begin{problem}{Infinitesimal Lorentz transformations}{p3}
   Consider an infinitesimal Lorentz transformation \(\tilde{x}^\mu = \Lambda\indices{^\mu_\nu}x^\nu\) where \(\Lambda\indices{^\mu_\nu} = \delta\indices{^\mu_\nu} + \omega\indices{^\mu_\nu}.\)
   \begin{enumerate}[label=(\alph*)]
      \item Show that \(\omega_{\mu\nu} = -\omega_{\nu\mu}\).
      \item For a rotation, what is the form of \(\omega_{\mu\nu}\)?
      \item For a scalar field \(\phi(x)\) obtain the conserved current associated to a Lorentz transformation.
   \end{enumerate}
\end{problem}
\begin{proof}[Solution]
    As Lorentz transformations must preserve the pseudo inner product, we have
    \begin{align*}
       g_{\mu\nu} \tilde{x}^\mu \tilde{x}^\nu = g_{\mu\nu}x^\mu x^\nu 
       &\implies g_{\mu\nu}\left(\delta\indices{^\mu_\rho} + \omega\indices{^\mu_\rho}\right) \left(\delta\indices{^\nu_\sigma} + \omega\indices{^\nu_\sigma}\right) x^\rho x^\sigma - g_{\mu\nu} x^{\mu} x^\nu = 0\\
       &\implies g_{\mu\nu}x^\rho x^\sigma \left[\left(\delta\indices{^\mu_\rho} + \omega\indices{^\mu_\rho}\right) \left(\delta\indices{^\nu_\sigma} + \omega\indices{^\nu_\sigma}\right) - \delta\indices{^\mu_\rho} \delta\indices{^\nu_\sigma}\right] = 0\\
       &\implies x^\rho x^\sigma \left[\left(g_{\rho \nu} + \omega_{\nu \rho}\right)\left(\delta\indices{^\nu_\sigma} + \omega\indices{^\nu_\sigma}\right) - g_{\rho\sigma}\right] = 0\\
       &\implies x^\rho x^\sigma \left[g_{\rho\sigma} + \omega_{\rho \sigma} + \omega_{\sigma \rho} + O(\omega^2) - g_{\rho \sigma}\right] = 0\\
       &\implies x^\rho x^\sigma \left(\omega_{\rho \sigma} + \omega_{\sigma \rho}\right) = 0
    \end{align*}
    for all \(x,\) hence \(\omega_{\rho \sigma} = -\omega_{\sigma \rho}\).

    As rotations are proper and orthochronous Lorentz transformations with the additional property that \(g_{ij} \tilde{x}^i \tilde{x}^j = g_{ij} x^i x^j,\) we have \(\tilde{x}^0 = x^0,\) hence \(\omega_{0\mu} = 0.\) For a rotation along the direction \(\vetor{n} = n^i \vetor{e}_i\) by an infinitesimal angle \(\vartheta,\) we have
    \begin{equation*}
       \tilde{x}^k =  (\vetor{x} + \vartheta \vetor{n} \times \vetor{x})^k = x^k + \vartheta n_i x_j \epsilon^{ijk} = \left(\delta\indices{^k_\ell} + \vartheta n_i g_{j \ell}\epsilon^{ijk}\right) x^\ell,
    \end{equation*}
    then 
    \begin{equation*}
       \omega\indices{^k_\ell} = \vartheta n_i \epsilon\indices{^i_{\ell}^k}
    \end{equation*}
    are the other components of \(\omega\indices{^\mu_\nu}\). As an example, the rotation along the \(\vetor{e}_z\) direction by an infinitesimal angle \(\vartheta\) has \(\omega_{12} = -\vartheta = -\omega_{21}\) and all other components equal to zero.

    Under the Lorentz transformation
    \begin{equation*}
       x^\mu \mapsto \tilde{x}^\mu = \left(\delta\indices{^\mu_\nu} + \omega\indices{^\mu_\nu}\right)x^\nu
       \quad\text{and}\quad
       \phi(x) \mapsto \tilde{\phi}(\tilde{x}) = \phi(x)
    \end{equation*}
    we have \(\Delta x^\mu = \omega\indices{^\mu_\nu} x^\nu\) and
    \begin{equation*}
       \delta \phi(x) = \tilde{\phi}(x) - \phi(x) = \phi(x - \omega x) - \phi(x) = - \omega\indices{^\mu_\nu} x^\nu \partial_\mu \phi(x).
    \end{equation*}
    Let \(\mathcal{L} = \mathcal{L}[\phi, \partial \phi, x]\) be a Lagrangian density for which Lorentz transformations are symmetries, then by \nameref{thm:noether}, the conserved current is
    \begin{align*}
       j^\rho 
       = \diffp{\mathcal{L}}{(\partial_\rho \phi)} \delta \phi + \mathcal{L} \Delta x^\rho
       &= -\diffp{\mathcal{L}}{(\partial_\rho \phi)} \omega\indices{^\mu_\nu} x^\nu \partial_\mu \phi + \mathcal{L} \omega\indices{^\rho_\nu}x^\nu\\
       &= \omega\indices{^\mu_\nu}x^\nu \left[\delta\indices{^\rho_\mu}\mathcal{L} - \diffp{\mathcal{L}}{(\partial_\rho \phi)} \partial_\mu \phi\right]\\
       &= \omega_{\sigma \nu} x^\nu \left[\mathcal{L} g^{\rho \sigma} - \diffp{\mathcal{L}}{(\partial_\rho\phi)} \partial^\sigma \phi\right]\\
       &= \omega_{\nu\sigma} x^\nu T^{\rho \sigma}\\
       &= \frac12 \omega_{\nu \sigma} x^\nu T^{\rho \sigma} + \frac12 \omega_{\sigma \nu} x^\sigma T^{\rho \nu}\\
       &= \frac12 \omega_{\nu \sigma} \left(x^\nu T^{\rho \sigma} - x^\sigma T^{\rho \nu}\right),
    \end{align*}
    where \(T^{\rho \sigma}\) is the stress-energy tensor. As the Lorentz transformation components are arbitrary, the tensor \(M^{\nu \rho \sigma} = x^\nu T^{\rho \sigma} - x^\sigma T^{\rho \nu}\) is conserved with \(\partial_\rho M^{\nu \rho \sigma} = 0\).
\end{proof}
